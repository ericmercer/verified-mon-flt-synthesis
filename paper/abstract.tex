Cyber-physical systems, such as avionics, must be tolerant to
cyber-attacks in the same way they are tolerant to random faults: they
either gracefully recover or safely shut down as requirements dictate.
The DARPA Cyber Assured Systems Engineering program is developing
tools for design, analysis, and verification that enable systems
engineers to design-in cyber-resiliency in a Model-Based Systems
Engineering environment.  This paper describes automated model
transformations that introduce high-assurance cyber-resiliency
components into a system, in particular filters and monitors that
prevent malicious input and detect supply chain attacks, respectively.
A formal specification defines each high-assurance component, and is
used to verify that the component addresses system level cyber
requirements.  Implementations for these high-assurance components are
directly synthesized from their specifications, and are automatically
proven to preserve the exact meaning of the specifications all the way
down to the binary code level.  The model transformations are
integrated into the Open Source AADL Tool Environment (OSATE).  The
paper further reports on a case study applying cyber-resiliency model
transformations to a UAV system that uses the Air Force Research
Laboratory's OpenUxAS services for route planning.  In the case study,
the model transformations add filters, to guard against malformed
input, as well as monitors, to guard against ground station spoofing
and malicious flight plans from OpenUxAS.
