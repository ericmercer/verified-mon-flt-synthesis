Cyber-physical systems, such as avionics, must be tolerant to
cyber-attacks in the same way they are tolerant to random faults: gracefully recover or safely shut down as requirements dictate.
\brfcs\ is a model-based systems engineering environment that extends the Open Source AADL Tool Environment to add new design, analysis, and code generation capabilities for building cyber-resilient systems.
This manuscript describes a mechanism implemented by \brfcs\ to create, and insert, high-assurance components into a system for cyber-resiliency.
Such components can be filters that guard against malformed input or monitors that guard against spoofing and other malicious behavior.
A formal specification in the form of a code contract defines each high-assurance component and is used with test contracts for testing the specified behavior.
Once tested, the code contract is used to verify that the added component addresses system-level cyber requirements.
Implementations for these
high-assurance components are directly synthesized from their code
contracts and are backed up by proofs showing that high-level
specifications map in a meaning-preserving way to code.
The manuscript further reports on a case study that used \brfcs\ to cyber-harden a UAV system by inserting high assurance components to isolate the open source Air Force Research Laboratory's OpenUxAS services for
route planning.  The case study
demonstrates that synthesizing implementations from code contracts is feasible in real-world systems
engineering.
