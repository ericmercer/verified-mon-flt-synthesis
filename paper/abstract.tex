Cyber-physical systems, such as avionics, must be tolerant to
cyber-attacks in the same way they are tolerant to random faults: they
either gracefully recover or safely shut down as requirements dictate.
The DARPA Cyber Assured Systems Engineering program is developing
tools for design, analysis, and verification that enable systems
engineers to design-in cyber-resiliency in a Model-Based Systems
Engineering environment.  This paper describes automated model
transformations that introduce high-assurance cyber-resiliency
components into a system, in particular filters and monitors that
prevent malicious input and detect supply chain attacks, respectively.
A formal specification in the form of a code-contract defines each high-assurance component and is
used to verify that the component addresses system level cyber
requirements.  Implementations for these high-assurance components are
directly synthesized from their code-contracts, and are automatically
proven to preserve the exact meaning of the code-contract all the way
down to the binary code level.  The model transformations are
integrated into the Open Source AADL Tool Environment (OSATE).  The
paper further reports on two case studies, one by formal method experts 
and the other by non-experts,
applying cyber-resiliency model
transformations to two different systems. 
One is a UAV system that uses the Air Force Research
Laboratory's OpenUxAS services for route planning.
The other is a tablet used for inflight communication between pilots in an Apache helicopter.
In each case study, the model transformations add filters, to guard against malformed
input, as well as monitors, to guard against spoofing and other malicious behavior.
The case studies show the MBSE is feasible in real-world systems engineering.