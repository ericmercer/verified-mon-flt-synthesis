\newjunk{%
Avionics need to be engineered to be cyber-resilient meaning that systems are able to detect and recover from attacks or safely shutdown.
%
As there are few development tools for cyber-resiliency, designers rely on guidelines and checklists, sometimes missing vulnerabilities until late in the process where remediation is expensive.
%
Our solution is a model-based approach with cyber-resilience-improving transforms that insert high-assurance components such as filters to block malicious data or monitors to detect and alarm anomalous behavior.
%
Novel is our use of model checking and a verified compiler to specify, verify, and synthesize these components.
%
We define \emph{code contracts} as formal specifications that designers write for high-assurance components, and \emph{test contracts} as tests to validate their behavior.
%
A model checker proves whether or not code contracts satisfy test contracts in an iterative development cycle.
%
The same model checker also proves whether or not a system with the inserted components, assuming they adhere to their code contracts, provide the desired cyber-resiliency for the system.
%
We define an algorithm to synthesize implementations for code contracts in a semantics-preserving way that is backed by a verified compiler.
%
The entire workflow is implemented as part of the open source \emph{\brfcs} toolkit.
%
We report on our experience using \brfcs\ with a case study on a UAV system that is transformed to be cyber-resilient to communication and supply chain cyber attacks.
%
Our case study demonstrates that synthesizing correct implementations from code
contracts is feasible in real-world systems engineering.
}

\oldjunk{%
We have developed a workflow for creating and inserting high-assurance
components that enforce cyber-resiliency requirements into a
model-based systems engineering environment.  Example high-assurance
components are filters, which guard against malformed input, and
runtime monitors, which guard against spoofing and other malicious
behavior. A formal specification in the form of a \emph{code contract}
defines each high-assurance component and is developed with the
support of \emph{test contracts} for testing the specified behavior.
Model checking is used to verify that an inserted high-assurance
component indeed addresses system-level cyber
requirements. Implementations for high-assurance components are
directly synthesized from their code contracts and are backed up by
proofs showing that high-level specifications map in a
semantics-preserving way to code generated by a verified compiler.  We
report on a case study that cyber-hardened a UAV system by inserting
high assurance components to harden the open source Air Force Research
Laboratory's OpenUxAS services for route planning.  The case study
demonstrates that synthesizing correct implementations from code
contracts is feasible in real-world systems engineering.
}