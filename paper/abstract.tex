Cyber-physical systems, such as avionics, must be tolerant to
cyber-attacks in the same way they are tolerant to random faults: they
should either gracefully recover or safely shut down as requirements dictate.
%
%% The DARPA Cyber Assured Systems Engineering program is developing
%% tools for design, analysis, and verification that enable systems
%% engineers to design-in cyber-resiliency in a Model-Based Systems
%% Engineering (MBSE) environment.
%
This paper describes automated model transformations that introduce
high-assurance cyber-resiliency components into a system; in
particular, filters that guard against malformed input, as well as
monitors that guard against spoofing and other malicious behavior. A
formal specification in the form of a code contract defines each
high-assurance component and is used to verify that the component
addresses system-level cyber requirements. Implementations for these
high-assurance components are directly synthesized from their code
contracts, and are backed up by proofs showing that high-level
specifications map in a meaning-preserving way to code. The model
transformations are integrated into the Open Source AADL Tool
Environment (OSATE). 
We further report a case study applying the model
transformations to a UAV system
utilizing the Air Force Research Laboratory's OpenUxAS services for
route planning.  The case study
demonstrates that our MBSE approach is feasible in real-world systems
engineering.


% We further report on two case studies, one by
% formal method experts and the other by non-experts, applying model
% transformations to two different systems to improve cyber resiliency
% using high-assurance filters and monitors.  One is a UAV system
% utilizing the Air Force Research Laboratory's OpenUxAS services for
% route planning.  The other is a tablet used for situational awareness
% in a CH-47F Chinook helicopter application.  The case studies
% demonstrate that our MBSE approach is feasible in real-world systems
% engineering.
