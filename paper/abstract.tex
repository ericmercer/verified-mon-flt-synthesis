Cyber-physical systems, such as avionics, must be tolerant to
cyber-attacks in the same way they are tolerant to random faults: they
must gracefully recover, or safely shut down, as requirements dictate.
We have developed a workflow for creating, and inserting,
high-assurance components implementing cyber-resiliency into a
model-based systems engineering environment.  Example high-assurance
components are filters, which guard against malformed input, and
runtime monitors, which guard against spoofing and other malicious
behavior. A formal specification in the form of a \emph{code contract}
defines each high-assurance component and is developed with the
support of \emph{test contracts} for testing the specified behavior.
Once tested, model checking used to verify that the added
high-assurance component indeed addresses system-level cyber
requirements.  Implementations for these high-assurance components are
directly synthesized from their code contracts and are backed up by
proofs showing that high-level specifications map in a
semantics-preserving way to code generated by a verified compiler.  We
report on a case study that cyber-hardened a UAV system by inserting
high assurance components to harden the open source Air Force Research
Laboratory's OpenUxAS services for route planning.  The case study
demonstrates that synthesizing correct implementations from code
contracts is feasible in real-world systems engineering.
