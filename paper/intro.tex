Set the context for this work: define system level theorem that we intend to prove with the added components? Big picture of the overall goal of the project.

Components (complexity (and expressiveness) increases along the way):
* Filter: takes a stream of data and delivers a new stream of data where the property is enforced--reasons over an infinite stream of date. Connects to the system correctness over traces. Predicate on a single piece of data, paying no attention to previous history--no temporal awareness, deciting if it passes or not. Can be very secure and very efficient.
* Monitor: Captures a relation over time on the data. Is able to reason about temporal properties. Supports multiple inputs and is able to do arbitrary computation for complex output over the inputs.

The attestation gate is an aside to show how the tools we have are rich enough to create other complex things that blend the two: filter and monitor.

This paper addresses only the verified components. Here is what we verify:

* The resulting assembly code exactly implements the specification--preserves the meaning of the original specification in the actual assembly code.

-- Scheduling and Trace Relationships: we assume a stream of correct data-types and the existence of some scheduler to schedule components appropriately. In the later section we give an example of a system that uses seL4 with a pacer to schedule over different domains to provide memory isolation. Scheduling atd the relation to the trace semantics is not part of this paper.

\cite{kalhauge2018sound}

