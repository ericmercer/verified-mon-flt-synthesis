In recent years, aerospace stakeholders have realized that avionics systems are subject to possible cyber-attacks just like other cyber-physical systems.  Thus, in addition to being fault-tolerant, safety-critical avionics systems must also be {\em cyber-resilient}. Cyber-resiliency means that the system is tolerant to cyber-attacks just as safety-critical systems are tolerant to random faults: they recover and continue to execute their mission function, or safely shut down, as requirements dictate.

Unfortunately, systems engineers are currently given few development tools to help answer even basic questions about potential vulnerabilities and mitigations, and instead rely on process-oriented checklists and guidelines.  Cyber vulnerabilities are often discovered during penetration testing late in the development process; or worse yet, they may be discovered only after the product has been fielded, necessitating extremely expensive and time-consuming remediation. This is not a sustainable development model.

The DARPA Cyber Assured Systems Engineering (CASE) project is targeted at developing tools for design, analysis, and verification that enable systems engineers to {\em design-in} cyber-resiliency for complex cyber-physical systems.\footnote{This work was funded in part by the Defense Advanced Research Projects Agency (DARPA).  The views expressed are those of the authors and do not reflect the official policy or position of DARPA or the U.S. Government.}
We have developed a Model-Based Systems Engineering (MBSE) environment called {\em BriefCASE} which is based on the Architecture Analysis and Design Language (AADL)~\cite{aadl}.  BriefCASE extends the Open Source AADL Tool Environment (OSATE) to add new design, analysis, and code generation capabilities for building cyber-resilient systems.

BriefCASE incorporates model-level cyber analysis tools (presently
GearCASE~\cite{gearcase2020} and DCRYPPS~\cite{dcrypps2019}) which can
examine AADL models for potential vulnerabilities and suggest
cyber-security requirements to mitigate them.  A library of
architectural transforms guides the system engineer through automated
model transformations that modify the architecture to address these
requirements, possibly inserting new high-assurance components into
the system.  Implementations for the new components are synthesized
from formal specifications using
SPLAT~\cite{slind-hcss2020},~\cite{formal-filter-synth-langsec}
(Semantic Properties for Language and Automata Theory).

Formal
verification that the transformed system model meets its requirements
is accomplished via AGREE~\cite{agree2013} (Assume Guarantee Reasoning
Environment).
%The AGREE analysis \emph{assumes} properties on the inputs of a given component of the system, and attempts to formally prove the conjectured \emph{guarantees} of the output.
AGREE is a {\em compositional assume-guarantee} style model checker
for AADL models that attempts to prove properties about one layer of
an architecture using properties allocated to its subcomponents.
Cyber-resilient code implementing the verified model is then
automatically generated using the High Assurance Modeling and Rapid
Engineering for Embedded Systems (HAMR) toolkit~\cite{hamr}.  If
desired, this code can be targeted to the formally verified seL4
secure microkernel~\cite{sel4-2009}.

%% DSH  This paragraph seems mostly duplicative of the paragraph above; attempted to merge the two.
%A novel aspect of the BriefCASE approach is the ability to automatically transform the model to insert high-assurance %components to satisfy requirements added to mitigate cyber vulnerabilities. In this work, the behavior of the system is %modeled by contracts on components expressed in the AGREE modeling language. These contracts are modified as a result %of the cyber-vulnerability analysis with the new requirements for mitigation.  The requirements assume properties on the %input and guarantee properties of the output that reflect a hardened system. In the absence of any transformation to %implement such requirements, AGREE fails to prove the system meets these new requirements.

Two cyber-resiliency transformations are discussed: (1) the insertion
of a filter to prevent malformed data from a malicious actor from
being propagated to downstream components, and (2) the insertion of a
monitor to detect (and alert) unexpected behaviors arising from
untrusted components. These transformations not only change the
architecture of the model by adding in new components; they also
generate a formal specification in the form of a \emph{code contract} in the AGREE language for their intended behavior. 
A code contract describes the computation of a high-assurance component and not just what it computes.
These code contracts
are sufficient for model checking to prove that---due to the newly
included high-assurance components---the hardened system meets its
cyber-resiliency requirements.

Another novel aspect of the approach is the synthesis of the AGREE
code contracts for the high-assurance components to CakeML, a verified
compiler implementation for the functional programming language
ML \cite{cakeml}. This paper describes in detail the synthesis path
from code contracts to CakeML code, providing a formal framework in
which to argue correctness. CakeML then provides a verified
compilation path to several different target binaries (and also
proving that the meaning of the CakeML source code is exactly
preserved in the final binaries). Assuming that the execution schedule
of the deployed cyber-hardened system is as intended by the AADL
model, and that the HAMR-generated communication fabric delivers
messages between components as expected, the AGREE model checking
results then hold for the deployed system, \ie, the system will detect
and prevent the indicated cyber-vulnerabilities over all possible
finite inputs. Preliminary work has shown how to lift this result to
infinite input traces as these systems are inherently reactive and
intended to run forever~\cite{case-verified-filter}, \cite{cakeml-space-cost}.

Our approach currently applies to common cyber-vulnerabilities, such
as overflow, lack of input validation, supply-chain issues, and
identity verification; however, other cyber-vulnerabilities such as
side-channel attacks and denial of service are not yet dealt with in
our work.  Here, we do not report on the invention of a new type of
filter or monitor, in terms of capability; instead, our contribution
is in the automated synthesis of security-improving components from
formal specifications, and a means to show that the synthesis is
correct.

The approach is illustrated by a simple example in
\secref{sec:example}. AGREE verification and specification in the AGREE
language are presented in \secref{sec:agree} and \secref{agree-semantics}.
Code contracts are defined in \secref{sec:code-contracts} with a method
for testing code contracts in the AGREE framework before synthesis given
\secref{sec:testing}.
The synthesis pathway is covered in Section~\ref{sec:synthesis}. A
case study applying these transformations to an Unmanned Aerial
Vehicle (UAV) system that uses the Air Force Research Laboratory's
OpenUxAS services for route planning is presented in
Section~\ref{sec:case-study}. Here the transforms add filters to guard
against malformed input and monitors to guard against ground station
spoofing and malicious flight plans from OpenUxAS. The case study
system is significantly more complex than the simple example and shows
the viability of the modeling approach to a full-scale industrial
design.

The BriefCASE tools are open source and publicly
available \cite{fmide}, as is the motivating example \cite{repo}, and
the full UxAS model with its deployment in seL4 \cite{phase2, camkes}.
Videos demonstrating the use of the BriefCASE tools to build the UAV
example presented in Section~\ref{sec:case-study} are also
available \cite{case}.
