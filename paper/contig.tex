The formal specification of a component, and the synthesis of that specification, relies on \emph{contiguity types} (cite contiguity). A contiguity type is a self-describing specification for messages. Its formalism has basis in formal languages. Similar to how a regular expression implies a set of words that form its language, so does a contiguity type specification imply a set of messages for its language where a message is a finite sequence of contiguous bytes (e.g., a string). 

What makes contiguity type specification more expressive than regular expressions is that it is self-describing meaning that the contents of the message itself may determine the rest of the message. An example is the \texttt{AutomationResponse} from the system in the previous section with its contiguity type specification.
{\small
\begin{verbatim}
  {TaskID : i64
   Length : u8
   Waypoints : Waypoint[Length]
  }
\end{verbatim}
}
\noindent The \texttt{Waypoints} array size depends on the value of \texttt{Length} so the actual number of bytes in the message depends on the contents of the message itself. 

The type specifications may also carry meta-information about the contents of the message.
{\small
\begin{verbatim}
  {Latitude : float
   lt-rng : Assert (-90 <= Latitude <= 90) 
   Longitude : float
   lng-rng : Assert (-180 <= Longitute <= 180)
   Altitude : float
   a-rng : Assert (10000 <= Altitude <= 15000)
  }
\end{verbatim}
}
\noindent Here the specification encodes the allowed ranges for each field of the waypoint. These can be checked while constructing a message from a sequence of bytes.

Every contiguity type specification has a corresponding \emph{matcher} that when given a message string returns true or false if that message belongs to the language of the specification. If the message does belong to the language, an \emph{environment} is provided to access each part of the message. An environment, $\theta: \lval \mapsto \konst{string}$ binds \emph{L-values} to strings. An L-value is an expression that can appear on the left hand side of an assignment. Alpha renaming is assumed so that every $\lval$ is unique.

The matcher itself is synthesized from the specification to CakeML. The synthesis includes a corresponding proof that the matcher recognizes the language of the specification, and the resulting environment, $\theta$, from a matched message produces the same message as the one matched when serialized.
