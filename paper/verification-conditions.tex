
\newcommand{\globally}{\konst{Always}}
\newcommand{\historically}{\konst{Hist}}
\newcommand{\assumes}{\ensuremath{A}}
\newcommand{\guarantees}{\ensuremath{P}}
\newcommand{\inputs}{\ensuremath{I}}
\newcommand{\outputs}{\ensuremath{O}}
\newcommand{\components}{\ensuremath{C}}
\newcommand{\component}{\ensuremath{c}}

The \agr\ specification language, and semantics, are inspired by the
Lustre language \cite{10.1145/41625.41641}. Lustre semantics are
synchronous data-flow where the inputs and outputs of components are
streams.  Informally, streams assign values to expressions at each
point in time.  \agr\ contracts express relationships between input
and output streams by assuming a set of allowed input streams and then
guaranteeing a corresponding set of allowed output streams.

\subsection{Components}
A system implementation is described by its ports, its subcomponents,
and the connections between ports on different subcomponents.  Ports
in \agr\ are inherited from AADL.  There are three kinds of port:
\emph{event} ports, \emph{data} ports, and \emph{event data} ports.
An event port carries no data and only signals occurrence (or not) of the
event. A data port is a sampled port: the data can be read at anytime.
An event data port indicates when data is present with an event.

Data flows through the subcomponents of a system in dependency order,
with inputs being propagated to outputs through all contracts until
they stabilize \ie, cannot propagate further. Therefore, subcomponent
contracts, and thus the top-level model, must be acyclic.  Cyclic
systems are made acyclic by breaking cycles with delays.  Once the
data propagation has stabilized, the model proceeds to the next input
data in the input streams. It is worth noting that the semantics do
not model computation or communication delay: the output of one
contract is seen at the input of any downstream contract in the same
step of the input data stream.



\subsection{Contracts and verification conditions}

From the system and component contracts, \agr\ generates a set of
verification conditions to show that a system's component
implementation is correct~\cite{agree2013}.  The \agr\ model checker
is then invoked to prove or disprove the verification
conditions. The contracts and verification conditions are expressed in
\emph{past-time linear temporal logic} (PLTL) \cite{10.1093/jigpal/8.1.55}.

PLTL is a logic enhanced with temporal operators able to reason about
the truth values of formulas through time.  It combines past-time
operators that reason over the history of computation up to a point
(PTLTL) and future-time operators that reason over the computation
that follows that point (LTL).  Its semantics are defined relative to
a point in time $i$, a finite trace of system states $\pi = s_0, s_1,
\ldots, s_i$, and some logic formula $f$.

The two PLTL operators necessary for the \agr\ generated verification
conditions are $\globally$ that looks forward in time along
the trace and $\historically$ (historically) that looks backward in
time along the trace.  These are defined as
\begin{eqnarray*}
 (\pi, i) \models \globally(f) & \iff & \forall j \ge i, (\pi, j) \models f \\
(\pi, i) \models \historically(f) & \iff & \forall 0 \le j \le i, (\pi, j) \models f
\end{eqnarray*}
The $\models$-operator is read as \emph{satisfies}.  A trace at a
moment in time satisfies $\globally(f)$ if and only if it satisfies
$f$ in the current and all future states of $\pi$.  $\globally(f)$ is
invariant from the current moment into the future and $\historically$ is
invariant from the beginning of the trace to the current moment.

A \emph{system} is a tuple $(\inputs, \outputs, \assumes,
\guarantees,C)$, where $\inputs$ is the input set, $\outputs$ is the
output set, $\assumes$ is the set of assumptions, $\guarantees$ is the
set of guarantees, and $C$ are subcomponents.  Assumptions and guarantees only use the past-time operators from PLTL. A subcomponent
$\component$ is, hierarchically, also a system, and may be designated
by its own tuple $(\inputs_\component, \outputs_\component,
\assumes_\component, \guarantees_\component, C_c)$.  From the
components and their connections, $\mathbb{I}_\component$ is defined
to be the set of components providing input to some component
$\component$ in the system, and $\mathbb{O}$ is defined to be the set
of components that provide the output for the system.

A system is
\emph{correct} if and only if for all components $c \in C$ the
following two verification conditions hold:
\begin{equation}\label{eq:assumes}
            \globally(\historically(\assumes \wedge
            \bigwedge_{\component^\prime \in \mathbb{I}_\component} P_{\component^\prime})
            \implies \assumes_\component)
\end{equation}
\begin{equation}\label{eq:guarantees}
            \globally(\historically(\assumes \wedge
            \bigwedge_{\component^\prime \in \mathbb{O}} \guarantees_{\component^\prime})
            \implies \guarantees)
\end{equation}
The conditions from \eqref{eq:assumes} verify the input assumptions on
each component under the system assumptions and upstream component
guarantees.  They check if the component guarantees and system
assumptions are strong enough to imply input assumptions on all
immediate downstream components.  The condition in
\eqref{eq:guarantees} checks the output guarantees of the system under
the system assumptions and component guarantees that provide the
output.  It checks if the guarantees on components providing primary
outputs are strong enough to imply the system guarantees.

If all the verification conditions hold (\agr\ uses $k$-inductive
model checking to automatically prove or disprove each generated
verification condition), then the system is said to be \emph{correct},
meaning that the system composition meets input assumptions at each
input as well as the guarantees on the system output. A consequence of
this result is that $\globally(\historically(\assumes) \implies
\guarantees)$ holds for the system contract.  Therefore the Liskov
substitution principle applies and the implementation is a safe
substitution for the system contract \cite{10.1145/62139.62141}.

The expanded property lists in \figref{fig:example-certificate} and
\figref{fig:hardened-certificate} are the results from verifying or
disproving the above verification conditions.  The additional
unexpanded results at the bottom of the figures prove
\emph{self-consistency} in the contracts.
A contract is self-consistent if it does not guarantee two different values in the same moment of time, \ie, the contract is not self-contradicting.
\agr\ generates additional verification
conditions that prove each component contract, and the composition of
contracts, self-consistent.
