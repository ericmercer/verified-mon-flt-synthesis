
\newcommand{\globally}{\ensuremath{\mathbf{G}}}
\newcommand{\historically}{\ensuremath{\mathbf{H}}}
\newcommand{\assumes}{\ensuremath{A}}
\newcommand{\guarantees}{\ensuremath{P}}
\newcommand{\dispatch}{\ensuremath{\mathit{dispatch}}}
\newcommand{\complete}{\ensuremath{\mathit{complete}}}
\newcommand{\same}[1]{\ensuremath{\mathit{same}(#1)}}
\newcommand{\inputs}{\ensuremath{I}}
\newcommand{\outputs}{\ensuremath{O}}
\newcommand{\system}{\ensuremath{S}}
\newcommand{\components}{\ensuremath{C}}
\newcommand{\component}{\ensuremath{c}}
\newcommand{\schedule}{\ensuremath{\phi}}
\newcommand{\valid}{\ensuremath{\mathit{valid}}}
\newcommand{\dpred}{\ensuremath{\delta^\phi}}
\newcommand{\dispred}{\ensuremath{\mathbb{D}^\phi}}
\newcommand{\compred}{\ensuremath{\mathbb{C}^\phi}}
\newcommand{\dispredp}{\ensuremath{\mathbb{D}^{\phi\prime}}}
\newcommand{\compredp}{\ensuremath{\mathbb{C}^{\phi\prime}}}

AGREE generates, and then proves or disproves with model checking, a set of verification conditions to show that a system's component implementation is correct~\cite{agree2013}.
The verification conditions are formalized in \emph{past-time linear temporal logic} (PLTL).
PLTL is a first-order predicate calculus with temporal operators to reason about the truth values of formulas through time.
Its semantics are defined relative to a finite trace of system states $\pi = s_0, s_1, \ldots$, and a moment of evaluation in the trace, $i$.

The two PLTL operators necessary for the AGREE generated verification conditions are $\globally$ (globally) that looks forward in time along the trace and $\historically$ (historically) that looks backward in time along the trace.
These are defined as
\begin{eqnarray*}
 (\pi, i) \models \globally(f) & \iff & \forall j \ge i, (\pi, j) \models f \\
(\pi, i) \models \historically(f) & \iff & \forall 0 \le j \le i, (\pi, j) \models f
\end{eqnarray*}
The $\models$-operator is read as \emph{satisfies}.
A trace at a moment in time satisfies $\globally(f)$ if and only if it satisfies $f$ in the current and all future states of $\pi$.
Globally is invariant from the current moment into the future and historically is invariant from the beginning of the trace to the current moment.

A \emph{system} $\system = (\inputs, \outputs, \assumes,
\guarantees)$, where $\inputs$ is the input set, $\outputs$ is the
output set, $\assumes$ is the set of assumptions, and $\guarantees$ is
the set of guarantees.  A system is composed of a set of
\emph{components}, $\components$, together with their
inter-connections.  A component $\component$ is, hierarchically, also
a system, and may be designated by its own tuple
$(\inputs_\component, \outputs_\component, \assumes_\component,
\guarantees_\component)$.  From the components and connections,
$\mathbb{I}_\component$ is defined to be the set of components
providing input to some component $\component$ in the system, and
$\mathbb{O}$ is defined to be the set of components that provide the
output for the system.  A system $S$ is \emph{correct} if and only if for all
components $c \in C$:
\begin{equation}
            \globally(\historically(\assumes \wedge
            \bigwedge_{\component^\prime \in \mathbb{I}_\component} P_{\component^\prime})
            \implies \assumes_\component)
\end{equation}
\begin{equation}
            \globally(\historically(\assumes \wedge
            \bigwedge_{\component^\prime \in \mathbb{O}} \guarantees_{\component^\prime})
            \implies \guarantees)
\end{equation}
Condition (1) verifies the input assumptions on each component under
the system assumptions and upstream component guarantees.  It checks
if the component guarantees and system assumptions are strong enough
to imply input assumptions on all immediate downstream components.
Condition (2) checks the output guarantees of the system under the
system assumptions and component guarantees that provide the output.
It checks if the guarantees on components providing primary outputs
are strong enough to imply the system guarantees.

AGREE uses $k$-inductive model checking to prove or disprove each
generated verification condition.  If all the verification conditions
hold, then the system is said to be \emph{correct}, meaning that the
system composition meets input assumptions at each input as well as
the guarantees on the system output. A consequence of this result is
that $\globally(\historically(\assumes) \implies \guarantees)$ holds
for the system contract. As such, the system contract is a sound
abstraction of the system implementation.

The expanded property lists in \figref{fig:example-certificate} and
\figref{fig:hardened-certificate} are the results from verifying or
disproving the above verification conditions.  The additional
unexpanded results at the bottom of the figures prove
\emph{self-consistency} in the contracts.  It is not uncommon to
accidentally write contracts that are self-contradicting.  For
example, a contract may guarantee an output be two different values in
the same moment of time.  AGREE generates additional verification
conditions that prove each component contract, and the composition of
contracts, self-consistent.\footnote{KLS:Is this the same as
satisfiable?}

\subsection{Syntax and semantics of AGREE}
\label{agree-semantics}

The syntax of AGREE is essentially that of quantifier-free first order
logic (FOL) supplemented with a few temporal operators. The terms
(\emph{e}) are arithmetic expressions built from variables ($v$) and
numeric and boolean literals ($c$), while formulas (\emph{b}) are
built using logical connectives from atomic formulas (\emph{a}) based
on the familiar comparison operators.
\[
\begin{array}{rcl}
e & ::= & v \mid c \mid e \;\set{+,*,/}\; e \\
a & ::= & e\; \set{=,<}\; e \\
b & ::= & v \mid c \mid a \mid \neg b
            \mid b \; \set{\land,\lor,\imp,\iff}\; b
\end{array}
\]

There is also a conditional, $\itelse{b}{(-)}{(-)}$, for both terms
and formulas. Lastly, there are temporal operators $\konst{pre}(-)$,
delay $(-) \to (-)$, and $\konst{Hist}(-)$.

The semantics of terms and formulas is in terms of \emph{streams of
values}. Values encompass at least booleans and numbers, but can be
readily extended to include records and arrays. A value stream is a
total function from time (natural numbers) to values:
\[
 \konst{stream} = \mathbb{N} \to \konst{value}
\]
Given an \emph{environment} $E : \konst{name} \mapsto \konst{stream}$
binding variable names to value streams, the semantics $\sem{-}^E_t$
of terms and formulas, omitting the temporal operators, is
conventional. The value of a variable $v$ at time $t$ is found by
looking up the stream bound to $v$ in $E$ (call it $s$) and extracting
the value of $s$ at time $t$.

\[
\begin{array}{rcl}
\sem{v}^E_t & = & E(v)(t) \\
\sem{c}^E_t & = & c \\
\sem{e_1 + e_2}^E_t & = & \sem{e_1}^E_t + \sem{e_2}^E_t \\
   & \cdots & \\
\sem{b_1 \land b_2}^E_t & = & \sem{b_1}^E_t \land \sem{b_2}^E_t \\
   & \cdots & \\
\end{array}
\]

The temporal operators deal with time in more significant ways. The
value of $\konst{pre}(e)$ at time $t$ is the value of $e$ at time
$t-1$ (at time zero, \konst{pre} is undefined).  A delay $e_1 \to e_2$
temporally ``shifts'' $e_2$ by means of prepending the first element
of $e_1$ to it.

\[
\begin{array}{rcl}
\sem{\konst{pre}(e)}^E_t & = & \sem{e}^E_{t-1}, \mathrm{when}\ t > 0 \\
\sem{e_1 \to e_2}^E_t & = & \itelse{t=0}{\sem{e_1}^E_0}{\sem{e_2}^E_t} \\
\sem{\konst{Hist}(b)}^E_t & = & \forall n \leq t.\; \sem{b}^E_n
\end{array}
\]

Although basic, these definitions can be used to define higher-level
operators from PTLTL, such as \konst{Once} and \konst{Since}.

\subsection{Code contracts}

Generally, AGREE specifications do not describe the computation
that a component performs. This is entirely by design: AGREE is
intended to reason about component behavior solely at the
specification level. However, the syntax of AGREE specifications
provides enough expressiveness to support the notion of a
\emph{code contract}: a component contract from which an implementation
can be extracted.

\begin{definition}[Output guarantee]
An \emph{output guarantee} is a stylized guarantee that fully
specifies the data written to an output port. There are three
possibilities according to whether the output port $p$ is
a \konst{data} port, an \konst{event} port, or an \konst{event data}
port:
\[
\begin{array}{ll}
\konst{data}: &  p = \mathit{e} \\
\konst{event}: &  \konst{event} (p) = \mathit{b} \\
\konst{event data}: & \itelse{b}{\konst{event} (p) \land p = e}{\neg \konst{event}(p)} \\
\end{array}
\]
\end{definition}

\begin{definition}[Code contract]
A component of the form $(I_c,O_c,A_c,P_c)$ is a \emph{code contract}
if $\mathit{Eqs} \cup G \subseteq P_c$, where
\[\mathit{Eqs} = \set{v_1 = e_1, \cdots , v_n = e_n} \] is a non-empty set
of \konst{eq} statements and $G$ is the set of output guarantees, one for
each element of $O_c$. In the interpretation as code, the order of
elements of $\mathit{Eqs}$ is important, and is simply taken to be the
occurrence order of the \konst{eq} statements in the syntax. Thus we
will work with the
\emph{list} of equations $\mathit{Eqs} = [v_1 = e_1; \cdots ; v_n = e_n]$.
\end{definition}
