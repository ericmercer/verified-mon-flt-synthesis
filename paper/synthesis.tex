\emph{Discussion on mapping program from stream environment to logic
function and then to CakeML}

Synthesis maps from model and specifications to code. The synthesis
algorithm traverses the system architecture looking for occurrences of
filter and monitor specifications;  for each such occurrence it
generates a CakeML program. In the following, we examine both filter
and monitor synthesis. The latter is typically much more involved, and
we will therefore devote more attention to it.

\subsection{Filter Generation}

A filter is intended to be simple, although it may make deep semantic
checks. A filter has one input port and one output; messages on the
input that the filter policy admits pass unchanged to the output port;
all others are dropped (not passed on). We have investigated two kinds
of filter. In the first, a relatively shallow scan of the input
suffices to enforce the policy. For example, we have used the
expressive power of Contiguity Types \cite{contiguity-types} to
enforce \emph{lightweight} bounds constraints on GPS coordinates in
UxAS messages. On the other hand, a filter may need to parse the input
buffer into a data structure specified in AGREE and apply a
user-defined \emph{wellformedness} property, also specified in AGREE,
to the data. Arbitrarily complex wellformedness checks can be made in
this way. \figref{fig:filter-spec} shows a combination where the
checking specified by {\small\verb+WELL_FORMED_AUTOMATION_RESPONSE+}
depends on an underlying check specified by the contiguity type
checking bounds on waypoints.

The verdict of a filter is made and performed within one thread
invocation. Thus, in its given time slice, the
following steps must be completed:

\begin{enumerate}

\item The filter checks to see if there is any input available.  If there is none
then it yields control; otherwise:

\item The input is read (and parsed if need be);

\item The wellformedness predicate is evaluated on the input;

\item If the predicate returns \konst{true} then the input buffer
 is copied to the output, otherwise no action is taken; and

\item The filter yields control.
\end{enumerate}

\begin{remark}[Partiality]

Partiality is an important consideration: steps 2 and 3 above can
fail; the data might not be parseable or the wellformedness
computation could be badly written and fail at runtime. In such cases,
the filter should recover and yield control without passing the input
onwards. In these cases, the filter is behaving as it should, but we
must also guard against situations in which a \emph{correctly
  specified} filter fails at runtime. This kind of defect arises when
the filter \emph{ought} to accept a message, but lack of resources
results in the filter failing to do so. For example, the parse of a
message might need more space than has been allocated; another example
could be if the time slice provided by the scheduler is too short for
the wellformedness computation to finish. Thus resource bounds need to
be included in the correctness argument.

\end{remark}


\newsavebox{\contig}
\begin{lrbox}{\contig}
\begin{lstlisting}[style=myML]
  Waypoint =
    {Latitude  : f64
     Longitude : f64
     Altitude  : f32
     Check     : Assert
      (~90.0 <= Latitude and Latitude <= 90.0 and
       ~180.0 <= Longitude and Longitude <= 180.0 and
       1000.0 <= Altitude and Altitude <= 15000.0)}

  AutomationResponse =
    {TaskID : i64
     Length : u8
     Waypoints : Waypoint [3]}

 fun WELL_FORMED_AUTOMATION_RESPONSE(aresp) =
   (forall wpt in aresp.Waypoints, WELL_FORMED_WAYPOINT(wpt))
   and ... ;
\end{lstlisting}
\end{lrbox}

\begin{figure}
  \begin{center}
    \begin{tabular}{c}
      \scalebox{0.60}{\usebox{\contig}}
    \end{tabular}
  \end{center}
  \caption{Filter specification.}
  \label{fig:filter-spec}
\end{figure}


\newsavebox{\cml}
\begin{lrbox}{\cml}
\begin{lstlisting}[style=myML]
fun filter_step () =
 let val () = Utils.clear_buf buffer
     val () = API.callFFI "get_input" "" buffer
 in
    if WELL_FORMED_AUTOMATION_RESPONSE buffer
    then
      API.callFFI "put_output" buffer Utils.emptybuf
    else print"Filter rejects message.\n"
end
\end{lstlisting}
\end{lrbox}

\begin{figure}
  \begin{center}
    \begin{tabular}{c}
      \scalebox{0.60}{\usebox{\cml}}
    \end{tabular}
  \end{center}
  \caption{Synthesized CakeML for the filter.}
  \label{fig:filter-cakeml}
\end{figure}

The contiguity type specification and wellformedness predicate for
the filter are shown in \figref{fig:filter-spec} and the synthesized
CakeML code is in \figref{fig:filter-cakeml}. The code is called at
dispatch by the scheduler. The \texttt{API.callFFI} is the link to the
communication fabric to capture input and provide output. The body of
the function restates the filter contract to make the appropriate
assignments in a way that matches the truth value of the predicate in
the filter guarantee.  The auto-generated AGREE specification raises
an alert output when the relation is violated.

\subsection{Monitor Generation}


Monitors are intended to track and analyze the externally visible
behavior of system components through time. Therefore, they require
more extensive computational ability than filters. In particular, our
basic notion of a monitor is that it embodies a predicate over its
input and output streams, and is able to access the value of a stream at any
earlier point in time, if necessary. Monitors commonly use state to
keep track of earlier values, unlike filters which, for us, are
typically stateless components. (However, there is nothing in our
approach that forbids stateful filters: they can be realized by
monitors.) A monitor specification is mapped by code generation to a
state transformation function of the following abstract type:
\[
\konst{stepFn} : \mathit{input} \times \mathit{stateVars} \to \mathit{stateVars} \times \mathit{output}
\]

The system scheduler \emph{activates} components in some order. It is
an obligation on the system that the scheduler follows some sensible
partial order of component activation and allows each component
sufficient time for its computation.  Activating a monitor component
takes the form of the following pseudo-code, in which the monitor
evaluates the \konst{stepFn} on its current inputs and the current
values of the state variables, returning the new state and the output
values.
\[
\begin{array}{ll}
 \mathit{(i_1,\ldots)} & = \konst{readInputs}(); \\
 (v_1,\ldots) & = \konst{readState}() ; \\
 ({v_1}',\ldots), ({o_1}',\ldots) & = \konst{stepFn} ((i_1,\ldots),(v_1,\ldots)) ; \\
 \multicolumn{2}{l}{\konst{writeState}({v_1}',\ldots);} \\
 \multicolumn{2}{l}{\konst{writeOutputs}({o_1}'\ldots);} \\
\end{array}
\]

\subsubsection{Initialization}

A monitor may need to accumulate a certain minimum number of
observations before being able to make a meaningful assessment of
behavior. Until that threshold is attained, the monitor is essentially
in its \emph{initialization} phase. In order for correct code to
be generated, monitor specifications need to spell out the values of
output ports when in their initialization phases. For example, suppose a
monitor does some kind of differential assessment of inputs at
adjacent time slices, alerting when (say) the measured location of a
UAV at times $t$ and $t+1$ is such that the distance between the two
locations is unusually large. Such a monitor needs two measurements
before making its first judgement, but at the time of its first
output, only one measurement will have been made. The specification
must then explicitly state the correct value for the first output.

\subsubsection{Step function}

The \konst{stepFn} works as follows:

\begin{enumerate}

\item Each input is parsed into data of the type specified by the port
  type;

\item New values for the state variables are computed, in dependency
  order. The discussion above on initialization now comes into
  play. Suppose the variable declarations have the following form:
\[
\begin{array}{l}
  v_1 = i_1 \longrightarrow e_1 \\
  \cdots \\
  v_n = i_n \longrightarrow e_n \\
\end{array}
\]
In the generated code, for the first invocation of \konst{stepFn} only,
the initializations are executed in order:
\[
\begin{array}{l}
  v_1 = i_1; \\
  \cdots \\
  v_n = i_n; \\
\end{array}
\]
In all subsequent steps, the \emph{non-initialization} assignments are performed:
\[
\begin{array}{l}
  v_1 = e_1; \\
  \cdots \\
  v_n = e_n; \\
\end{array}
\]

\item Values of the outputs are computed;

\item Outputs are written and the new state is written;

\item The monitor yields control.
\end{enumerate}

The \konst{stepFn} for the monitor of the example described in
Section~\ref{sec:example} is displayed in \figref{fig:monitor-cakeml}.

\newsavebox{\monFn}
\begin{lrbox}{\monFn}
\begin{lstlisting}[style=myML]
stepFn (Request,Response)
       (req,rsp,current,previous,policy,alert) =
let val stateVars' =
     if !initStep then
        let val req = event(Request)
            val rsp = event(Response)
            val current = (req = rsp)
            val previous = req and not(rsp)
            val policy = current or previous
            val alert = not policy
            val () = (intStep := False)
        in (req,rsp,current,previous,policy,alert)
        end
     else
        let val req = event(Request)
            val rsp = event(Response)
            val current = (req = rsp)
            val previous = pre(req and not rsp) and (not req and rsp)
            val policy = current or previous
            val alert = (is_latched and pre(alert)) or not(policy)
        in (req,rsp,current,previous,policy,alert)
        end
    val (_,rsp',_,_,_,alert') = stateVars'
    val Alert = if alert' then Some () else None
    val Output =
       if alert' then None else
       if rsp'   then Some Response
       else None
in
   (stateVars', (Alert,Output))
end
\end{lstlisting}
\end{lrbox}

\begin{figure}
  \begin{center}
    \begin{tabular}{c}
      \scalebox{0.60}{\usebox{\monFn}}
    \end{tabular}
  \end{center}
  \caption{Synthesized CakeML for the monitor.}
  \label{fig:monitor-cakeml}
\end{figure}

%% \subsection{Component Behavior}

%% Intuitively, for monitor specification $s$, \konst{stepFn} is the
%% concrete embodiment of $\konst{SynthEval}\;s$, as defined in Section
%% \ref{agree-semantics}. Its correctness amounts to showing that, given
%% a sequence of inputs, and an initial state meeting the initialization
%% constraints, iterating \konst{stepFn} produces a $\pi$ s.t. $\pi \in
%% \Lang{s}$; and taking the union over all input sequences and
%% initial states produces $\Lang{s}$ itself.
