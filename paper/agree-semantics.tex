We now give an overview of a formal model for AGREE. This provides a
setting in which we are able to relate the contract correctness
results discussed above, obtained via model-checking, with the code
generated from high-assurance contracts.  The syntax of AGREE is
essentially that of quantifier-free first order predicate logic
supplemented with a few temporal operators. The terms (\emph{e}) are
arithmetic expressions built from variables ($v$) and numeric and
boolean literals ($c$), while formulas (\emph{b}) are built using
logical connectives from atomic formulas (\emph{a}) based on the
familiar comparison operators.
\[
\begin{array}{rcl}
e & ::= & v \mid c \mid e \;\set{+,*,/}\; e \\
a & ::= & e\; \set{=,<}\; e \\
b & ::= & v \mid c \mid a \mid \neg b
            \mid b \; \set{\land,\lor,\imp,\iff}\; b
\end{array}
\]

There is also a conditional, $\itelse{b}{(-)}{(-)}$, for both terms
and formulas. Lastly, there are temporal operators $\konst{pre}(-)$,
\emph{delay} $(-) \to (-)$, and $\konst{Hist}(-)$.

The semantics of terms and formulas is in terms of \emph{streams of
values}. Values encompass at least booleans and numbers, but can be
readily extended to include records and arrays. A value stream is a
total function from time (natural numbers) to values:
\[
 \konst{stream} = \mathbb{N} \to \konst{value}
\]
Given an \emph{environment} $E : \konst{name} \mapsto \konst{stream}$
binding variable names to value streams, the semantics $\sem{-}^E_t$
of terms and formulas defines the meaning of compound syntax in terms
of the meaning of subexpressions. The value of a variable $v$ at time
$t$ is found by looking up the stream bound to $v$ in $E$ (call it
$s$) and returning $s_t$. Some clauses of the semantics follow,
omitting the temporal operators:
\[
\begin{array}{rcl}
\sem{v}^E_t & = & E(v)(t) \\
\sem{c}^E_t & = & c \\
\sem{e_1 + e_2}^E_t & = & \sem{e_1}^E_t + \sem{e_2}^E_t \\
   & \cdots & \\
\sem{b_1 \land b_2}^E_t & = & \sem{b_1}^E_t \land \sem{b_2}^E_t \\
   & \cdots & \\
\end{array}
\]

The temporal operators deal with time in more significant ways. The
value of $\konst{pre}(e)$ at time $t$ is the value of $e$ at time
$t-1$ (at time zero, \konst{pre} is undefined).  A delay $e_1 \to e_2$
temporally ``shifts'' $e_2$ by means of prepending the first element
of $e_1$ to it.

\[
\begin{array}{rcl}
\sem{\konst{pre}(e)}^E_t & = & \sem{e}^E_{t-1}, \mathrm{when}\ t > 0 \\
\sem{e_1 \to e_2}^E_t & = & \itelse{t=0}{\sem{e_1}^E_0}{\sem{e_2}^E_t} \\
\sem{\konst{Hist}(b)}^E_t & = & \forall n \leq t.\; \sem{b}^E_n
\end{array}
\]

Although basic, these definitions can be used to define higher-level
operators from PTLTL, such as \konst{Once} and \konst{Since}.
