\begin{comment}
This section gives an abstract mathematical overview of AGREE's
compositional reasoning system, defining assume-guarantee contracts
and the verification conditions that AGREE checks in order to
modularly establish overall system correctness at the model level.
The syntax and semantics of AGREE are sketched in some detail.  The
discussion then becomes more concrete, showing---by example---how the
AGREE contract language is used to specify the cyber-hardened system
of \figref{fig:hardened}. The discussion proceeds from the
system-level contract to the filter and monitor components, which rely
on the notion of a \emph{code contract} to support code generation.
\end{comment}


We now give an overview of a formal model for AGREE syntax and semantics.
The discussion then becomes more concrete, showing by example, how the AGREE language is used to specify the properties of the SW system introduced in \secref{sec:example}.
Any implementation of the SW system must guarantee the specification output under the system assumptions.

The syntax of AGREE is
essentially that of quantifier-free first order predicate logic
supplemented with a few temporal operators. The terms (\emph{e}) are
arithmetic expressions built from variables ($v$) and numeric and
boolean literals ($c$), while formulas (\emph{b}) are built using
logical connectives from atomic formulas (\emph{a}) based on the
familiar comparison operators.
\[
\begin{array}{rcl}
e & ::= & v \mid c \mid e \;\set{+,*,/}\; e \\
a & ::= & e\; \set{=,<}\; e \\
b & ::= & v \mid c \mid a \mid \neg b
            \mid b \; \set{\land,\lor,\imp,\iff}\; b
\end{array}
\]
There is also a conditional, $\itelse{b}{(-)}{(-)}$, for both terms
and formulas. Lastly, there are temporal operators \emph{previously} $\konst{pre}(-)$,
\emph{followed-by} $(-) \to (-)$, and historically $\konst{Hist}(-)$.
These temporal operators are sufficient to define all other past-time operators \cite{monitor}.

The semantics of terms and formulas, as mentioned in the previous section, is in terms of \emph{streams of
values}. Values encompass at least booleans and numbers, but can be
readily extended to include records and arrays. A value stream is a
total function from time (natural numbers including 0) to values:
\[
 \konst{stream} = \mathbb{N}_0 \to \konst{value}
\]

Given an \emph{environment} $E : \konst{name} \mapsto \konst{stream}$
binding variable names to value streams, the semantics $\sem{-}^E_t$
of terms and formulas defines the meaning of compound syntax in terms
of the meaning of subexpressions. The value of a variable $v$ at time
$t$ is found by looking up the stream bound to $v$ in $E$ (call it
$s$) and returning $s_t$. Some clauses of the semantics follow,
omitting the temporal operators:
\[
\begin{array}{rcl}
\sem{v}^E_t & = & E(v)(t) \\
\sem{c}^E_t & = & c \\
\sem{e_1 + e_2}^E_t & = & \sem{e_1}^E_t + \sem{e_2}^E_t \\
   & \cdots & \\
\sem{b_1 \land b_2}^E_t & = & \sem{b_1}^E_t \land \sem{b_2}^E_t \\
   & \cdots & \\
\end{array}
\]

The temporal operators deal with time in more significant ways. The
value of $\konst{pre}(e)$ at time $t$ is the value of $e$ at time
$t-1$ (at time zero, \konst{pre} is undefined).
A followed-by, $e_1 \to e_2$, evaluates to $e_1$ at time zero; otherwise it is $e_2$.
Followed-by is often used to give a value to a stream in the first step so that subsequent steps are able to use $\konst{pre}(e)$.
\[
\begin{array}{rcl}
\sem{\konst{pre}(e)}^E_t & = & \sem{e}^E_{t-1}, \mathrm{when}\ t > 0 \\
\sem{e_1 \to e_2}^E_t & = & \itelse{t=0}{\sem{e_1}^E_0}{\sem{e_2}^E_t} \\
\sem{\konst{Hist}(b)}^E_t & = & \forall n \leq t.\; \sem{b}^E_n
\end{array}
\]

Although basic, these definitions can be used to define higher-level
past-time operators from PLTL such as \konst{Once} and \konst{Since}.
These can be expressed in a recursive style as follows:
%% \[
%%    \begin{array}{rcl}
%%       \sem{\konst{Once}(b)}^E_t & = & \sem{b \vee (\konst{false} \to \konst{pre}(o))}^E_t \\
%%       \sem{\konst{Since}(a, b)}^E_t & = & \sem{b \vee (a and (\konst{false} \to \konst{pre}(o)))}^E_t \\
%%    \end{array}
%% \]
\[
\begin{array}{rcl}
\konst{Once}(b) & = & b \to b \vee \konst{pre}(\konst{Once}(b)) \\
\konst{Since}(a, b) & = & b \to b \lor (a \land \konst{pre}(\konst{Since}(a, b))
\end{array}
\]
\noindent $\konst{Once}(b)$ is true if $b$ has ever been true before, or in, the
current moment.  $\konst{Since}(a, b)$ is true if $a$ has been true in
all moments up to the present one since $b$ most recently became true.
All other past-time operators of PLTL can be defined similarly \cite{monitor}.
