\documentclass[conference]{IEEEtran}
\IEEEoverridecommandlockouts
\usepackage{cite}
\usepackage{amsmath}
\usepackage{url}
\hyphenation{op-tical net-works semi-conduc-tor}

\begin{document}
\title{
  Synthesizing Verified Components for Cyber Assured Systems Engineering 
  \thanks{
    Funded by DARPA Cyber Assured Systems Engineering (CASE)
  }
}


% author names and affiliations
% use a multiple column layout for up to three different
% affiliations
\author{
\IEEEauthorblockN{
Konrad Slind, 
Junaid Babar, and 
Isaac Amundson}
\IEEEauthorblockA{
Trusted Systems Group   \\
Collins Aerospace       \\
Minneapolis, Minnesota  \\
}
\and
\IEEEauthorblockN{Eric Mercer}
\IEEEauthorblockA{Department of Computer Science\\
Brigham Young University\\
Provo, Utah}
}

% conference papers do not typically use \thanks and this command
% is locked out in conference mode. If really needed, such as for
% the acknowledgment of grants, issue a \IEEEoverridecommandlockouts
% after \documentclass

\maketitle

\begin{abstract}
Cyber-physical systems, such as avionics, must be tolerant to
cyber-attacks in the same way they are tolerant to random faults: they
must gracefully recover, or safely shut down, as requirements dictate.
We have developed a workflow for creating, and inserting,
high-assurance components implementing cyber-resiliency into a
model-based systems engineering environment.  Example high-assurance
components are filters, which guard against malformed input, and
runtime monitors, which guard against spoofing and other malicious
behavior. A formal specification in the form of a \emph{code contract}
defines each high-assurance component and is developed with the
support of \emph{test contracts} for testing the specified behavior.
Once tested, model checking used to verify that the added
high-assurance component indeed addresses system-level cyber
requirements.  Implementations for these high-assurance components are
directly synthesized from their code contracts and are backed up by
proofs showing that high-level specifications map in a
semantics-preserving way to code generated by a verified compiler.  We
report on a case study that cyber-hardened a UAV system by inserting
high assurance components to harden the open source Air Force Research
Laboratory's OpenUxAS services for route planning.  The case study
demonstrates that synthesizing correct implementations from code
contracts is feasible in real-world systems engineering.

\end{abstract}

\IEEEpeerreviewmaketitle

\section{Introduction}

\egm{Add to introduction and related work, that what is done at
Galois called Copilot. Its a stream processing langague that generates
code from the specification. Copilot to C. Effectively CodeGen from
Lustre. Bounds how far needed to look in past for any value.}

In recent years, aerospace stakeholders have realized that avionics
systems are subject to possible cyber-attacks just like other
cyber-physical systems.
%
\footnote{This work was funded in part by the
Defense Advanced Research Projects Agency (DARPA).  The views
expressed are those of the authors and do not reflect the official
policy or position of DARPA or the U.S. Government.}
%
Thus, in addition to being fault-tolerant, safety-critical avionics
systems must also be {\em cyber-resilient}.  Cyber-resiliency means
that the system is tolerant to cyber-attacks just as safety-critical
systems are tolerant to random faults: they recover and continue to
execute their mission function, or safely shut down, as requirements
dictate.

Unfortunately, systems engineers are currently given few development
tools to help answer even basic questions about potential
vulnerabilities and ways to mitigate vulnerabilities.  They instead
rely on process-oriented checklists and guidelines.  Cyber
vulnerabilities are often discovered during penetration testing late
in the development process; or worse yet, they may be discovered only
after the product has been fielded, necessitating extremely expensive
and time-consuming remediation. This is not a sustainable development
model.

%% The DARPA Cyber Assured Systems Engineering (CASE) project is targeted
%% at developing tools for design, analysis, and verification that enable
%% systems engineers to {\em design-in} cyber-resiliency for complex
%% cyber-physical systems.

We have been developing the {\em BriefCASE} toolsuite to address this need.
\brfcs\ is a Model-Based Systems Engineering (MBSE) environment
built in the Open Source AADL\footnote{AADL is the acronym for
Architecture Analysis and Design Language~\cite{aadl}.}  Tool
Environment (OSATE) to add new design, analysis, and code generation
capabilities for building cyber-resilient systems.

In this paper we describe how \brfcs\ facilitates inserting,
specifying, testing, and synthesizing high assurance components into a
system to improve its cyber-resiliency.  The main organizing concept
is that of an architecture-to-architecture \emph{security-improving
transform}, achieved via the insertion of a new architectural
component aimed at mitigating a cyber-vulnerability.  We describe two
cyber-resiliency transformations in this paper: (1) the insertion of a
filter to prevent malformed data from a malicious actor being
propagated to downstream components, and (2) the insertion of a
runtime monitor to detect (and alert) unexpected behaviors arising
from untrusted components.

A code contract is a formal specification in the Assume Guarantee
Reasoning Environment (\agr) language.
\agr\ is a compositional reasoning verification engine that uses \emph{contracts} on components to specify input and output properties and then prove whether or not those properties hold when given a sub-component implementation \cite{agree2013}.
The code contract language is Turing complete allowing the designer to
specify arbitrarily complex behavior.  These contracts are unit tested
for correctness with \emph{test contracts}.  Test contracts define
test scenarios to be implemented by the code contract under test.
\agr\ proves whether or not that code contract implementation is correct, enabling the designer to iteratively test the component behavior inside the \brfcs\ environment.
Once the behavior of the code contract is verified, \agr\ proves that---due to the newly
included high-assurance components---the hardened system meets its
cyber-resiliency requirements.

Another novel aspect of our workthe approach discussed in this manuscript is
the synthesis of the code contracts for the high-assurance components
to \ckml, a verified compiler implementation for the functional
programming language ML \cite{cakeml}.  This manuscript describes in
detail the synthesis path from code contracts to \ckml\ code,
providing a formal framework in which to argue correctness. \ckml\
then provides a verified compilation path to several different target
binaries (and also proving that the meaning of the \ckml\ source code
is exactly preserved in the final binaries).  A key contribution here
is that the code contract semantics are defined in such a way that
the \agr\ verification results for the code contract hold for the
deployed component, \ie, the component will detect and prevent the
indicated cyber-vulnerabilities over all possible finite
inputs. Preliminary work has shown how to lift this result to infinite
input traces as these systems are inherently reactive and intended to
run forever~\cite{case-verified-filter}, \cite{cakeml-space-cost}.

The manuscript further details a case study applying these
transformations with \brfcs\ to an Unmanned Aerial Vehicle (UAV)
system that uses the Air Force Research Laboratory's OpenUxAS services
for route planning.  OpenUxAS, as an open source product, is
considered \emph{untrusted}.  The UAV system is thus transformed to be
resilient to malicious behavior that may arise in the untrusted
component.  Here the transforms add filters to guard against malformed
input and monitors to guard against malicious flight plans from
OpenUxAS. The case study system is complex and shows the viability of
the approach in potential full-scale industrial design.

\brfcs\ is open source and publicly available \cite{fmide} as are the examples and case study discussed in this manuscript \cite{repo, phase2, camkes, case}.
Our approach currently applies to common cyber-vulnerabilities, such
as overflow, lack of input validation, and supply-chain issues;
however, other cyber-vulnerabilities such as side-channel attacks and
denial of service are not yet dealt with in our work.  Here, we do not
report on the invention of a new type of high-assurance component in
terms of capability; instead, our contribution is in the automated
synthesis of security-improving components from formal specifications
and a means to show that the synthesis is correct.

We now give a summary of the contributions detailed in this
manuscript.
\begin{compactitem}
  \item The language and semantics of code contracts to specify the behavior of high assurance components.
  \item Test contracts as a stylized mechanism to unit test code contracts inside the \brfcs\ framework with \agr.
  \item A synthesis path from code contracts to \ckml\ with a formal framework to argue correctness.
  \item A formal argument that \agr\ verification of code contracts carry over to the resulting binaries from the \ckml\ compiler.
  \item A case study that used the implementation of the approach in \brfcs\ to add cyber-resiliency to a non-trivial UAV system.
\end{compactitem}
The rest of this manuscript is organized as follows.
\secref{sec:overview} is an overview of the \brfcs\ environment and key tools in that environment relative to the contributions of this work. The approach is illustrated by a simple example in
\secref{sec:example}. Essential background on \agr\ verification and specification are
presented in \secref{sec:agree}.
The language and semantics of code contracts are defined in \secref{sec:code-contracts}, and test contracts are defined in \secref{sec:testing}.
The synthesis pathway is covered in Section~\ref{sec:synthesis}.
\secref{sec:case-study} discusses the case study.
This is followed by related work in \secref{sec:related-work}.
The conclusions and future work are in \secref{sec:conclusion}.

\begin{comment}
  BriefCASE incorporates model-level cyber analysis tools (presently
  GearCASE~\cite{gearcase2020} and DCRYPPS~\cite{dcrypps2019}) which can
  examine AADL models for potential vulnerabilities and suggest
  cyber-security requirements to mitigate them.  A library of
  architectural transforms guides the system engineer through automated
  model transformations that modify the architecture to address these
  requirements, possibly inserting new high-assurance components into
  the system.  Implementations for the new components are synthesized
  from formal specifications using
  SPLAT~\cite{slind-hcss2020},~\cite{formal-filter-synth-langsec}
  (Semantic Properties for Language and Automata Theory).


  Formal
  verification that the transformed system model meets its requirements
  is accomplished via \agr~\cite{agree2013} (Assume Guarantee Reasoning
  Environment).
  %The AGREE analysis \emph{assumes} properties on the inputs of a given component of the system, and attempts to formally prove the conjectured \emph{guarantees} of the output.
  \agr\ is a {\em compositional assume-guarantee} style model checker
  for AADL models that attempts to prove properties about one layer of
  an architecture using properties allocated to its subcomponents.
  Cyber-resilient code implementing the verified model is then
  automatically generated using the High Assurance Modeling and Rapid
  Engineering for Embedded Systems (HAMR) toolkit~\cite{hamr}.  If
  desired, this code can be targeted to the formally verified seL4
  secure microkernel~\cite{sel4-2009}.
\end{comment}


\section{Simple Example}
Show a filter, monitor, and gate form Phase II and indicate each of their intended roles.

\section{Specification Language}
Synthesize ML currently. 

1) Contiguity types have a proven path to binary

Should be able to prove a theorem that reasons over buffering.

\section{Verified Synthesis}
Synthesis maps from model and specifications to code. The synthesis
algorithm traverses the system architecture looking for occurrences of
filter and monitor specifications;  for each such occurrence it
generates a CakeML program. In the following, we examine both filter
and monitor synthesis. The latter is typically much more involved, and
we will therefore devote more attention to it.

\subsection{Filter Generation}

A filter is intended to be very simple; it is expected to have one
input port and one output; messages on the input that the filter
policy admits pass unchanged to the output port; all others are
dropped (not passed on). We have explored in our work two kinds of
filter. In the first, a relatively shallow scan of the input buffer
can enforce the policy. For example, we have used the expressive power
of Contiguity Types \cite{contiguity-types} to enforce
\emph{lightweight} bounds constraints on GPS coordinates in UxAS
messages. On the other hand, the second kind of filter will parse the
input buffer into a data structure specified in AGREE and apply an
user-defined \emph{well-formedness} property, also specified in AGREE,
to the data. This allows arbitrarily complex well-formedness checking.

The decision of a filter is made and performed within one thread
invocation. Thus, in its given time slice, the filter does the following:

\begin{enumerate}

\item checks to see if there is any input available; if there is none
then it yields control; otherwise,

\item the input is read (and parsed if need be),

\item the wellformedness predicate is evaluated,

\item if the predicate returns \konst{true} then the input buffer
 is copied to the output; otherwise no action is taken, and

\item the filter yields control
\end{enumerate}

\begin{remark}[Partiality]

The role of partiality should be emphasized: steps 2 and 3 can fail;
the data might not be parseable or the wellformedness computation
could be badly written and fail at run time. In such cases, the filter
should recover and yield control without passing the input onwards. In
these cases, the filter is behaving as it should, but there are also
cases in which a correctly specified filter would not meet its
specification at runtime. This situation arises when the
filter \emph{ought} to accept a message, but lack of resources result
in the filter failing to do so. Examples of this would be, for
example, if the parse of a message needed more space than allocated;
another example would be if the timeslice provided by the scheduler is
too small for the wellformedness computation to finish.

\end{remark}


{\emph{Need some discussion of filters and their step-wise properties
  in relation to their infinitary properties. Reference to Johannes'
  work.}

\newsavebox{\contig}
\begin{lrbox}{\contig}
\begin{lstlisting}[style=myML]
  Waypoint =
    {Latitude  : f64
     Longitude : f64
     Altitude  : f32
     Check     : Assert
      (~90.0 <= Latitude and Latitude <= 90.0 and
       ~180.0 <= Longitude and Longitude <= 180.0 and
       1000.0 <= Altitude and Altitude <= 15000.0)}

  AutomationResponse =
    {TaskID : i64
     Length : u8
     Waypoints : Waypoint [3]}
\end{lstlisting}
\end{lrbox}

\begin{figure}
  \begin{center}
    \begin{tabular}{c}
      \scalebox{0.60}{\usebox{\contig}}
    \end{tabular}
  \end{center}
  \caption{Contiguity type specification for filter.}
  \label{fig:filter-contig}
\end{figure}


\newsavebox{\cml}
\begin{lrbox}{\cml}
\begin{lstlisting}[style=myML]
fun filter_step () =
 let val () = Utils.clear_buf buffer
     val () = API.callFFI "get_input" "" buffer
 in
    if WELL_FORMED_AUTOMATION_RESPONSE buffer
    then
      API.callFFI "put_output" buffer Utils.emptybuf
    else print"Filter rejects message.\n"
end
\end{lstlisting}
\end{lrbox}

\begin{figure}
  \begin{center}
    \begin{tabular}{c}
      \scalebox{0.60}{\usebox{\cml}}
    \end{tabular}
  \end{center}
  \caption{Synthesized CakeML for the filter.}
  \label{fig:filter-cakeml}
\end{figure}

The contiguity type specification for the filter is shown in
\figref{fig:filter-contig}. The synthesized CakeML code for the filter is shown in
\figref{fig:filter-cakeml}. The code is called at dispatch by the
scheduler. The \texttt{API.callFFI} is the link to the communication
fabric to capture input and provide output. The body of the function
restates the filter contract to make the appropriate assignments in a
way that matches the truth value of the predicate in the filter
guarantee.  The auto-generated AGREE specification raises an alert
output when the relation is violated.

% A \emph{system} is a collection of \emph{components}, \emph{connections}
% between components, a \emph{scheduler} to order execution, and a
% \emph{system environment} for primary inputs.

\subsection{Monitor Generation}

Monitors, since they are intended to track and analyze the externally
visible behavior of system components through time, require more
computational features than filters. In particular, our conception of
a monitor is a predicate on its input (and output) streams, being able
to access the value of a stream at any earlier point in time, if
necessary. Thus monitors commonly use state to keep track of earlier
values, unlike filters which are stateless. This reasoning leads us to
specify the computation for a monitor component by a step function of
the following form:
\[
\konst{stepFn} : \mathit{input} \times \mathit{stateVars} \to \mathit{stateVars} \times \mathit{output}
\]
Thus, in its given timeslice, a monitor evaluates the \konst{stepFn} on its inputs and the current values of the state variables. In detail, it takes the following steps:

\begin{enumerate}

\item each available input is parsed into data of the type specified
by the port type;

\item the stateful variables in $\mathit{stateVars}$  are evaluated in dependency order.

\item values of outputs are computed

\item outputs are written and the new state is written

\item control is yielded
\end{enumerate}

Our earlier remarks on partiality apply here too, of course.


The scheduler \emph{activates} components in some order.It is an
obligation on the system that the scheduler follows some sensible
partial order of component activation and allows each component
sufficient time for its computation.  Activating a monitor component
takes the form of the following pseudo-code:
\[
\begin{array}{ll}
 \mathit{(i_1,\ldots)} & = \konst{readInputs}(); \\
 (v_1,\ldots) & = \konst{readState}() ; \\
 ({v_1}',\ldots), ({o_1}',\ldots) & = \konst{stepFn} ((i_1,\ldots),(v_1,\ldots)) ; \\
 \multicolumn{2}{l}{\konst{writeState}({v_1}',\ldots);} \\
 \multicolumn{2}{l}{\konst{writeOutputs}({o_1}'\ldots);} \\
\end{array}
\]

\paragraph{Initialization}
A monitor may need to accumulate a certain minimum number of
observations before being able to make a meaningful assessment of
behavior. Until that threshold is attained, the monitor is essentially
in a kind of \emph{initialization} phase. In order for correct code to
be generated, monitor specifications need to spell out the values of
output ports when in an initialization phase. For example, suppose a
monitor does some kind of differential assessment of inputs at
adjacent timeslices, alerting when (say) the measured location of a
UAV at times $t$ and $t+1$ is such that the distance between the two
locations is unusually large. Such a monitor needs two measurements
before making its first judgement, but at the time of its first
output, only one measurement will have been made. The specification
for must explicitly state what the correct first output is.


\section{OSATE Integration for AADL}
\input{aadl-agree-osate}

\section{Conclusion}
Cyber-physical systems must be tolerant to cyber-attacks in the same way they are tolerant to faults. The DARPA CASE program is creating an MBSE environment for designers to integrate cyber-vulnerability analysis and mitigation through the design process to harden systems early in the design process. BriefCASE is the result of that program, and it provides cyber-analysis tools that add cyber requirements to the AGREE specification for the design and architectural transforms with AGREE specifications to satisfy the cyber requirements. BriefCASE is fully integrated into the OSATE AADL modeling environment for ease of use by system engineers.

The filter architectural transformation in BriefCASE prevents malformed data from being propagated downstream to other components while the monitor transformation enforces temporal properties to detect when untrusted components behave maliciously. The components are specified by corresponding auto-generated AGREE contracts that only require the system designer to state the policies to enforce.These can be adapted from the cyber requirements in the system design. The specifications are automatically synthesized to CakeML that can then be compiled to a wide array of backend targets. The synthesis to CakeML is done in a way that preserves the meaning of the specifications.

The BriefCASE tool suite demonstrated in action on a full-scale case study with the UxAS open source UAV AI planning system. The complex system requires several architectural transformations to meet cyber-requirements, and the entire system is synthesized and deployed on the seL4 platform running on ODRIOD using the synthesis tool described here and the BriefCASE HAMR tool to generate the communication fabric. The size and scale of the case study gives some evidence that BriefCASE, with its transformations, scales to the complexity demands often seen in real-world design.

Ongoing work is using BriefCASE on the Collin' Common Avionics Architecture System (CAAS) in the context of a CH47 rotary wing platform \cite{caas}. The filter and monitor transformations are being employed to protect against \emph{Automatic Dependent Surveillance-Broadcast} (ADS-B) spoofing by malicious actors in the airspace. That work is expanding synthesis with uninterpreted functions to support non-linear and transcendental functions. Other ongoing work is mechanizing the synthesis proof in HOL4 and lifting the proof results given for finite streams to infinite streams.

% An example of a double column floating figure using two subfigures.
% (The subfig.sty package must be loaded for this to work.)
% The subfigure \label commands are set within each subfloat command,
% and the \label for the overall figure must come after \caption.
% \hfil is used as a separator to get equal spacing.
% Watch out that the combined width of all the subfigures on a 
% line do not exceed the text width or a line break will occur.
%
%\begin{figure*}[!t]
%\centering
%\subfloat[Case I]{\includegraphics[width=2.5in]{box}%
%\label{fig_first_case}}
%\hfil
%\subfloat[Case II]{\includegraphics[width=2.5in]{box}%
%\label{fig_second_case}}
%\caption{Simulation results for the network.}
%\label{fig_sim}
%\end{figure*}
%
% Note that often IEEE papers with subfigures do not employ subfigure
% captions (using the optional argument to \subfloat[]), but instead will
% reference/describe all of them (a), (b), etc., within the main caption.
% Be aware that for subfig.sty to generate the (a), (b), etc., subfigure
% labels, the optional argument to \subfloat must be present. If a
% subcaption is not desired, just leave its contents blank,
% e.g., \subfloat[].

% trigger a \newpage just before the given reference
% number - used to balance the columns on the last page
% adjust value as needed - may need to be readjusted if
% the document is modified later
%\IEEEtriggeratref{8}
% The "triggered" command can be changed if desired:
%\IEEEtriggercmd{\enlargethispage{-5in}}

% references section
\bibliographystyle{IEEEtran}
\bibliography{paper}
\end{document}


