% TODO: add in the referee tag for the submission
%\documentclass[global,twocolumn,referee]{svjour}
\documentclass[global,twocolumn,keeplastbox]{svjour}
\usepackage{cite}
\usepackage{amsmath}
% \usepackage{amsthm}
\usepackage{amsfonts}
\usepackage{url}
\usepackage{comment}
\usepackage{paralist}
\usepackage{graphicx}
\graphicspath{{./figs/}}
\usepackage{listings}
\usepackage[usenames,dvipsnames,svgnames,table]{xcolor}
\usepackage{flushend}
\usepackage{hyperref}
\usepackage[firstpage]{draftwatermark}
\usepackage[lighttt]{lmodern}
\usepackage[normalem]{ulem}

\definecolor{dkgreen}{rgb}{0,0.6,0}
\definecolor{mauve}{rgb}{0.58,0,0.82}
\definecolor{light-gray}{gray}{0.88}

\lstdefinestyle{myML}
{frame=none,
  basicstyle=\ttfamily,
  language=ML,
  aboveskip=3mm,
  belowskip=3mm,
  showstringspaces=false,
  columns=flexible,
  numbers=none,
  numberstyle=\tiny\color{gray},
  commentstyle=\color{dkgreen},
  stringstyle=\color{mauve},
  breaklines=false,
  breakatwhitespace=true,
  tabsize=2,
  linewidth=2\linewidth
}

\lstdefinestyle{agree}
{frame=none,
  basicstyle=\ttfamily,
  language=ML,
  aboveskip=3mm,
  belowskip=3mm,
  showstringspaces=false,
  columns=flexible,
  numbers=none,
  numberstyle=\tiny\color{gray},
  commentstyle=\color{dkgreen},
  stringstyle=\color{mauve},
  breaklines=false,
  breakatwhitespace=true,
  tabsize=2,
  linewidth=2\linewidth,
  morecomment=[l]{--},
  escapeinside={*}{*},
  morekeywords={eq, bool, guarantee, assume, true, false, pre, not, and, or, property, const, Historically, Since, Once, event, annex, Implementation, process, subcomponents}
}

\hyphenation{op-tical net-works semi-conduc-tor Cake-ML}

\newcommand{\konst}[1]{\ensuremath{\mathsf{#1}}}
\newcommand{\imp}{\Rightarrow}
\newcommand{\set}[1]{\ensuremath{\{ {#1} \}}}
\newcommand{\sem}[1]{\ensuremath{[\![ #1 ]\!]}}
\newcommand{\kstar}[1]{\ensuremath{{#1}^{*}}}
\newcommand{\Lang}[1]{\ensuremath{{\mathcal L}({#1})}}
\newcommand{\Eqs}{\ensuremath{\mathit{Eqs}}}
\newcommand{\itelse}[3]{\mbox{$\mathtt{if}\ {#1}\ \mathtt{then}\ {#2}\ \mathtt{else}\ {#3}$}}

\newcommand{\figref}[1]{Fig.~\ref{#1}}
\newcommand{\secref}[1]{Sec.~\ref{#1}}
\newcommand{\lineref}[1]{Line~\ref{#1}}
\newcommand{\linesref}[2]{Lines~\ref{#1}--\ref{#2}}
% \newcommand{\eqref}[1]{Eq.~\ref{#1}}

% Latex trickery for infix div operator, from stackexchange

\makeatletter
\newcommand*{\bdiv}{%
  \nonscript\mskip-\medmuskip\mkern5mu%
  \mathbin{\operator@font div}\penalty900\mkern5mu%
  \nonscript\mskip-\medmuskip
}
\makeatother

\newcommand{\ie}{\textit{i.e.}}
\newcommand{\eg}{\textit{e.g.}}
\newcommand{\etal}{\textit{et al.}}
\newcommand{\etc}{\textit{etc.}}
\newcommand{\adhoc}{\textit{ad hoc}}

% \theoremstyle{plain}
% \newtheorem{theorem}{Theorem}
% \newtheorem{lemma}{Lemma}

% \theoremstyle{definition}
% \newtheorem{definition}{Definition}

% \theoremstyle{remark}
% \newtheorem{remark}{Remark}

\journalname{Software and Systems Modeling}

\newif\ifREVISIONS
\REVISIONSfalse

\begin{document}
\title{
  Synthesizing Verified Components for Cyber Assured Systems Engineering
  \thanks{DISTRIBUTION STATEMENT A.  Approved for public release.}
}

\author{
  Eric Mercer\inst{1}    \and
  Konrad Slind\inst{2}   \and
  Isaac Amundson\inst{2} \and
  Darren Cofer\inst{2}   \and
  Junaid Babar\inst{3}   \and
  David Hardin\inst{3}
}

\institute{
  Brigham Young University        \\
  Provo, Utah                     \and
  Applied Research and Technology \\
  Collins Aerospace               \\
  Minneapolis, Minnesota          \and
  Applied Research and Technology \\
  Collins Aerospace               \\
  Cedar Rapids, Iowa
}

\date{Received: date / Revised version: date}

\maketitle

\begin{abstract}
Cyber-physical systems, such as avionics, must be tolerant to
cyber-attacks in the same way they are tolerant to random faults: they
either gracefully recover or safely shut down as requirements dictate.
The DARPA Cyber Assured Systems Engineering program is developing
tools for design, analysis, and verification that enable systems
engineers to design-in cyber-resiliency in a Model-Based Systems
Engineering environment.  This paper describes automated model
transformations that introduce high-assurance cyber-resiliency
components into a system, in particular filters and monitors that
prevent malicious input and detect supply chain attacks, respectively.
A formal specification in the form of a code-contract defines each high-assurance component and is
used to verify that the component addresses system level cyber
requirements.  Implementations for these high-assurance components are
directly synthesized from their code-contracts, and are automatically
proven to preserve the exact meaning of the code-contract all the way
down to the binary code level.  The model transformations are
integrated into the Open Source AADL Tool Environment (OSATE).  The
paper further reports on two case studies, one by formal method experts 
and the other by non-experts,
applying cyber-resiliency model
transformations to two different systems. 
One is a UAV system that uses the Air Force Research
Laboratory's OpenUxAS services for route planning.
The other is a tablet used for inflight communication between pilots in an Apache helicopter.
In each case study, the model transformations add filters, to guard against malformed
input, as well as monitors, to guard against spoofing and other malicious behavior.
The case studies show the MBSE is feasible in real-world systems engineering.
\end{abstract}

\section{Introduction} \label{sec:intro}
In recent years, aerospace stakeholders have become aware that avionics systems are subject to possible cyber-attacks just like other cyber-physical systems.  In addition to being fault-tolerant, safety-critical avionics systems must also be {\em cyber-resilient}. Cyber-resiliency means that the system is tolerant to cyberattacks just as safety-critical systems are tolerant to random faults: they recover and continue to execute their mission function, or safely shut down, as requirements dictate. 

Unfortunately, systems engineers are currently given few development tools to help answer even basic questions about potential vulnerabilities and mitigations, and instead rely on process-oriented checklists and guidelines.  Cyber vulnerabilities are often discovered during penetration testing late in the development process.  Worse yet, they may be discovered after the product has been fielded, necessitating extremely expensive and time-consuming remediation. This is not a sustainable development model.

The DARPA Cyber Assured Systems Engineering (CASE) program is targeted to developing design, analysis, and verification tools that enable systems engineers to {\em design-in} cyber-resiliency for complex cyber-physical systems.\footnote{The
  views expressed are those of the authors and do not reflect the
  official policy or position of the Defense Advanced
  Research Projects Agency (DARPA) or the U.S. Government.}
The result is a Model-Based Systems Engineering (MBSE) environment called {\em BriefCASE} which is based on the Architecture Analysis and Design Language (AADL)~\cite{aadl}.  BriefCASE extends the Open Source AADL Tool Environment (OSATE) to add new design, analysis, and code generation capabilities targeted at building cyber-resilient systems.  

BriefCASE provides access to two analysis tools (GearCASE~\cite{gearcase2020} and DCRYPPS~\cite{dcrypps2019}) that can examine AADL models to detect potential cyber vulnerabilities and suggest requirements for mitigation.  
A library of architectural transforms guides systems engineers through automated model transformations that modify the architecture to address these requirements, possibly inserting new high-assurance components into the system. 
Implementations for these new high-assurance components are synthesized from formal specifications using the Semantic Properties for Language and Automata Theory (SPLAT) tool~\cite{slind-hcss2020},~\cite{formal-filter-synth-langsec}.
Formal verification that the transformed system model satisfies its cyber requirements is accomplished via model checking using the Assume Guarantee Reasoning Environment (AGREE)~\cite{agree2013}.  Cyber-resilient code implementing the verified model is automatically generated using the High Assurance Modeling and Rapid Engineering for Embedded Systems (HAMR) toolkit~\cite{hamr}.  If desired, this code can be targeted to the formally verified seL4 secure microkernel~\cite{sel4-2009}.

A novel aspect of the approach is the ability to automatically transformation the model to insert high-assurance components to harden the system against potential cyber-vulnerabilities. The behavior of the model is given by contracts on components expressed in the AGREE modeling language. These contracts are modified as a result of the cyber-vulnerability analysis to assume properties on the input and and guarantee properties of the output that reflect a hardened system. In the absence of any transformation to implement the hardening, AGREE fails to prove the system cyber-hardened.

The two automatic transforms discussed in this paper are the insertion of filters to prevent malformed data from malicious actors from being propagated to downstream components, and monitors to detect and alert unexpected behaviors arising from untrusted components. These transformations not only change the architecture of the model by adding in new components, but they generate a complete formal specification for the behavior of the inserted high-assurance components in the AGREE language. Those specifications are sufficient for model checking to prove that with the high-assurance components, hardened system meets its cyber-resiliency requirements.

The other novel aspect of the approach is the synthesis of the AGREE specifications for the high-assurance components to CakeML. This paper agues that synthesis to CakeML to be correct. CakeML then provides a verified compilation path to several different target binaries proving that the meaning of the CakeML is exactly preserved in the final binaries. Assuming the deployed cyber-hardened system is scheduled as intended by the AADL model, and the HAMR generated communication fabric delivers messages between components as expected, then the AGREE model checking results hold for the deployed system in that in detects and prevents the indicated cyber-vulnerabilities over all possible finite inputs. Future work is lifting this result to infinite input traces as these systems are inherently reactive and intended to run forever.

The approach is motivated, and illustrated, in a simple example. A full case study applying these transformations to a UAV system that uses the Air Force Research Laboratory's OpenUxAS services for route planning is briefly reported. Here the transforms add filters to guard against malformed input and monitors to guard against ground station spoofing and malicious flight plans from OpenUxAS. The case study system is significantly more complex than the example and shows the viability of the modeling approach to full-scale industrial design.

\begin{comment}
Here is the basic process:
\begin{enumerate}
  \item CakeML provides a mechanism to take a defined logic in HOL4 and turn it into CakeML in a way that exactly preserves the meaning of the logic.
  \item Verified components are created by defining the meaning of the components in HOL4, and then using the existing theories to turn that meaning into CakeML
  \item Contiguity types provide a formalism for defining a language and recognizing membership in a language via a matcher
  \item Contiguity types include dependent types and the ability to define when a type is malformed
  \item Define a contract language that makes assumptions on input and guarantees properties on output
  \item Give meaning to the contract language in terms of traces (not easy to do because of the state)
  \item Show how the contract can be expressed in terms of a step function 
  \item The step-function is lifted to traces
  \item A contract defines a language over traces
  \item A step function recognizes membership in that language (prefix closed)
  \item Follow the same pattern used in contiguity types
  \item The CakeML comes for free once the HOL4 theories are shown
\end{enumerate}


Set the context for this work: define system level theorem that we intend to prove with the added components? Big picture of the overall goal of the project.

Components (complexity (and expressiveness) increases along the way):
* Filter: takes a stream of data and delivers a new stream of data where the property is enforced--reasons over an infinite stream of date. Connects to the system correctness over traces. Predicate on a single piece of data, paying no attention to previous history--no temporal awareness, deciting if it passes or not. Can be very secure and very efficient.
* Monitor: Captures a relation over time on the data. Is able to reason about temporal properties. Supports multiple inputs and is able to do arbitrary computation for complex output over the inputs.

The attestation gate is an aside to show how the tools we have are rich enough to create other complex things that blend the two: filter and monitor.

This paper addresses only the verified components. Here is what we verify:

* The resulting assembly code exactly implements the specification--preserves the meaning of the original specification in the actual assembly code.

Scheduling and Trace Relationships: we assume a stream of correct data-types and the existence of some scheduler to schedule components appropriately. In the later section we give an example of a system that uses seL4 with a pacer to schedule over different domains to provide memory isolation. Scheduling atd the relation to the trace semantics is not part of this paper.


Revised outline:

Describe the system level context with the AGREE analysis that proves system level properties of streams. A system is a collection of components and connections. We assume a finite set of defined datatypes for the connections. A well-formed system is acyclic and provides some suitable set of input streams to consume.

Every component is defined by its inputs, outputs, and a contract. The contract language is simplified AGREE: requires, guarantees, eq, and expressions that include \emph{pre}. Requires are pre-conditions assumed on the input. Guarantees are post-conditions the output must meet. A component consumes a stream of input and produces and stream of output. A well-formed component is also acyclic. 

For a component to by synthesized, its contract must be strong enough to define the meaning of every output (e.g., the value assumed by the output and under what conditions it assumes that value). 

The AGREE analysis reasons about the system with its component composition proving that all contracts are respected. Failures are counter-examples where some input requirement is violated. The analysis provides a system level guarantee. 

Component synthesis compiles the contract to imperative code that implements the contract. The claim is that the final compiled code preserves the meaning of the contract: it produces the same output stream for any input stream that conforms to its pre-condition. The synthesis is accomplished with the CakeML framework that provides the additional guarantee that the post-condition holds over an infinite stream (e.g., that it the component does not terminate and always satisfies the post-condition if the pre-condition holds---Hoare logic extended to statement that diverge).

The semantics is synchronous data-flow. Computation takes zero time as does transport. The system must be well-formed. 

Motivate with some example: watchdog example on DNS monitoring for DOS attacks? Estimates the growth rate of the input request queue over time and sends a throttle signal if the growth is too fast (looks suspicious)---any bounded response of behavior based watch-dog is fine.
\end{comment}

\ifREVISIONS
\subsection{SOSYM Journal Details}
The due date is \textbf{March} $22^\mathrm{nd}$, \textbf{2022}.
Go to \url{https://mc.manuscriptcentral.com/sosym/}. Create a new account (if you don't have one), login and go to the "Author"-section, select the link "Start New Submission" under Author Dashboard and select "Special Section Paper" as the paper type, and choose "MODELS 2021 Special Issue". Also, make sure to mention the MODELS 2021 special issue in the cover letter.
\fi

\ifREVISIONS
\subsection{Revisions}
\begin{compactitem}
  \item Revise the end of the abstract regarding the case studies to be consistent with what ends up in this version of the paper.
  \item Revise the end of the introduction that discusses the case studies to be consistent with what ends up in this version of the paper.
  \item Fix the link to the motivating example to point to the new repository once it is created. Fix in the next section too.
  \item Add discussion on the limitation of the approach: what attacks can and cannot be covered, can valid messages be rejected, etc.
\end{compactitem}
\fi

\section{Illustrative example}
\label{sec:example}
\begin{figure*}
  \begin{center}
    \begin{tabular}{c}
      \includegraphics[scale=0.4]{flowchart.png} \\
    \end{tabular}
  \end{center}
\caption{AGREE failure certificate for initial design.}
\label{fig:flowchart}
\end{figure*}

\begin{figure*}[h]
  \begin{center}
    \begin{tabular}{c}
      \includegraphics[scale=0.4]{example.png}
    \end{tabular}
  \end{center}
\caption{Initial design for an automated UAV route planning system.}
\label{fig:example}
\end{figure*}

\begin{figure}
  \begin{center}
    \begin{tabular}{c}
      \includegraphics[scale=0.4]{example-certificate.png} \\
    \end{tabular}
  \end{center}
\caption{AGREE failure certificate for initial design.}
\label{fig:example-certificate}
\end{figure}

\egm{
  Merge the AGREE overview here into this section.
  Condense it down to just the opening two paragraphes that currently exist in the overview section.
  Walk through the first part of the example up to where AGREE fails because of Cyber Requirements.
  Introduce the new figure showing its workflow.
  Finish the example walking through the new figure. 
  Add the tests to occur before final AGREE check.
  Send to SPLAT to get verified CakeML code. 
}

\figref{fig:flowchart} illustrates a simplified \brfcs\ workflow with the shaded boxes representing the contributions discussed in this manuscript: test contracts, code contracts, and SPLAT synthesis.
The unshaded boxes are the existing tools that generate evidence for the assurance case.
The rest of this section follows an illustrative example through the workflow.

The baseline AADL model is a simple software system, named SW, for route planning
and automated control of a UAV. A picture of the AADL architectural
model for SW is in \figref{fig:example}. SW is loosely based on one of
the case studies in Section~\ref{sec:case-study}.  The source for the
entire model is found at \cite{repo}.

From time to time, SW receives a message on the \texttt{AutomationRequest} input,
forwarding it to a route planner (AI). The function of AI is to 
compute the flight path (a list of waypoints) for the UAV, outputting
the resulting \emph{mission command} on \texttt{AutomationResponse}.
The waypoint manager (WM) receives the mission command from AI and
starts the UAV flying the mission by putting an \emph{event} on the
\texttt{Start} output port, continuing to issue waypoints to the UAV
flight controller on the \texttt{Waypoint} port as the UAV location
updates with messages on \texttt{AirVehicleLocation}.
An event is repeatedly generated on \texttt{Alert} if there is no
mission command from a request. The AI component is third-party
software and WM is a legacy component.

%% that cannot be modified so it critically relies on
%% assumptions about its input behavior to guarantee its intended output
%% behavior.

The expected behavior of the SW system, and the components
implementing the system, are modeled with AGREE contract
specifications.  
These contracts assume and guarantee the absence
of any malicious, or unspecified, component behavior.  More precisely,
the contract for SW assumes that its inputs are \emph{well-formed} and that
there is never more than one automation request pending at a time.
Well-formed generally refers to a syntactic restriction on
data at a port. For example, a waypoint is well-formed if it falls
within bounds for latitude, longitude, and altitude.)  The guarantees
for SW ensure that
\begin{compactitem}
\item a start coincides with a new waypoint message being output;
\item a start is within one cycle of an automation request and if not, then it persistently alerts;
\item new waypoints coincide with air vehicle location updates; and
\item all outputs are well-formed.
\end{compactitem}

The contracts for the sub-components assume their inputs are
well-formed, and they guarantee their outputs are well-formed.  The AI
contract guarantees it only responds to automation requests and always
in the same cycle.  The legacy WM contract guarantees the following if
its input assumptions hold:
\begin{compactitem}
  \item it generates a start from a response from the AI and always in the same cycle;
  \item the start always coincides with a waypoint being output; and
  \item any further waypoints coincide with air vehicle location updates.
\end{compactitem}
These initial specifications pass verification, meaning that the
contract composition of the components with the system satisfies all
component input assumptions and system output guarantees.
AGREE background and the baseline contracts are discussed in detail in \secref{sec:agree} and \secref{sec:agree-semantics}.

\subsection{Adding cyber requirements}

A cyber-vulnerability analysis identifies the potential of a
supply chain attack through the AI route planner (provided by a
third-party vendor without source code).  Based on the analysis, the designer adds a requirement that the system must guard agaist malicious AI behavior, marks the component as \emph{untrusted}, and removes all output guarantees from its contract.  The AI output is now unconstrained in the
assume-guarantee reasoning of AGREE and is able to generate any value
on its output at anytime. 

The output from the AGREE analysis using the untrusted AI
specification is shown in \figref{fig:example-certificate}.  The red
exclamation points designate component assumptions or system output
guarantees that do not hold, and each failure comes with a
corresponding counter-example.  

The first violation is that the mission command on \texttt{AutomationResponse} from
the AI to the WM is no longer guaranteed to be well-formed.  The
consequence of that failing input assumption is that the WM outputs
are no longer guaranteed; they are effectively unconstrained.  These
missing output guarantees from the WM lead to the rest of the failures
in \figref{fig:example-certificate} because the WM provides the system
level outputs.

\begin{figure*}
  \begin{center}
    \begin{tabular}{c}
      \includegraphics[scale=0.3]{hardened.png}
    \end{tabular}
  \end{center}
  \caption{Cyber-hardened design for an automated UAV route planning system}
  \label{fig:hardened}
\end{figure*}

\begin{figure}
  \begin{center}
    \begin{tabular}{c}
      \includegraphics[scale=0.38]{agree-test-output.png} \\
      (a) \\ \\
      \includegraphics[scale=0.4]{hardened-certificate.png} \\
      (b)    
    \end{tabular}
  \end{center}
  \caption{AGREE verification certificates. (a) Test contract verification results. (b) Cyber-hardened design verification results.}
  \label{fig:hardened-certificate}
\end{figure}

\emph{Adding high assurance components}

The system designer now uses \brfcs\ to cyber-harden SW by inserting
high-assurance components in the form of a filter and a monitor, as
shown in \figref{fig:hardened}.
These isolate the AI component and prevent it from violating the WM component input assumptions.
A filter enforces an invariant over
each datum in the data stream by not forwarding input to its output if the invariant does not hold.
\brfcs\ generates the initial code contract for the filter which the designer completes by writing the invariant property.
That property it is usually based on the existing assumptions made by
downstream components that consume the filter output.

A monitor captures a relation on input data over time and is thus able
to reason about the temporal, and other invariant, properties of that input.  A monitor raises
an alert if the specified properties are ever violated.  
As with the filter, \brfcs\ generates a template code contract that is completed by the designer with the desired behavior that is typically defined by the SW system and its implementing component contracts. 
The code contract in our example states that an
automation response can only be generated in conjunction with an
automation request; and further, that response must come with the
request or in the next step after the request.  
Code contracts are discussed in detail in \secref{sec:code-contracts}.

Code contracts for high assurance components can be arbitrarily complex since they can have persistent state that evolves over time.
Test contracts allow the designer to validate the behavior, and general input to output properties, of a code contract through unit testing.
For example, one of the test contracts for the monitor is that it should not alert and pass its input when the monitor sees the \texttt{AutomationReponse} one step after it sees the \texttt{AutomationRequest}.
Test contracts are discussed in detail in \secref{sec:test-contracts}.

The AGREE analysis of the cyber-hardened implementation is shown in
\figref{fig:hardened-certificate}.
\figref{fig:hardened-certificate}(a) is the test contract results and \figref{fig:hardened-certificate}(b) is the hardened system composition.
Here AGREE has generated high-level
evidence justifying the claim that the high-assurance components meet their intended purposes.
Having passed AGREE verification, the high-assurance components are ready to be
synthesized.

\subsection{Synthesizing code from code contracts}

High-assurance components are automatically synthesized by SPLAT from
the code contracts to equivalent programs in the CakeML
language.
%% Too strong at the moment!
%The synthesis toolchain generates a proof that equates the
%% meaning of the AGREE specification to the behavior of the CakeML
%% program.
 In other words, for any set of input streams that meet the
 component's contract assumptions, the output streams produced from
 the synthesized CakeML code exactly match those defined by the
 high-assurance component's guarantees.  The CakeML compiler provides
 verified compilation to binaries for several different platforms,
 meaning that the resulting binaries exactly preserve the meaning of
 the original CakeML code \cite{cakeml}.

Preserving the input to output relationship of streams between the
AGREE contracts and CakeML lifts the AGREE contract verification
results to the actual deployed system.  If the contract model
verification succeeds, then the meaning of those results hold for the
deployed system.  These results, however, are only valid under
additional assumptions on the deployed system:
\begin{compactitem}
\item contracts for non-synthesized components accurately model their deployed
counterparts;
\item an appropriate schedule exists to sequence component
  execution following the dependent data-flow in the design; and
\item the communication fabric stitching components together works in harmony
  with the schedule.
\end{compactitem}
\noindent Support and automation for these aspects of the design process are
discussed in other works \cite{gearcase2020, dcrypps2019, 10.1007/978-3-030-89159-6_18, 10.1007/978-3-030-89159-6_17, sel4-2009, scheduled-agree, 9734792}.
Synthesis from code contracts is discussed in detail in \secref{sec:sythesis}.

\section{Compositional reasoning}
\label{sec:agree}

\newcommand{\globally}{\ensuremath{\mathbf{G}}}
\newcommand{\historically}{\ensuremath{\mathbf{H}}}
\newcommand{\assumes}{\ensuremath{A}}
\newcommand{\guarantees}{\ensuremath{P}}
\newcommand{\dispatch}{\ensuremath{\mathit{dispatch}}}
\newcommand{\complete}{\ensuremath{\mathit{complete}}}
\newcommand{\same}[1]{\ensuremath{\mathit{same}(#1)}}
\newcommand{\inputs}{\ensuremath{I}}
\newcommand{\outputs}{\ensuremath{O}}
\newcommand{\system}{\ensuremath{S}}
\newcommand{\components}{\ensuremath{C}}
\newcommand{\component}{\ensuremath{c}}
\newcommand{\schedule}{\ensuremath{\phi}}
\newcommand{\valid}{\ensuremath{\mathit{valid}}}
\newcommand{\dpred}{\ensuremath{\delta^\phi}}
\newcommand{\dispred}{\ensuremath{\mathbb{D}^\phi}}
\newcommand{\compred}{\ensuremath{\mathbb{C}^\phi}}
\newcommand{\dispredp}{\ensuremath{\mathbb{D}^{\phi\prime}}}
\newcommand{\compredp}{\ensuremath{\mathbb{C}^{\phi\prime}}}

The AGREE specification language is based on stream concepts, and
operators, from the Lustre language \cite{10.1145/41625.41641}. Thus
the setting is synchronous dataflow where the inputs and outputs of
components are streams, and contracts express relationships between
input and output streams. When considering a system of components,
data flows through the components in dependency order, with inputs
being propagated to outputs through all contracts until they stabilize
(can't propagate further). Therefore, the subcomponent contracts, and
thus the top-level model, must be acyclic. (An apparent syntactic
cycle, where a component is linked back to itself, may be broken
temporally by inserting delay elements.)  Once the data propagation
has stabilized, the model proceeds to the next input data in the input
streams. The semantics do not model computation or communication
delay. The output of one contract is seen at the input of any
downstream contract in the same step of the input data stream.

From the system and component contracts, AGREE generates a set of
verification conditions to show that a system's component
implementation is correct~\cite{agree2013}.  The AGREE model checker
is then invoked to prove or disprove the verification
conditions. Contracts and verification conditions are expressed in
\emph{past-time linear temporal logic} (PLTL).\footnote{KLS: citation
needed.}  PLTL is a logic enhanced with temporal operators able to
reason about the truth values of formulas through time.  Its semantics
are defined relative to a point in time $i$ and a finite trace of
system states $\pi = s_0, s_1, \ldots, s_i$.

The two PLTL operators necessary for the AGREE generated verification
conditions are $\globally$ (globally) that looks forward in time along
the trace and $\historically$ (historically) that looks backward in
time along the trace.  These are defined as
\begin{eqnarray*}
 (\pi, i) \models \globally(f) & \iff & \forall j \ge i, (\pi, j) \models f \\
(\pi, i) \models \historically(f) & \iff & \forall 0 \le j \le i, (\pi, j) \models f
\end{eqnarray*}
The $\models$-operator is read as \emph{satisfies}.  A trace at a
moment in time satisfies $\globally(f)$ if and only if it satisfies
$f$ in the current and all future states of $\pi$.  $\globally(f)$ is
invariant from the current moment into the future and $\historically$ is
invariant from the beginning of the trace to the current moment.

A \emph{system} $\system = (\inputs, \outputs, \assumes,
\guarantees,C)$, where $\inputs$ is the input set, $\outputs$ is the
output set, $\assumes$ is the set of assumptions, $\guarantees$ is the
set of guarantees, and $C$ are subcomponents.  A subcomponent
$\component$ is, hierarchically, also a system, and may be designated
by its own tuple $(\inputs_\component, \outputs_\component,
\assumes_\component, \guarantees_\component, C_c)$.  From the
components and their connections, $\mathbb{I}_\component$ is defined
to be the set of components providing input to some component
$\component$ in the system, and $\mathbb{O}$ is defined to be the set
of components that provide the output for the system.  A system $S$ is
\emph{correct} if and only if for all components $c \in C$ the
following two verification conditions hold:
\begin{equation}
            \globally(\historically(\assumes \wedge
            \bigwedge_{\component^\prime \in \mathbb{I}_\component} P_{\component^\prime})
            \implies \assumes_\component)
\end{equation}
\begin{equation}
            \globally(\historically(\assumes \wedge
            \bigwedge_{\component^\prime \in \mathbb{O}} \guarantees_{\component^\prime})
            \implies \guarantees)
\end{equation}
Condition (1) verifies the input assumptions on each component under
the system assumptions and upstream component guarantees.  It checks
if the component guarantees and system assumptions are strong enough
to imply input assumptions on all immediate downstream components.
Condition (2) checks the output guarantees of the system under the
system assumptions and component guarantees that provide the output.
It checks if the guarantees on components providing primary outputs
are strong enough to imply the system guarantees.

If all the verification conditions hold (AGREE uses $k$-inductive
model checking to automatically prove or disprove each generated
verification condition), then the system is said to be \emph{correct},
meaning that the system composition meets input assumptions at each
input as well as the guarantees on the system output. A consequence of
this result is that $\globally(\historically(\assumes) \implies
\guarantees)$ holds for the system contract.

The expanded property lists in \figref{fig:example-certificate} and
\figref{fig:hardened-certificate} are the results from verifying or
disproving the above verification conditions.  The additional
unexpanded results at the bottom of the figures prove
\emph{self-consistency} in the contracts.  It is not uncommon to
accidentally write contracts that are self-contradicting.  For
example, a contract may guarantee an output be two different values in
the same moment of time.  AGREE generates additional verification
conditions that prove each component contract, and the composition of
contracts, self-consistent.

\subsection{Syntax and semantics of AGREE}
\label{agree-semantics}

We now give an overview of a formal model for AGREE. This provides a
setting in which we are able to relate the contract correctness
results discussed above, obtained via model-checking, with the code
generated from high-assurance contracts.  The syntax of AGREE is
essentially that of quantifier-free first order predicate logic
supplemented with a few temporal operators. The terms (\emph{e}) are
arithmetic expressions built from variables ($v$) and numeric and
boolean literals ($c$), while formulas (\emph{b}) are built using
logical connectives from atomic formulas (\emph{a}) based on the
familiar comparison operators.
\[
\begin{array}{rcl}
e & ::= & v \mid c \mid e \;\set{+,*,/}\; e \\
a & ::= & e\; \set{=,<}\; e \\
b & ::= & v \mid c \mid a \mid \neg b
            \mid b \; \set{\land,\lor,\imp,\iff}\; b
\end{array}
\]

There is also a conditional, $\itelse{b}{(-)}{(-)}$, for both terms
and formulas. Lastly, there are temporal operators $\konst{pre}(-)$,
\emph{delay} $(-) \to (-)$, and $\konst{Hist}(-)$.

The semantics of terms and formulas is in terms of \emph{streams of
values}. Values encompass at least booleans and numbers, but can be
readily extended to include records and arrays. A value stream is a
total function from time (natural numbers) to values:
\[
 \konst{stream} = \mathbb{N} \to \konst{value}
\]
Given an \emph{environment} $E : \konst{name} \mapsto \konst{stream}$
binding variable names to value streams, the semantics $\sem{-}^E_t$
of terms and formulas defines the meaning of compound syntax in terms
of the meaning of subexpressions. The value of a variable $v$ at time
$t$ is found by looking up the stream bound to $v$ in $E$ (call it
$s$) and returning $s_t$. Some clauses of the semantics follow,
omitting the temporal operators:
\[
\begin{array}{rcl}
\sem{v}^E_t & = & E(v)(t) \\
\sem{c}^E_t & = & c \\
\sem{e_1 + e_2}^E_t & = & \sem{e_1}^E_t + \sem{e_2}^E_t \\
   & \cdots & \\
\sem{b_1 \land b_2}^E_t & = & \sem{b_1}^E_t \land \sem{b_2}^E_t \\
   & \cdots & \\
\end{array}
\]

The temporal operators deal with time in more significant ways. The
value of $\konst{pre}(e)$ at time $t$ is the value of $e$ at time
$t-1$ (at time zero, \konst{pre} is undefined).  A delay $e_1 \to e_2$
temporally ``shifts'' $e_2$ by means of prepending the first element
of $e_1$ to it.

\[
\begin{array}{rcl}
\sem{\konst{pre}(e)}^E_t & = & \sem{e}^E_{t-1}, \mathrm{when}\ t > 0 \\
\sem{e_1 \to e_2}^E_t & = & \itelse{t=0}{\sem{e_1}^E_0}{\sem{e_2}^E_t} \\
\sem{\konst{Hist}(b)}^E_t & = & \forall n \leq t.\; \sem{b}^E_n
\end{array}
\]

Although basic, these definitions can be used to define higher-level
operators from PTLTL, such as \konst{Once} and \konst{Since}.

\subsection{Code contracts}
\label{code-contracts}

Generally, AGREE specifications do not describe the computation that a
component performs. This is entirely by design: AGREE is intended to
reason about component behavior solely at the specification
level. However, the syntax of AGREE specifications provides enough
expressiveness to support the notion of a \emph{code contract}: a
contract from which an implementation can be extracted. First we must
discuss a class of guarantees---\emph{output guarantees}---which
determine the values on all output ports of a component.

\begin{definition}[Output guarantee]
An \emph{output guarantee} is a stylized guarantee that fully
specifies the data written to an output port. There are three
possibilities according to whether the output port $p$ is
a \konst{data} port, an \konst{event} port, or an \konst{event data}
port:
\[
\begin{array}{ll}
\konst{data}: &  p = \mathit{e} \\
\konst{event}: &  \konst{event} (p) = \mathit{b} \\
\konst{event data}: & \itelse{b}{\konst{event} (p) \land p = e}{\neg \konst{event}(p)} \\
\end{array}
\]
\end{definition}

Informally, a code contract treats its \konst{eq} ``statements'' as
defining a list of assignments to state variables, and its output
guarantees as directives for producing output.

\begin{definition}[Code contract] A
  leaf component of the form $(I,O,A,P,\emptyset)$ is a
  \emph{code contract} if $\mathit{Eqs} \cup G \subseteq P$, where
\[\mathit{Eqs} = \set{v_1 = e_1, \cdots , v_n = e_n} \] is a non-empty set
of \konst{eq} statements and $G$ is the set of output guarantees, one for
each element of $O$. In the interpretation as code, the order of
elements of $\mathit{Eqs}$ is important, and is simply taken to be the
occurrence order of the \konst{eq} statements in the syntax. Thus we
will work with the
\emph{list} of equations $\mathit{Eqs} = [v_1 = e_1; \cdots ; v_n = e_n]$.
\end{definition}


\section{Syntax and semantics of AGREE}
\label{agree-semantics}
\begin{comment}
This section gives an abstract mathematical overview of AGREE's
compositional reasoning system, defining assume-guarantee contracts
and the verification conditions that AGREE checks in order to
modularly establish overall system correctness at the model level.
The syntax and semantics of AGREE are sketched in some detail.  The
discussion then becomes more concrete, showing---by example---how the
AGREE contract language is used to specify the cyber-hardened system
of \figref{fig:hardened}. The discussion proceeds from the
system-level contract to the filter and monitor components, which rely
on the notion of a \emph{code contract} to support code generation.
\end{comment}

We now give an overview of a formal model for AGREE syntax and
semantics.  The discussion then becomes more concrete, showing by
example, how the AGREE language is used to specify the properties of
the SW system introduced in \secref{sec:example}.  Any implementation
of the SW system must guarantee the specification output under the
system assumptions.

The syntax of AGREE is essentially that of quantifier-free first order
predicate logic supplemented with a few temporal operators. The terms
(\emph{e}) are arithmetic expressions built from variables ($v$) and
numeric and boolean literals ($c$), while formulas (\emph{b}) are
built using logical connectives from atomic formulas (\emph{a}) based
on the familiar comparison operators.
\[
\begin{array}{rcl}
e & ::= & v \mid c \mid e \;\set{+,*,/}\; e \\
a & ::= & e\; \set{=,<}\; e \\
b & ::= & v \mid c \mid a \mid \neg b
            \mid b \; \set{\land,\lor,\imp,\iff}\; b
\end{array}
\]
AGREE includes a conditional, $\itelse{b}{(-)}{(-)}$, for both terms
and formulas. To support port handling, AGREE defines a predicate
$\konst{event}(p)$, which tests if an event (or event-data) port $p$ is
signaled. Lastly, AGREE supports temporal operators \emph{previously}
$\konst{pre}(-)$, \emph{followed-by} $(-) \to (-)$, and historically
$\konst{Hist}(-)$.

\subsection{Semantics}
The semantics of terms and formulas, as mentioned in the previous
section, is in terms of \emph{streams of values}. Values encompass at
least booleans and numbers, but can be readily extended to include
records and arrays. A value stream is a total function from time
(natural numbers including 0) to values:
\[
 \konst{stream} = \mathbb{N}_0 \to \konst{value}
\]
Given an \emph{environment} $E : \konst{name} \mapsto \konst{stream}$
binding variable names to value streams, the semantics $\sem{-}^E_t$
of terms and formulas defines the meaning of compound syntax in terms
of the meaning of subexpressions. The value of a variable $v$ at time
$t$ is found by looking up the stream bound to $v$ in $E$ (call it
$s$) and returning $s_t$. Some clauses of the semantics follow,
omitting the temporal operators:
\[
\begin{array}{rcl}
\sem{v}^E_t & = & E(v)(t) \\
\sem{c}^E_t & = & c \\
\sem{e_1 + e_2}^E_t & = & \sem{e_1}^E_t + \sem{e_2}^E_t \\
   & \cdots & \\
\sem{b_1 \land b_2}^E_t & = & \sem{b_1}^E_t \land \sem{b_2}^E_t \\
   & \cdots & \\
\end{array}
\]

The temporal operators deal with time in more significant ways. The
value of $\konst{pre}(e)$ at time $t$ is the value of $e$ at time
$t-1$ (at time zero, \konst{pre} is undefined).  A followed-by, $e_1
\to e_2$, evaluates to $e_1$ at time zero; otherwise it is $e_2$.
Followed-by is typically used to give a value to a stream in the first
step so that subsequent steps are able to use $\konst{pre}(e)$.
\[
\begin{array}{rcl}
\sem{\konst{pre}(e)}^E_t & = & \sem{e}^E_{t-1}, \mathrm{when}\ t > 0 \\
\sem{e_1 \to e_2}^E_t & = & \itelse{t=0}{\sem{e_1}^E_0}{\sem{e_2}^E_t} \\
\sem{\konst{Hist}(b)}^E_t & = & \forall n \leq t.\; \sem{b}^E_n
\end{array}
\]

Although basic, these definitions can be used to define higher-level
past-time operators from PLTL such as \konst{Once} and \konst{Since}.
These can be expressed in a recursive style as follows:
%% \[
%%    \begin{array}{rcl}
%%       \sem{\konst{Once}(b)}^E_t & = & \sem{b \vee (\konst{false} \to \konst{pre}(o))}^E_t \\
%%       \sem{\konst{Since}(a, b)}^E_t & = & \sem{b \vee (a and (\konst{false} \to \konst{pre}(o)))}^E_t \\
%%    \end{array}
%% \]
\[
\begin{array}{rcl}
\konst{Once}(b) & = & b \to b \vee \konst{pre}(\konst{Once}(b)) \\
\konst{Since}(a, b) & = & b \to b \lor (a \land \konst{pre}(\konst{Since}(a, b))
\end{array}
\]
\noindent $\konst{Once}(b)$ is true if $b$ has ever been true before, or in, the
current moment.  $\konst{Since}(a, b)$ is true if $a$ has been true in
all moments up to the present one since $b$ most recently became true.
All other past-time operators of PLTL can be defined similarly \cite{monitor}.

\paragraph{AGREE notation.}
AGREE uses ASCII notation for the mathematical syntax we have
defined. The following table spells out the mapping.
\[
\begin{tabular}{l|l|l}
 Operation & Math. notation & AGREE \\ \hline
 followed-by &  $e \to e$          & \verb+e -> e+  \\
 logical implication &  $e \imp e$ & \verb+e => e+ \\
 if-and-only-if &  $e \iff e$      & \verb+e <=> e+ \\
 logical and &  $e \land e$      & \verb+e and e+ \\
 logical or &  $e \lor e$      & \verb+e or e+ \\
 logical not & $\neg b$      & \verb+not b+ \\
 variable definition & $v = e$     & \verb+eq v e+ \\
 Previously  & $\konst{pre}(b)$ & \verb+pre(b)+ \\
 Historically  & $\konst{Hist}(b)$ & \verb+Historically(b)+ \\
\end{tabular}
\]


\subsection{Formal contract for the example system}
\newsavebox{\sw}
\begin{lrbox}{\sw}
\begin{lstlisting}[style=agree,numbers=left]
eq req : bool = event(AutomationRequest);*\label{line:sw-event-def-start}*
eq avl : bool = event(AirVehicleLocation);
eq wp : bool = event(Waypoint);
eq strt: bool = event(Start);
eq alrt : bool = event(Alert);*\label{line:sw-event-def-end}*

assume "Automation requests are well-formed" : *\label{line:sw-assume-1}*
  req => WELL_FORMED_AUTOMATION_REQUEST(AutomationRequest);
assume "Air vehicle locations are well-formed" : *\label{line:sw-assume-2}*
  avl => WELL_FORMED_WAYPOINT(AirVehicleLocation);    
assume "One automation request in flight at a time" : *\label{line:sw-assume-3}*
  true -> 
  (req => pre(Historically(not req) or Since(not req, strt)));
      
guarantee "Waypoints coincide with air vehicle locations": *\label{line:sw-guarantee-1}*
  wp => avl;
guarantee "Starts include a new waypoint" : *\label{line:sw-guarantee-2}*
  strt => wp;
guarantee "Waypoints are well-formed" :  *\label{line:sw-guarantee-3}*
  wp => WELL_FORMED_WAYPOINT(Waypoint);
guarantee "Starts within one cycle of requests if not alerting" : *\label{line:sw-guarantee-4}*
  (strt => ((not alrt) and req)) -> 
  (strt => ((not alrt) and (req or pre(req))));
guarantee "Alert if not started within one cycle of requests" : *\label{line:sw-guarantee-5}*
    true -> ((pre(req and not strt) and not strt) => alrt);
guarantee "Once alerted always alerted" : *\label{line:sw-guarantee-6}*
  not alrt or (Once(alrt) and alrt);
\end{lstlisting}
\end{lrbox}

\begin{figure}
  \begin{center}
    \scalebox{0.62}{\usebox{\sw}}
  \end{center}
  \caption{The SW component contract.}
  \label{fig:sw}
\end{figure}

The AGREE specification language uses stream concepts, and operators, from the Lustre language \cite{10.1145/41625.41641}.
As with Lustre, the semantics are synchronous dataflow where the inputs, outputs, and expressions are data streams that comply with the input assumptions.
Contracts are evaluated in dependency order with inputs being propagated to outputs through all contracts until they stabilize; as such, the contracts, and thereby the top-level model, must be acyclic.\footnote{An apparent syntactic cycle, where a component is linked back to itself, may be broken temporally by inserting delay elements.}

Once the contracts have stabilized, the model takes a synchronous step to the next input data in the stream.
The semantics do not model computation or communication delay.
The output of one contract is seen at the input of any downstream contract in the same step of the input data stream.
The language is best introduced through example.

The AGREE specification for the SW component in the example of Section~\ref{sec:example} is given in \figref{fig:sw}.
The specification uses \texttt{eq} statements to define variables local to the contract specification.
For example, \lineref{line:sw-event-def-start} defines the \texttt{req} variable to be equivalent to the \texttt{event} expression.

All the named ports in the corresponding AADL component are in the scope of the specification, and there are additional implicit boolean \emph{event} inputs (or outputs) associated with event ports.
An \texttt{event} expression refers to that implicit input (or output) boolean value and is true when data is present on the named port and false otherwise.
\linesref{line:sw-event-def-start}{line:sw-event-def-end} create local variables that are true when data is present on the corresponding event ports for the component.
The local variables here are purely for convenience in writing the specification.

The \texttt{assume} statement is a string description followed by a predicate.
Those on \lineref{line:sw-assume-1} and \lineref{line:sw-assume-2} are implications requiring that when data is present it is well-formed.
The well-formed predicates themselves are defined elsewhere using AGREE functions.

The assumption on \lineref{line:sw-assume-3} constrains when a request can arrive by reasoning about the \emph{state} of the contract.
Streams are associated with every expression in AGREE, meaning that expressions are defined through time, and are therefore stateful.
When writing assumptions and guarantees, it is important to differentiate pre-state, before a contract updates its state, and post-state, after a contract updates its state, in response to the current input. AGREE provides the \texttt{pre} operator to make that distinction.

The \texttt{pre} operator returns the previous value of the enclosed expression by looking back one step in time on the expression's stream.
In general, assumptions regarding state should be evaluated in the pre-state of the component with the current inputs, and guarantees regarding state should be evaluated on the post-state of the component given the current inputs.
Guarantees should use the \texttt{pre} operator anytime they need to reason about the current state relative to the previous state. 

The assumption on \lineref{line:sw-assume-3} uses a \emph{followed-by} operator, \texttt{->}, to \emph{guard} the \texttt{pre} operator at time 0.
A followed-by expression defines what happens at the first instance of the stream, time 0, and then what follows after. 
Here the assumption is \texttt{true} at time 0 and then it is the truth value of the implication in all future instances.
The followed-by guards the \texttt{pre} because expressions are undefined before time 0 and the followed-by prevents the \texttt{pre} from being evaluated before time 1.
So after time 0, the assumption depends on the presence of a request and the pre-state of the component.

The stream semantics in AGREE mean that it is possible to use PLTL operators that look back in time.
The \texttt{Historically} operator is the same as $\historically$ defined previously; it is true if the enclosed expression has been true in all moments up to the present one. 
The \texttt{Since} operator is true if the first expression has been true in all moments up to the present one since the second expression most recently became true.
The assumption on \lineref{line:sw-assume-3} is that if there is an incoming request, it is either the very first one, \texttt{\textbf{Historically}(\textbf{not} req)}, or it is after the component has output a start event in response to a previous request, \texttt{\textbf{Since}(\textbf{not} req, strt)}--\emph{not request since start}.

The guarantees on \linesref{line:sw-guarantee-1}{line:sw-guarantee-3} coincide events and assert well-formed output.
The guarantees on \linesref{line:sw-guarantee-4}{line:sw-guarantee-6} define temporal properties of the component.
\lineref{line:sw-guarantee-4} insists that a start happens with a request or one step after a request.
The guarantee uses the assumption on \lineref{line:sw-assume-3} and does not check for two requests in a row without a start as the assumption precludes that input behavior.
It differentiates with the followed-by operator what is required at time 0, the start must coincide with the request, with what is required after time 0, the start must coincides with the request or is a response to a request one step earlier.
The guarantee also does not force the start to always happen, it only says that if it does happen, it is under the defined conditions.

\lineref{line:sw-guarantee-5} forces the alert to sound if the start does not arrive within the one-step bound.
Together with \lineref{line:sw-guarantee-4} the contract model allows for non-alerting and alerting behavior.
The final guarantee on \lineref{line:sw-guarantee-6} defines additional behavior for the alert.

The \texttt{Once} operator is true if the enclosed expression has ever been true before the current moment.
\lineref{line:sw-guarantee-6} ensures that once alerted always alerted because either there has never been an alert, \texttt{\textbf{not} alrt}, or there has been an alert sometime in the past and so it is alerting now, \texttt{\textbf{Once}(alrt) \textbf{and} alrt}.
These six guarantees define the behavior of the SW component under the three assumptions.


\ifREVISIONS
\subsection{Revisions}
\begin{compactitem}
  \item \sout{Add in the AGREE specification for the system}
  \item \sout{Add in the AGREE verification conditions (e.g., similar to those in the Scheduled Components NFM submission)}
  \item \sotu{Add reference to the Liskov principle of safe-substitution when stating that the system contract is a sound abstraction of the implementation.}
  \item Add back in a few sentences on the contract for the filter since those have been removed with the addition of the formal definitions.
\end{compactitem}
\fi

\section{Code contracts for cyber components}
\label{sec:code-contracts}
%% \begin{figure}
%%   \begin{center}
%%     \begin{tabular}{c}
%%     \includegraphics[scale=0.3]{dialogue.png}
%%     \end{tabular}
%%   \end{center}
%%   \caption{BriefCASE dialogue for filter transformation.}
%%   \label{fig:dialogue}
%% \end{figure}


Generally, AGREE contracts do not describe the computation that a
component performs. This is entirely by design: AGREE is intended to
reason about component behavior solely at the specification
level. However, the syntax of AGREE provides enough expressiveness to
support the notion of a \emph{code contract}: a contract from which an
implementation can be extracted. First we must discuss a class of
guarantees---\emph{output guarantees}---which determine the values on
all output ports of a component.

\begin{definition}[Output guarantee]
An \emph{output guarantee} is a stylized guarantee that fully
specifies the data written to an output port. There are three
possibilities according to whether the output port $p$ is
a \konst{data} port, an \konst{event} port, or an \konst{event data}
port:
\[
\begin{array}{ll}
\konst{data}: &  p = \mathit{e} \\
\konst{event}: &  \konst{event} (p) = \mathit{b} \\
\konst{event data}: & \itelse{b}{\konst{event} (p) \land p = e}{\neg \konst{event}(p)} \\
\end{array}
\]
\end{definition}

Informally, a code contract treats its \konst{eq} ``statements'' as
defining a list of assignments to state variables, and its output
guarantees as directives for producing output.

\begin{definition}[Code contract] A
  leaf component of the form $(I,O,A,P,\emptyset)$ is a
  \emph{code contract} if $\mathit{Eqs} \cup G \subseteq P$, where
\[\mathit{Eqs} = \set{v_1 = e_1, \cdots , v_n = e_n} \] is a non-empty set
of \konst{eq} statements and $G$ is the set of output guarantees, one for
each element of $O$. In the interpretation as code, the order of
elements of $\mathit{Eqs}$ is important, and is simply taken to be the
occurrence order of the \konst{eq} statements in the syntax. Thus we
will work with the
\emph{list} of equations $\mathit{Eqs} = [v_1 = e_1; \cdots ; v_n = e_n]$.
\end{definition}


The semantics of Section \ref{agree-semantics} supports a formal
connection between the original contract---and verification results of
the AGREE model checker being run on it---and code generated from
specifications of leaf level cyber-components. It also provides the
root meaning at the base of a chain of translation steps moving from
an AGREE contract to a CakeML executable.

The first step in the chain maps the contract to a code-focused
representation. Assume given a code contract
$(I,O,A,\mathit{Eqs} \cup G \cup P,\emptyset)$, environment $E$, and
time $t$. The \emph{evaluation} of $\mathit{Eqs}$ followed by the
evaluation of $G$ results in a new environment $E'$ where the stream
values for state variables and outputs have been computed for time
$t$. The step from $E$ to $E'$ is one full cycle in the repeated
evaluation of the component.

\begin{definition}[Evaluation]
We overload existing notation and write the evaluation of
$\mathit{Eqs}$ and then $G$ in environment $E$ at time $t$ as
$\sem{\mathit{Eqs}\cdot G}^E_t$. The evaluation of $v = e \in {\Eqs}$
is an environment transformation
\[
 \sem{v = e}^E_t = E[v(t) \mapsto\sem{e}^E_t] \ .
\]
modifying $E$ so that stream $v$ has value $\sem{e}^E_t$ at time $t$.
The list {\Eqs} is evaluated by folding the transformation left-to-right
through it. The transformation is similarly applied to the list of output
guarantees $G$, computing the values on output ports. The details are
omitted, being a bit messy because of the different kinds of output
ports available.
\end{definition}


\begin{definition}[Code contract correctness]
Assume code contract $(I,O,A,\mathit{Eqs} \cup G \cup P,\emptyset)$.
The contract is \emph{correct}, if for all $E$ and $t$, whenever the
assumptions (historically) hold in $E$ and evaluation steps from $E$
to $E'$, then the guarantees hold in $E'$:
\[
\sem{\konst{Hist}(A)}^E_t \land E' = \sem{\mathit{Eqs} \cdot G}^E_t \imp \sem{P}^{E'}_t
\]
\end{definition}

How does this notion of correctness integrate with the verification
conditions generated by AGREE? In \emph{contract correctness}, the
model checker is proving the following property
\[
\konst{Hist}(A) \imp P
\]
and in \emph{code contract correctness} we essentially prove a Hoare
triple of the form
\[
\set{\konst{Hist}(A)}\; (\mathit{Eqs} \cdot G) \; \set{P}
\]
This accords with intuition, namely that AGREE is a kind of
`program-free program logic'; by pulling out a program
$(\mathit{Eqs}\cdot G)$ from a code contract we find ourselves in the
domain of imperative program verification. For example, at this
point one could apply a verification condition generator to help break
down the proof obligation. In order to use the code contract result in
system contract verification conditions, one has to show that the code
contract result implies the system verification condition.

\subsubsection*{Imperative code contracts.}
Stream-based evaluation allows values arbitrarily `deep in the past'
to be accessed by use of $\konst{pre}(-)$. However, the programming
languages targeted by SPLAT do not directly support streams.  We are
developing a translation to eliminate such deep temporal accesses by
the introduction of intermediate variables. The essential property the
translation needs to satisfy is a syntactic restriction ---call
it \emph{Pascalish}---that supports easy translation to standard
imperative languages.  The guiding intuition is that a Pascalish
{\Eqs} accesses at most one step in the past; also, accesses to a
past value of a variable are only possible until the variable is
updated. In informal terms, this means that ${\Eqs}\cdot G$ in a code
contract has the following properties:
\begin{itemize}
\item Followed-by only occurs at the top level of a variable definition, \eg, $v = e_1 \to e2$;
\item $\konst{pre}(e)$ is allowed only when $e$ is a variable; and
\item $\konst{pre}(v)$ is allowed only before the defining equation $v = e$ (and may occur in $e$).
\end{itemize}
For example, the Fibonacci sequence $1,1,2,3,5,8,13,\ldots$ can be
defined, albeit in a non-Pascalish manner, by
{\small
\begin{lstlisting}[style=agree]
  Fib = 1 -> pre(1 -> Fib + pre Fib)
\end{lstlisting}
}
\noindent or by (again non-Pascalish)
{\small
\begin{lstlisting}[style=agree]
  N = 0 -> 1 + pre(N)
  Fib = if N <= 1 then 1 else pre(Fib) + pre(pre(Fib))
\end{lstlisting}
}
\noindent or by (Pascalish):
{\small
\begin{lstlisting}[style=agree]
  N = 0 -> 1 + pre(N)
  F2 = 42 -> pre(F1)
  F1 = 42 -> pre(Fib)
  Fib = if N <= 1 then 1 else F1 + F2
\end{lstlisting}
}
(Note that the initial values of \verb+F1+ and \verb+F2+ are never
accessed, so 42 could be replaced by any number in their defining
equations.) A Pascalish ${\Eqs} \cdot G$ is straightforward to
implement in conventional programming languages, as we will see in
Section \ref{sec:synthesis}.

\subsection{Code contracts for the hardened system}

\newsavebox{\flt}
\begin{lrbox}{\flt}
  \begin{lstlisting}[style=agree,numbers=left] -- start user
    -- start user definitions
    eq policy : bool =
      WELL_FORMED_AUTOMATION_RESPONSE(Input);
    -- stop user definitions

    guarantee "Filter output is well-formed" :
      if event(Input) and policy then
        event(Output) and Output = Input
      else
        not event(Output);
  \end{lstlisting}
\end{lrbox}

\newsavebox{\mntr}
\begin{lrbox}{\mntr}
  \begin{lstlisting}[style=agree,numbers=left]
    const is_latched : bool = true;

    -- start user definitions
    assume "One automation request in flight at a time" : *\label{line:mon-assume}*
      true -> (req => pre(Historically(not req) or Since(not req, rsp)));

    const MAX_LATENCY : int = 1; *\label{line:mon-const}*

    eq rsp : bool = event(Response);
    eq req : bool = event(Request);

    eq preIsPending : bool = pre(isPending);*\label{line:mon-pre-pending}*
    eq isPending : bool = Since(not rsp, req and not rsp);*\label{line:mon-pending}*
    eq latency : int = 0 -> (if req then 0 else pre(latency) + 1);*\label{line:mon-latency}*

    eq policy : bool = (rsp => req) -> *\label{line:mon-policy}*
                         (    (isPending => latency < MAX_LATENCY)
                          and (rsp => (req or preIsPending)));
    -- stop user definitions

    eq alert : bool = (not policy) -> *\label{line:mon-alert}*
                        ((is_latched and pre(alert)) or not policy);

    guarantee "Alert port tracks alert variable" :
      event(Alert) = alert;
    guarantee "Output if not alerted" :
      if (not(alert) and rsp) then
          event(Output) and (Output = Response)
      else
          not (event(Output));
  \end{lstlisting}
\end{lrbox}

\begin{figure}
  \begin{center}
    \begin{tabular}{c}
      \scalebox{0.62}{\usebox{\flt}} \\
    \end{tabular}
  \end{center}
  \caption{Code contract for high-assurance filter.}
  \label{fig:filter}
\end{figure}

\begin{figure}
  \begin{center}
    \begin{tabular}{c}
    \scalebox{0.62}{\usebox{\mntr}} \\
    \end{tabular}
  \end{center}
  \caption{Code contract for high-assurance monitor.}
  \label{fig:monitor}
\end{figure}

As previously discussed, transformations carried out in the BriefCASE
interface have created a cyber-hardened system, protecting against
malicious AI behavior by the insertion of a high-assurance filter and
monitor between the AI and the WM (see \figref{fig:hardened}). The
filter protects the WM from malformed data and the monitor protects
the WM from (possibly malicious) spontaneous or delayed responses.
These components were added, one at a time, by
\begin{enumerate}
  \item selecting the connection in the model where the
    component is to appear,
  \item choosing the appropriate transformation, and
  \item specifying the relevant filtering or monitoring \emph{policy} (behavior).
\end{enumerate}

%% The system designer provides configuration parameters for the
%% transformation in a dialogue box.  The dialogue box for adding a
%% filter is shown in \figref{fig:dialogue}.  All high-assurance
%% components rely on a \emph{policy} to define behavior as seen in the
%% last field of the dialogue box.  A filter policy defines well-formed
%% data while a monitor policy defines an invariant over inputs and
%% outputs.  These policies can be stated directly in the wizard, or they
%% can be left blank and added later to the AGREE contract generated by
%% the transformation.

BriefCASE creates all the needed AADL for the new high-assurance
component, and its connections in the system implementation, as part
of the transformation.  It also creates a default code contract with
appropriate output guarantees leaving only the component behavior 
(policy) to be specified.

Code contracts for the filter and monitor are shown in
\figref{fig:filter} and \figref{fig:monitor}.
The filter is direct: it defines the policy according to the
well-formed requirement for data.  We now discuss the monitor in some
detail.

\subsubsection{Monitor specification}
The monitor policy is given by the \konst{eq}-statements in \linesref{line:mon-assume}{line:mon-policy} of
\figref{fig:monitor}.
\lineref{line:mon-alert}: The \texttt{alert} variable
(distinct from the \texttt{Alert} port) depends not just on the policy
but also on configuration details supplied when the component is
specified.  At configuration time, \texttt{is\_latched} has been set
to \konst{true}, thus making the alert persistent, meaning that once
the alert is raised, it is always raised; otherwise, it is the
complement of the policy value in the current step.

\lineref{line:mon-pre-pending}, \lineref{line:mon-pending}, and \lineref{line:mon-latency}:
The monitor policy is defined by marking when a request is not
satisfied, \texttt{isPending}, and
counting the number of steps between requests, \texttt{latency}.  
The \texttt{isPending} variable is true
whenever there is a request that does not coincide with a
response--to be read as \emph{there is no response since the most recent request without a corresponding response}.  The
\texttt{latency} variable starts at zero, resets on every request, and
otherwise increments by one. 
The \texttt{preIsPending} is needed to make the contract Pascalish since since the current and previous value of \texttt{isPending} are used in the definition of \texttt{policy}.

\lineref{line:mon-policy}: 
The policy makes sure the latency is bounded whenever there is a pending request, and it it makes sure that a response is a result of a request.
The initial step differentiates from later steps in only needing to be sure a response is a result of a request.

\lineref{line:mon-const}: The latency bound for the monitor is defined by the constant
\texttt{MAX\_LATENCY}.  Here the constant
is set to one to be consistent with the system specification from the
previous section.  At time zero the policy is trivial: a response
requires an accompanying request.  After time zero, if a request is
pending in the current time step, then the latency must still be under
the bound, and if there is a response, then it closes a request now or
a pending request from earlier in time.

\lineref{line:mon-assume}: The policy definition only works if there is never more than one
outstanding request at a time; otherwise, the latency counter resets
incorrectly.  That requirement is encoded in the assumption and is the same assumption used for the
system input.  The policy can be written without the assumption,
but the policy is much simpler to write with the assumption.

As shown in \figref{fig:hardened-certificate}, the cyber-hardened
system passes AGREE verification.  That means that the high-assurance
components protect the WM from the untrusted AI generating malformed
data, spontaneous responses, or delayed responses.  Synthesis must
generate faithful implementations of the code contracts for the AGREE verification result to apply to the
deployed system.

%%   Precise in this context means that the
%% implementations match the input-to-output relations defined in the
%% specifications.


\section{Testing Specifications with AGREE}
\label{sec:testing}
Writing and proving properties about code contracts is difficult because it is a formal process, and because it reasons about constraints that define the entire input space, and the entire output space, of a high-assurance component at once.
Case studies have shown that designers, as well as formal method practitioners, make mistakes writing these contracts, especially if the computation or temporal reasoning in the contract is complex.

We have observed three common types of mistakes in code contracts (aside from self-consistency, which is
already checked): vacuity, under-specification, and missed corner cases.
Vacuity is a common pitfall with implication when the left hand side is always false, making the implication true. 
Under-specification is very challenging, as \agr\ is not able to prove anything (or it can prove everything in the case of existential properties), so nothing is actually known after verification except that the contract is under-specified. 
Missed corner cases is a common problem to all disciplines.
It is an especially difficult challenge here because there is no direct way to exercise code contracts, \eg, there is no runtime \emph{per se}.

A tempting approach to exercising code contracts directly is to use \splt\ to create the target binary and then use that runtime with traditional testing.
Such an approach adds extra steps, including test harneses, that are not needed, as it is
possible to bridge the formal model in \agr, with its proof system, to
traditional validation using tests.
The intuition for that bridge comes from the compositional reasoning in AADL and \agr. 
Effectively, a code contract for a high assurance component should be a valid implementation of a test scenario.
In other words, the test scenario, as expressed in a test contract, is the desired behavior of the system, and the code contract implements that behavior.

A test contract defines the test scenario for the code contract to implement.
It does this by assuming the test input and guarantying the test output akin to a unit test. 
The code contract is used as the implementation, \eg, it is the test subject.
The \agr\ analysis then proves if the code contract, given the test inputs, is strong enough to prove the test contract outputs.
Here the test contract assumptions strengthen the code contract assumptions to a single input, and the test contract guarantees weaken, or leave unchanged, the code contract output for that test input, depending on the goal of the test.
In other words, according to Liskov substitution, \agr\ proves if the code contract is a safe substitution for the test contract.

\newsavebox{\tst}
\begin{lrbox}{\tst}
  \begin{lstlisting}[style=agree,numbers=left]
    process should_notAlertAndOutput_when_responseOneStepAfterRequest
      ... -- Omitted AADL features
      annex agree {**
        eq index : int = prev(index + 1, 0); *\label{line:tst-index}*
        
        assume "One Response one step after Request" : *\label{line:tst-assume-1}*
                ((index = 0) => not event(Response))
            and ((index = 1) => event(Response))
            and ((index >= 2) => not event(Response));
        assume "One Request" : *\label{line:tst-assume-2}*
            (event(Request) = true) ->
            (event(Request) = false);
        
        guarantee "Not Alert" : *\label{line:tst-guarantee-1}*
            not event(Alert);
        guarantee *\label{line:tst-guarantee-2}*
          "Output one step after Request at same time as Response" :
                ((index = 0) => not event(Output))
            and ((index = 1) => (event(Output) and Output = Response))
            and ((index >= 2) => not event(Output));
      **};
    end should_notAlertAndOutput_when_responseOneStepAfterRequest;
    
    process Implementation *\label{line:tst-imp-start}*
      should_notAlertAndOutput_when_responseOneStepAfterRequest.test
      subcomponents
        Monitor: thread CASE_Monitor_Thr.Impl; *\label{line:tst-imp-comp}*
      connections
        c00: port Response -> Monitor.Response;
        c01: port Request -> Monitor.Request;
        c02: port Monitor.Alert -> Alert;
        c03: port Monitor.Output -> Output;
    end should_notAlertAndOutput_when_responseOneStepAfterRequest.test; *\label{line:tst-imp-end}*
  \end{lstlisting}
\end{lrbox}

\begin{figure}
  \begin{center}
    \begin{tabular}{c}
    \scalebox{0.60}{\usebox{\tst}} \\
    \end{tabular}
  \end{center}
  \caption{A unit test for the monitor with its test contract.}
  \label{fig:test}
\end{figure}

\figref{fig:test} is a unit test, with its test contract, for the monitor in \figref{fig:hardened}.
The test contract checks if the monitor output coincides with the response that comes one step after the request.
It also checks that the alarm is not output.
\begin{compactitem}
\item \lineref{line:tst-index}: An index to use to define the test input and expected test output through time.
\item \lineref{line:tst-assume-1} and \lineref{line:tst-assume-2}: The test input. These define when the request and response message events should appear relative to the index.
\item \lineref{line:tst-guarantee-1} and \lineref{line:tst-guarantee-2}: The expected test output. 
The test additionally proves that nothing happens after the initial test input and expected output, \ie, no more input is provided and no more output is generated.
\item \linesref{line:tst-imp-start}{line:tst-imp-end}: The implementation with the component under test (\lineref{line:tst-imp-comp}).
\end{compactitem}
\figref{fig:hardened-certificate}(a) is the \agr\ output for the test showing that it passes. 

Test contracts are effective in detecting vacuity, under-specified behavior, and missed corner cases.
They enable designers to iteratively develop code contracts inside OSATE with \agr, and they build assurance that the code contract computes the desired behavior with black-box, white-box, and other test coverage metrics.
These same tests can be synthesized to the backend implementation for traditional testing, if for example, the code contract is a model of an existing legacy binary.

%% \section{Synthesis from code contracts}
%% \label{sec:semantics}
%% The semantics supports a formal connection between the results of the
AGREE model checker and code generated from specifications of leaf
level cyber-components. It also provides the root meaning at the base
of a chain of translation steps moving from AGREE specifications to
CakeML executables.

Code generation is a multi-step transformation that starts with a code
contract and ends with an executable. The first step maps the contract
to a code-focused representation. Assume given code contract
$(I,O,A,\mathit{Eqs} \cup G \cup P)$, environment $E$, and time
$t$. The evaluation of $\mathit{Eqs}$ followed by the evaluation of
$G$ results in a new environment $E'$ where the stream values for
state variables and outputs have been computed for time $t$. The step
from $E$ to $E'$ is one full cycle in the repeated evaluation of the
component.

\begin{definition}[Evaluation]
We overload existing notation and write the evaluation of
$\mathit{Eqs}$ and then $G$ in environment $E$ at time $t$ as
$\sem{\mathit{Eqs}\cdot G}^E_t$. Each $v = e \in {\Eqs}$ is treated as
an environment transformation
\[
 \sem{v = e}^E_t = E[v(t) \mapsto\sem{e}^E_t] \ .
\]
modifying $E$ so that stream $v$ has value $\sem{e}^E_t$ at time $t$.
{\Eqs} is evaluated by folding the transformation left-to-right
through it. The transformation is also applied to the list of output
guarantees $G$, computing the values on output ports. The details are
omitted, being a bit messy because of the different kinds of output
ports available.
\end{definition}


\begin{definition}[Code contract correctness]
Assume given code contract $(I,O,A,\mathit{Eqs} \cup G \cup P)$ and
environment $E$.  The contract is \emph{correct}, if for all $t$,
whenever the assumptions (historically) hold in $E$ and evaluation
steps from $E$ to $E'$, then the guarantees hold in $E'$:
\[
\sem{\konst{Hist}(A)}^E_t \land E' = \sem{\mathit{Eqs} \cdot G}^E_t \imp \sem{P}^{E'}_t
\]
\end{definition}
\footnote
{KLS: At this point we would like to assert that the AGREE ``consistency
 check'' for a contract implies code contract correctness.}

The evaluation semantics given above is based on the stream
computation model, where values `deep in the past' can be accessed in
computations. However, the conventional program languages we wish to
target do not support such a view. Thus code generation is only
defined for a class of \emph{well formed} equations.

\begin{definition}[Wellformedness]

\end{definition}

\subsubsection{Example}


\ifREVISIONS
\subsection{Revisions}
\begin{compactitem}
  \item Add in intuitive definition or AGREE leaf-component semantics
  \item State well-formed theorem
  \item State correctness theorem or any key theorems to the synthesis proof
\end{compactitem}
\fi

\section{Synthesis to CakeML}
\label{sec:synthesis}
Synthesis maps from model and specifications to code. The SPLAT tool
traverses the system architecture looking for occurrences of high
assurance components specified by code contracts; for each such
occurrence it generates a CakeML program. More generally, SPLAT supports
three kinds of behavioral specification from which to generate
code. In increasing order of expressiveness, these are:

\paragraph{Regular expressions.} Often the well-formedness of data on connections
can be enforced by matching against a regular expression. In other
words, the data can be characterized by a \emph{regular language}. As
is well known, regular expressions can be translated to efficient
deterministic finite state machine (DFA)
implementations. In \cite{formal-filter-synth-langsec,case-verified-filter}
we report on the creation and application of filters from a verified
regular expression-to-DFA compiler.

\paragraph{Contiguity types\cite{contiguity-types}.}
A contiguity type, like regular expressions, is effective at
describing data at the network message level, \eg, as flat
strings. The notation is however more expressive, being able to
declare many kinds of so-called \emph{self-describing} messages, \ie,
those where message structure is dependent on data held elsewhere in
the message. Figure \ref{fig:filter-spec} gives a contiguity type
specification for the welll-formed waypoints of our example system. We
have formalized and verified a contiguity type matcher. The matcher
code has been used in our case studies.

\paragraph{AGREE code contracts.} When full computational power is needed, we can use the
expressive power of AGREE code contracts to write arbitrarily complex
filter or monitor operations, as we have already seen. Our formal
model of AGREE syntax and semantics is applied to extract code
contracts and map them to equivalent logic functions.

\noindent For each of these three approaches, we take the following steps to
arrive at an executable.

\begin{enumerate}
\item Map from specification to computable function in logic, producing
a proof of equivalence.
\item Add parsers and printers (specified by contiguity types) for input and output.
\item Map to CakeML and add calls to input/output library routines via
      the CakeML foreign function interface.
\item Invoke the CakeML compiler.
\end{enumerate}

%% Starting from the semantics of the code contract described in
%% Section \ref{sec:code-contracts} a sequence of semantics-preserving
%% steps are taken:
%% \begin{itemize}
%% \item $\sem{{\Eqs}\cdot G}$ yields a function on arbitrary depth streams
%% \item transform to Pascalish
%% \item transform to logic function
%% \item attach buffer handling to inputs and outputs (contig type parsing)
%% \item translate to CakeML
%% \item run CakeML compiler
%% \end{itemize}

In the following, we examine further details in both filter and
monitor synthesis.

\subsection{Filter Generation}

A filter is intended to be simple, although it may make deep semantic
checks. A filter has one input port and one output; messages on the
input that the filter policy admits pass unchanged to the output port;
all others are dropped (not passed on). We have investigated two kinds
of filter. In the first, a relatively shallow scan of the input
suffices to enforce the policy. For example, we have used the
expressive power of regular expressions and Contiguity
Types \cite{contiguity-types} to enforce \emph{lightweight} bounds
constraints on GPS coordinates in UxAS messages. On the other hand, a
filter may need to parse the input buffer into a data structure
specified in AGREE and apply a user-defined \emph{wellformedness}
property, also specified in AGREE, to the data. Arbitrarily complex
wellformedness checks can be made in this
way. \figref{fig:filter-spec} shows a combination where the checking
specified by {\small\verb+WELL_FORMED_AUTOMATION_RESPONSE+} depends on
an underlying check specified by the contiguity type checking bounds
on waypoints.

The verdict of a filter is made and performed within one thread
invocation. Thus, in its given time slice, the
following steps must be completed:

\begin{enumerate}

\item The filter checks to see if there is any input available.  If there is none
then it yields control; otherwise:

\item The input is read (and parsed if need be);

\item The wellformedness predicate is evaluated on the input;

\item If the predicate returns \konst{true} then the input buffer
 is copied to the output, otherwise no action is taken; and

\item The filter yields control.
\end{enumerate}

\begin{remark}[Partiality]

Partiality is an important consideration: steps 2 and 3 above can
fail; the data might not be parseable or the wellformedness
computation could be badly written and fail at runtime. In such cases,
the filter should recover and yield control without passing the input
onwards. In these cases, the filter is behaving as it should, but we
must also guard against situations in which a \emph{correctly
specified} filter fails at runtime. This kind of defect arises when
the filter \emph{ought} to accept a message, but lack of resources
results in the filter failing to do so. For example, the parse of a
message might need more space than has been allocated; another example
could be if the time slice provided by the scheduler is too short for
the wellformedness computation to finish. Thus resource bounds need to
be included in the correctness argument. Some preliminary work on this
has been done in CakeML \cite{cakeml-space-cost}.

\end{remark}


\newsavebox{\contig}
\begin{lrbox}{\contig}
\begin{lstlisting}[style=myML]
  Waypoint =
    {Latitude  : f64
     Longitude : f64
     Altitude  : f32
     Check     : Assert
      (~90.0 <= Latitude and Latitude <= 90.0 and
       ~180.0 <= Longitude and Longitude <= 180.0 and
       1000.0 <= Altitude and Altitude <= 15000.0)}

  AutomationResponse =
    {TaskID : i64
     Length : u8
     Waypoints : Waypoint [3]}

 fun WELL_FORMED_AUTOMATION_RESPONSE(aresp) =
   (forall wpt in aresp.Waypoints, WELL_FORMED_WAYPOINT(wpt))
   and ... ;
\end{lstlisting}
\end{lrbox}

\begin{figure}
  \begin{center}
    \begin{tabular}{c}
      \scalebox{0.60}{\usebox{\contig}}
    \end{tabular}
  \end{center}
  \caption{Filter specification.}
  \label{fig:filter-spec}
\end{figure}


\newsavebox{\cml}
\begin{lrbox}{\cml}
\begin{lstlisting}[style=myML]
fun filter_step () =
 let val () = Utils.clear_buf buffer
     val () = API.callFFI "get_input" "" buffer
 in
    if WELL_FORMED_AUTOMATION_RESPONSE buffer
    then
      API.callFFI "put_output" buffer Utils.emptybuf
    else print"Filter rejects message.\n"
end
\end{lstlisting}
\end{lrbox}

\begin{figure}
  \begin{center}
    \begin{tabular}{c}
      \scalebox{0.60}{\usebox{\cml}}
    \end{tabular}
  \end{center}
  \caption{Synthesized CakeML for the filter.}
  \label{fig:filter-cakeml}
\end{figure}

The contiguity type specification and wellformedness predicate for
the filter are shown in \figref{fig:filter-spec} and the synthesized
CakeML code is in \figref{fig:filter-cakeml}. The code is called at
dispatch by the scheduler. The \texttt{API.callFFI} is the link to the
communication fabric to capture input and provide output. The body of
the function restates the filter contract to make the appropriate
assignments in a way that matches the truth value of the predicate in
the filter guarantee.  The auto-generated AGREE specification raises
an alert output when the relation is violated.

\subsection{Monitor Generation}

Monitors are intended to track and analyze the externally visible
behavior of system components through time. Therefore, they tend to
require more extensive computational ability than filters. In
particular, our basic notion of a monitor is that it embodies a
relation over its input and output streams, and is able to access the
value of a stream at any earlier point in time, if necessary. Monitors
commonly use state to keep track of earlier values, unlike filters
which, for us, are typically stateless components. (However, there is
nothing in our approach that forbids stateful filters: they can be
realized by monitors.) A monitor specification is mapped by code
generation to a state transformation function of the following
abstract type:
\[
\konst{stepFn} : \mathit{input} \times \mathit{stateVars} \to \mathit{stateVars} \times \mathit{output}
\]

The system scheduler \emph{activates} components in some order. It is
an obligation on the system that the scheduler follows some sensible
partial order of component activation and allows each component
sufficient time for its computation.  Activating a monitor component
takes the form of the following pseudo-code, in which the monitor
evaluates the \konst{stepFn} on its current inputs and the current
values of the state variables, returning the new state and the output
values.
\[
\begin{array}{ll}
 \mathit{(i_1,\ldots)} & = \konst{readInputs}(); \\
 (v_1,\ldots) & = \konst{readState}() ; \\
 ({v_1}',\ldots), ({o_1}',\ldots) & = \konst{stepFn} ((i_1,\ldots),(v_1,\ldots)) ; \\
 \multicolumn{2}{l}{\konst{writeState}({v_1}',\ldots);} \\
 \multicolumn{2}{l}{\konst{writeOutputs}({o_1}'\ldots);} \\
\end{array}
\]

\subsubsection{Initialization}

A monitor may need to accumulate a certain minimum number of
observations before being able to make a meaningful assessment of
behavior. Until that threshold is attained, the monitor is essentially
in its \emph{initialization} phase. In order for correct code to
be generated, monitor specifications need to spell out the values of
output ports when in their initialization phases. For example, suppose a
monitor does some kind of differential assessment of inputs at
adjacent time slices, alerting when (say) the measured location of a
UAV at times $t$ and $t+1$ is such that the distance between the two
locations is unusually large. Such a monitor needs two measurements
before making its first judgement, but at the time of its first
output, only one measurement will have been made. The specification
must then explicitly state the correct value for the first output.

\subsubsection{Step function}

The \konst{stepFn} works as follows:

\begin{enumerate}

\item Each input is parsed into data of the type specified by the port
  type;

\item New values for the state variables are computed, in dependency
  order. The discussion above on initialization now comes into
  play. Suppose the variable declarations have the following form:
\[
\begin{array}{l}
  v_1 = i_1 \longrightarrow e_1 \\
  \cdots \\
  v_n = i_n \longrightarrow e_n \\
\end{array}
\]
In the generated code, for the first invocation of \konst{stepFn} only,
the initializations are executed in order:
\[
\begin{array}{l}
  v_1 = i_1; \\
  \cdots \\
  v_n = i_n; \\
\end{array}
\]
In all subsequent steps, the \emph{non-initialization} assignments are performed:
\[
\begin{array}{l}
  v_1 = e_1; \\
  \cdots \\
  v_n = e_n; \\
\end{array}
\]

\item Values of the outputs are computed;

\item Outputs are written and the new state is written;

\item The monitor yields control.
\end{enumerate}

The \konst{stepFn} for the monitor of the example described in
Section~\ref{sec:example} is displayed in \figref{fig:monitor-cakeml}.

\newsavebox{\monFn}
\begin{lrbox}{\monFn}
\begin{lstlisting}[style=myML]
stepFn (Request,Response)
       (req,rsp,current,previous,policy,alert) =
let val stateVars' =
     if !initStep then
        let val req = event(Request)
            val rsp = event(Response)
            val current = (req = rsp)
            val previous = req and not(rsp)
            val policy = current or previous
            val alert = not policy
            val () = (intStep := False)
        in (req,rsp,current,previous,policy,alert)
        end
     else
        let val req = event(Request)
            val rsp = event(Response)
            val current = (req = rsp)
            val previous = pre(req and not rsp) and (not req and rsp)
            val policy = current or previous
            val alert = (is_latched and pre(alert)) or not(policy)
        in (req,rsp,current,previous,policy,alert)
        end
    val (_,rsp',_,_,_,alert') = stateVars'
    val Alert = if alert' then Some () else None
    val Output =
       if alert' then None else
       if rsp'   then Some Response
       else None
in
   (stateVars', (Alert,Output))
end
\end{lstlisting}
\end{lrbox}

\begin{figure}
  \begin{center}
    \begin{tabular}{c}
      \scalebox{0.60}{\usebox{\monFn}}
    \end{tabular}
  \end{center}
  \caption{Synthesized CakeML for the monitor.}
  \label{fig:monitor-cakeml}
\end{figure}

%% \subsection{Component Behavior}

%% Intuitively, for monitor specification $s$, \konst{stepFn} is the
%% concrete embodiment of $\konst{SynthEval}\;s$, as defined in Section
%% \ref{agree-semantics}. Its correctness amounts to showing that, given
%% a sequence of inputs, and an initial state meeting the initialization
%% constraints, iterating \konst{stepFn} produces a $\pi$ s.t. $\pi \in
%% \Lang{s}$; and taking the union over all input sequences and
%% initial states produces $\Lang{s}$ itself.


\ifREVISIONS
\subsection{Revisions}
\begin{compactitem}
  \item State theorem relating the step function to the meaning of the leaf-node semantics
\end{compactitem}
\fi

\section{Case studies}
\label{sec:case-study}
In this section, we apply the BriefCASE tool to the development of a UAV surveillance system.  The system includes  UAV receives commands from a ground station to conduct
surveillance along a geographical feature such as a river. The on-board mission computer then generates a flight plan consisting ofa series of waypoints that the UAV must traverse to complete it smission. The UAV is also given a set of keep-in and keep-out zones that may constrain its flight path.

The initial software architecture

\ifREVISIONS
\subsection{Revisions}
\begin{compactitem}
  \item Possible Outline:
  \begin{compactitem}
    \item Two case studies: phase 2 is the proof on concept on a industrial size problem while phase 3 tests if AGREE is expressive enough for stating complex properties and if designers are able to use it.
    \item Phase 2 write up focusses on the number of components and the relative ease for AGREE to prove out and for SPLAT to synthesize. It notes that the components were relatively simple.
    \item Phase 3 write up focusses on the shear complexity of the components and in particular the gate and the transport monitor. Here it the need for testing is is most obvious. Also obvious is the need to make AGREE more expressive and to separate AGREE into a specification and implementation language as seen in Dafny.
    \item Make the general observation that the transforms on the AADL model to add the components and connections doesn't really save much time when compared against the time it takes to write and test the specifications. The automations is nice, but not where time is saved. Where time is saved is in the assurance case because once a specification is done, it is synthesized and proved correct for ALL inputs and ALL outputs. Not just some. Not code inspections. Not additional test requirements. All that is saved.
  \end{compactitem}
  \item Clarify research questions answered by the case study
  \item Make the figures bigger
  \item Rewrite to focus on synthesized components and the AGREE specifications
  \item Add an interesting monitor from the TA6 model
  \item Discuss what efforts are optimized with the automations versus a fully manual process for creating the components--how does the automated transformation affect actual time for engineers
\end{compactitem}
\fi

\section{Related work}
\label{sec:related-work}
Assume-guarantee reasoning for compositional verification in reactive systems is well-studied \cite{10.1007/978-3-642-28891-3_13, composition1, 10.1145/2658982.2527272, 10.1007/978-3-319-17524-9_7}. Automated proofs of realizability for assume-guarantee reasoning are useful for engineers implementing components in the system \cite{10.1007/978-3-319-17524-9_13, 10.1007/978-3-319-29613-5_7}. Algorithms for actual component synthesis for Lustre models using k-induction or IC3/PDR provide an automated path from the assume-guarantee reasoning to an actual satisfying node implementation \cite{katis2017synthesis, 10.1007/978-3-319-89963-3_10}. These synthesis algorithms generate code in the Lustre modeling language but do not provide a path to a low-level implementation that could be fielded.


\ifREVISIONS
\subsection{Revisions}
\begin{compactitem}
  \item Add recent work published after 2018
\end{compactitem}
\fi

\section{Conclusion}
\label{sec:conclusion}
The DARPA CASE program is creating tools for systems engineers to integrate cyber-vulnerability analysis and mitigation. The resulting BriefCASE tool suite includes analysis tools for generating cyber requirements, cyber resiliency tools for addressing the requirements, verification tools for ensuring design correctness, and synthesis tools for generating provably correct code. Several of the BriefCASE transforms (filter, monitor, gate) insert components into the model whose behavior can be formally specified in the AGREE language.
The SPLAT tool can then automatically generate CakeML implementations for these components, along with proofs of correctness for assurance that the implementation satisfies the specification.

BriefCASE was applied to a full-scale case study using the Air Force Research Laboratory's OpenUxAS software, exercising a range of built-in cyber resiliency mitigations to meet cyber-requirements. 
The size and scale of the study suggests BriefCASE meets the complexity demands of real-world design.

We are currently in the process of applying BriefCASE to the design of an application using the Collins Common Avionics Architecture System (CAAS)~\cite{caas} on the CH-47.  Other ongoing work includes adding support for uninterpreted functions, mechanizing the synthesis proof in HOL4 and lifting the proof results to infinite streams.

\clearpage
\bibliographystyle{plain}
\bibliography{paper}

% \appendix
% \section{Contiguity Types}
% The formal specification of a component, and the synthesis of that specification, relies on \emph{contiguity types} (cite contiguity). A contiguity type is a self-describing specification for messages. Its formalism has basis in formal languages. Similar to how a regular expression implies a set of words that form its language, so does a contiguity type specification imply a set of messages for its language where a message is a finite sequence of contiguous bytes (e.g., a string). 

What makes contiguity type specification more expressive than regular expressions is that it is self-describing meaning that the contents of the message itself may determine the rest of the message. An example is the \texttt{AutomationResponse} from the system in the previous section with its contiguity type specification.
{\small
\begin{verbatim}
  {TaskID : i64
   Length : u8
   Waypoints : Waypoint[Length]
  }
\end{verbatim}
}
\noindent The \texttt{Waypoints} array size depends on the value of \texttt{Length} so the actual number of bytes in the message depends on the contents of the message itself. 

The type specifications may also carry meta-information about the contents of the message.
{\small
\begin{verbatim}
  {Latitude : float
   lt-rng : Assert (-90 <= Latitude <= 90) 
   Longitude : float
   lng-rng : Assert (-180 <= Longitute <= 180)
   Altitude : float
   a-rng : Assert (10000 <= Altitude <= 15000)
  }
\end{verbatim}
}
\noindent Here the specification encodes the allowed ranges for each field of the waypoint. These can be checked while constructing a message from a sequence of bytes.

Every contiguity type specification has a corresponding \emph{matcher} that when given a message string returns true or false if that message belongs to the language of the specification. If the message does belong to the language, an \emph{environment} is provided to access each part of the message. An environment, $\theta: \lval \mapsto \konst{string}$ binds \emph{L-values} to strings. An L-value is an expression that can appear on the left hand side of an assignment. The syntax for $\lval$ is given in the next section.

The matcher itself is synthesized from the specification to CakeML. The synthesis includes a corresponding proof that the matcher recognizes the language of the specification, and the resulting environment, $\theta$, from a matched message produces the same message as the one matched when serialized. 

The inputs and outputs for a high-assurance component are defined by contiguity type specifications. The synthesis from the component specification relies on the corresponding matchers from the contiguity type specifications. Details of which are in the next section.


% \section{System Model and Semantics}
% \begin{figure}
  \[
    \begin{array}{rcl}
      \mathit{c}    & = & \konst{input}\ [(f : \tau)\ldots] \\
                    &   & \konst{output}\ [(f : \tau)\ldots] \\
                    &   & \konst{components}\ [(f : c)\ldots]\\ 
                    &   & \konst{eq}\ [(\lval : \tau := exp) \dots] \\
                    &   & \konst{assume}\ [bexp\ldots] \\
                    &   & \konst{guarantee}\ [bexp\ldots] \\ \\

      \mathit{lval} & = & f \mid \mathit{lval} \, [ \mathit{exp} ]
                          \mid \mathit{lval} . f \\ \\

      f             & = & \mathit{varName} \\ \\

      \mathit{exp}  & = & \konst{Loc}\; \mathit{lval}
                          \mid \konst{nLit}\; \konst{nat}
                          \mid \mathit{constname} \\
                    & | & \mathit{exp} + \mathit{exp}
                          \mid \mathit{exp} * \mathit{exp} \\
                    & | & (exp\ \rightarrow\ exp) \\
                    & | & (\konst{pre}\ exp) \\
                    & | & (\konst{ite}\ bexp\ exp\ exp)\\
                    & | & bexp \\ \\
                          
      \mathit{bexp} & = & \konst{bLoc}\; \mathit{lval}
                          \mid  \konst{bLit}\; \konst{bool}
                          \mid  \neg \mathit{bexp}
                          \mid  \mathit{bexp} \land \mathit{bexp} \\
                    & | & \mathit{exp} = \mathit{exp} 
                    \mid  \mathit{exp} < \mathit{exp}
\end{array}
\]
\caption{Syntax for component specifications.}
\label{fig:syntax}
\end{figure}

A \emph{system} in this model of computation is a \emph{component} that is defined by a specification (see \figref{fig:syntax}). A specification defines inputs, outputs, sub-components, local variables ($\konst{eq}$), assumptions for pre-conditions, and guarantees for post-conditions. A type $\tau$ is a contiguity type which is a self-describing dependent type specification (add citation). An $\mathit{lval}$ is a reference to an \emph{L-value} from compiler concepts and is an expression that can appear on the right hand side of an assignment. Alpha renaming is assumed so that every \emph{lval} is unique.

An environment, $\theta: \mathit{lval} \mapsto \konst{string}$ binds L-values to strings. $\Delta : \konst{string} \to \mathbb{N}$ binds constant names to numbers. Functions $\konst{toN}:\konst{string}\to\mathbb{N}$ and $\konst{toB}:\konst{string}\to\konst{bool}$ interpret byte sequences to numbers and booleans, respectively. 

The semantics are synchronous data-flow defined over a sequence of environments where $\theta^i$ is the $i^\mathrm{th}$ environment in the stream. Expression evaluation is defined in the context of the environment stream in \figref{fig:eval}.
\begin{figure*}
\[
\begin{array}{l}
\konst{eval}\; i\; e =
\mathtt{case}\; e\
 \left\{
 \begin{array}{lcl}
    \konst{Loc}\; \lval & \Rightarrow & \konst{toN}(\theta^i(\lval)) \\
    \konst{nLit}\; n & \Rightarrow & n  \\
    \mathit{constname} & \Rightarrow & \Delta(\mathit{constname})  \\
    e_1 + e_2 & \Rightarrow & \konst{eval}\; i \; e_1 + \konst{eval}\; i \; e_2  \\
    e_1 * e_2 & \Rightarrow & \konst{eval}\; i \; e_1 * \konst{eval}\; i \; e_2  \\
    e_1 \rightarrow e_2 & \Rightarrow &  \mathbf{if}\; i = 0\; \mathbf{then}\; \konst{eval}\; i \; e_1\; 
                                         \mathbf{else}\; \konst{eval}\; i \; e_2 \\
    (\konst{pre}\; e) & \Rightarrow &  \konst{eval}\; i-1 \; e
  \end{array}
 \right.
 \\ \\
\konst{evalB}\; i \; b =
\mathtt{case}\; b\
 \left\{
 \begin{array}{lcl}
    \konst{bLoc}\; \lval & \Rightarrow & \konst{toB}(\theta^i(\lval)) \\
    \konst{bLit}\; b & \Rightarrow & b \\
    \neg b & \Rightarrow & \neg(\konst{evalB} \; b)  \\
    b_1 \lor b_2 & \Rightarrow & \konst{evalB}\; i \;b_1 \lor \konst{evalB}\; i \;b_2   \\
    b_1 \land b_2 & \Rightarrow & \konst{evalB}\; i \;b_1 \land \konst{evalB}\; i \;b_2   \\
    e_1 = e_2 & \Rightarrow & \konst{eval} \;e_1 = \konst{eval}\; i \;e_2   \\
    e_1 < e_2 & \Rightarrow & \konst{eval} \;e_1 < \konst{eval}\; i \;e_2
  \end{array}
 \right.
\end{array}
\]
\caption{Expression evaluation in the context of a stream on environments.}
\label{fig:eval}
\end{figure*}

The initial environment stream only contains mappings for primary inputs along the entire stream. Stepping the component updates the current environment. In other words, at the $i^\mathrm{th}$ step, $\theta^i$ is updated with the result of each component evaluation. Stepping a component is recursively defined similar to the eval functions but track the path for the environment binding.  

\end{document}
