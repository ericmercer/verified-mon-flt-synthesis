\documentclass[conference]{IEEEtran}
\IEEEoverridecommandlockouts
\usepackage{cite}
\usepackage{amsmath}
\usepackage{amsthm}
\usepackage{amsfonts}
\usepackage{url}
\usepackage{comment}
\usepackage{paralist}
\usepackage{graphicx}
\graphicspath{{./figs/}} 
\usepackage{listings}
\usepackage[usenames,dvipsnames,svgnames,table]{xcolor}

\definecolor{dkgreen}{rgb}{0,0.6,0}
\definecolor{mauve}{rgb}{0.58,0,0.82}
\definecolor{light-gray}{gray}{0.88}

\lstdefinestyle{myML}
{frame=none,
  basicstyle=\ttfamily,
  language=ML,
  aboveskip=3mm,
  belowskip=3mm,
  showstringspaces=false,
  columns=flexible,
  numbers=none,
  numberstyle=\tiny\color{gray},
  commentstyle=\color{dkgreen},
  stringstyle=\color{mauve},
  breaklines=false,
  breakatwhitespace=true,
  tabsize=2,
  linewidth=2\linewidth
}

\lstdefinestyle{agree}
{frame=none,
  basicstyle=\ttfamily,
  language=ML,
  aboveskip=3mm,
  belowskip=3mm,
  showstringspaces=false,
  columns=flexible,
  numbers=none,
  numberstyle=\tiny\color{gray},
  commentstyle=\color{dkgreen},
  stringstyle=\color{mauve},
  breaklines=false,
  breakatwhitespace=true,
  tabsize=2,
  linewidth=2\linewidth,
  morekeywords={eq, bool, guarantee, assume, true, false, pre, not, and, or, property, const}
}

\hyphenation{op-tical net-works semi-conduc-tor}

\newcommand{\konst}[1]{\ensuremath{\mathsf{#1}}}
\newcommand{\imp}{\Rightarrow}
\newcommand{\lval}{\ensuremath{\mathit{lval}}}
\newcommand{\set}[1]{\ensuremath{\{ {#1} \}}}
\newcommand{\kstar}[1]{\ensuremath{{#1}^{*}}}
\newcommand{\Lang}[1]{\ensuremath{{\mathcal L}({#1})}}
\newcommand{\LangTheta}[1]{\ensuremath{{\mathcal L}_{\theta}({#1})}}
\newcommand{\itelse}[3]{\mbox{$\mathtt{if}\ {#1}\ \mathtt{then}\ {#2}\ \mathtt{else}\ {#3}$}}

\newcommand{\figref}[1]{Fig.~\ref{#1}}

% Latex trickery for infix div operator, from stackexchange

\makeatletter
\newcommand*{\bdiv}{%
  \nonscript\mskip-\medmuskip\mkern5mu%
  \mathbin{\operator@font div}\penalty900\mkern5mu%
  \nonscript\mskip-\medmuskip
}
\makeatother

\newcommand{\ie}{\textit{i.e.}}
\newcommand{\eg}{\textit{e.g.}}
\newcommand{\etal}{\textit{et al.}}
\newcommand{\etc}{\textit{etc.}}
\newcommand{\adhoc}{\textit{ad hoc}}

\theoremstyle{plain}
\newtheorem{theorem}{Theorem}
\newtheorem{lemma}{Lemma}

\theoremstyle{definition}
\newtheorem{definition}{Definition}

\theoremstyle{remark}
\newtheorem{remark}{Remark}

\begin{document}
\title{
  Synthesizing Verified Components for Cyber Assured Systems Engineering 
  \thanks{{DISTRIBUTION STATEMENT A.  Approved for public release.}
  }
}

% author names and affiliations
% use a multiple column layout for up to three different
% affiliations
\author{
\IEEEauthorblockN{Eric Mercer}
\IEEEauthorblockA{Department of Computer Science\\
Brigham Young University\\
Provo, Utah}
\and
\IEEEauthorblockN{
Isaac Amundson and
Konrad Slind}
\IEEEauthorblockA{
Trusted Systems Group   \\
Collins Aerospace       \\
Minneapolis, Minnesota}
\and
\IEEEauthorblockN{
Junaid Babar and 
David Hardin}
\IEEEauthorblockA{
Trusted Systems Group   \\
Collins Aerospace       \\
Cedar Rapids, Iowa}
}
% conference papers do not typically use \thanks and this command
% is locked out in conference mode. If really needed, such as for
% the acknowledgment of grants, issue a \IEEEoverridecommandlockouts
% after \documentclass

\maketitle

\begin{abstract}
Cyber-physical systems, such as avionics, must be tolerant to
cyber-attacks in the same way they are tolerant to random faults: they
either gracefully recover or safely shut down as requirements dictate.
The DARPA Cyber Assured Systems Engineering program is developing
tools for design, analysis, and verification that enable systems
engineers to design-in cyber-resiliency in a Model-Based Systems
Engineering environment.  This paper describes automated model
transformations that introduce high-assurance cyber-resiliency
components into a system, in particular filters and monitors that
prevent malicious input and detect supply chain attacks, respectively.
A formal specification in the form of a code-contract defines each high-assurance component and is
used to verify that the component addresses system level cyber
requirements.  Implementations for these high-assurance components are
directly synthesized from their code-contracts, and are automatically
proven to preserve the exact meaning of the code-contract all the way
down to the binary code level.  The model transformations are
integrated into the Open Source AADL Tool Environment (OSATE).  The
paper further reports on two case studies, one by formal method experts 
and the other by non-experts,
applying cyber-resiliency model
transformations to two different systems. 
One is a UAV system that uses the Air Force Research
Laboratory's OpenUxAS services for route planning.
The other is a tablet used for inflight communication between pilots in an Apache helicopter.
In each case study, the model transformations add filters, to guard against malformed
input, as well as monitors, to guard against spoofing and other malicious behavior.
The case studies show the MBSE is feasible in real-world systems engineering.
\end{abstract}

\IEEEpeerreviewmaketitle

\section{Introduction}
In recent years, aerospace stakeholders have become aware that avionics systems are subject to possible cyber-attacks just like other cyber-physical systems.  In addition to being fault-tolerant, safety-critical avionics systems must also be {\em cyber-resilient}. Cyber-resiliency means that the system is tolerant to cyberattacks just as safety-critical systems are tolerant to random faults: they recover and continue to execute their mission function, or safely shut down, as requirements dictate. 

Unfortunately, systems engineers are currently given few development tools to help answer even basic questions about potential vulnerabilities and mitigations, and instead rely on process-oriented checklists and guidelines.  Cyber vulnerabilities are often discovered during penetration testing late in the development process.  Worse yet, they may be discovered after the product has been fielded, necessitating extremely expensive and time-consuming remediation. This is not a sustainable development model.

The DARPA Cyber Assured Systems Engineering (CASE) program is targeted to developing design, analysis, and verification tools that enable systems engineers to {\em design-in} cyber-resiliency for complex cyber-physical systems.\footnote{The
  views expressed are those of the authors and do not reflect the
  official policy or position of the Defense Advanced
  Research Projects Agency (DARPA) or the U.S. Government.}
The result is a Model-Based Systems Engineering (MBSE) environment called {\em BriefCASE} which is based on the Architecture Analysis and Design Language (AADL)~\cite{aadl}.  BriefCASE extends the Open Source AADL Tool Environment (OSATE) to add new design, analysis, and code generation capabilities targeted at building cyber-resilient systems.  

BriefCASE provides access to two analysis tools (GearCASE~\cite{gearcase2020} and DCRYPPS~\cite{dcrypps2019}) that can examine AADL models to detect potential cyber vulnerabilities and suggest requirements for mitigation.  
A library of architectural transforms guides systems engineers through automated model transformations that modify the architecture to address these requirements, possibly inserting new high-assurance components into the system. 
Implementations for these new high-assurance components are synthesized from formal specifications using the Semantic Properties for Language and Automata Theory (SPLAT) tool~\cite{slind-hcss2020},~\cite{formal-filter-synth-langsec}.
Formal verification that the transformed system model satisfies its cyber requirements is accomplished via model checking using the Assume Guarantee Reasoning Environment (AGREE)~\cite{agree2013}.  Cyber-resilient code implementing the verified model is automatically generated using the High Assurance Modeling and Rapid Engineering for Embedded Systems (HAMR) toolkit~\cite{hamr}.  If desired, this code can be targeted to the formally verified seL4 secure microkernel~\cite{sel4-2009}.

A novel aspect of the approach is the ability to automatically transformation the model to insert high-assurance components to harden the system against potential cyber-vulnerabilities. The behavior of the model is given by contracts on components expressed in the AGREE modeling language. These contracts are modified as a result of the cyber-vulnerability analysis to assume properties on the input and and guarantee properties of the output that reflect a hardened system. In the absence of any transformation to implement the hardening, AGREE fails to prove the system cyber-hardened.

The two automatic transforms discussed in this paper are the insertion of filters to prevent malformed data from malicious actors from being propagated to downstream components, and monitors to detect and alert unexpected behaviors arising from untrusted components. These transformations not only change the architecture of the model by adding in new components, but they generate a complete formal specification for the behavior of the inserted high-assurance components in the AGREE language. Those specifications are sufficient for model checking to prove that with the high-assurance components, hardened system meets its cyber-resiliency requirements.

The other novel aspect of the approach is the synthesis of the AGREE specifications for the high-assurance components to CakeML. This paper agues that synthesis to CakeML to be correct. CakeML then provides a verified compilation path to several different target binaries proving that the meaning of the CakeML is exactly preserved in the final binaries. Assuming the deployed cyber-hardened system is scheduled as intended by the AADL model, and the HAMR generated communication fabric delivers messages between components as expected, then the AGREE model checking results hold for the deployed system in that in detects and prevents the indicated cyber-vulnerabilities over all possible finite inputs. Future work is lifting this result to infinite input traces as these systems are inherently reactive and intended to run forever.

The approach is motivated, and illustrated, in a simple example. A full case study applying these transformations to a UAV system that uses the Air Force Research Laboratory's OpenUxAS services for route planning is briefly reported. Here the transforms add filters to guard against malformed input and monitors to guard against ground station spoofing and malicious flight plans from OpenUxAS. The case study system is significantly more complex than the example and shows the viability of the modeling approach to full-scale industrial design.

\begin{comment}
Here is the basic process:
\begin{enumerate}
  \item CakeML provides a mechanism to take a defined logic in HOL4 and turn it into CakeML in a way that exactly preserves the meaning of the logic.
  \item Verified components are created by defining the meaning of the components in HOL4, and then using the existing theories to turn that meaning into CakeML
  \item Contiguity types provide a formalism for defining a language and recognizing membership in a language via a matcher
  \item Contiguity types include dependent types and the ability to define when a type is malformed
  \item Define a contract language that makes assumptions on input and guarantees properties on output
  \item Give meaning to the contract language in terms of traces (not easy to do because of the state)
  \item Show how the contract can be expressed in terms of a step function 
  \item The step-function is lifted to traces
  \item A contract defines a language over traces
  \item A step function recognizes membership in that language (prefix closed)
  \item Follow the same pattern used in contiguity types
  \item The CakeML comes for free once the HOL4 theories are shown
\end{enumerate}


Set the context for this work: define system level theorem that we intend to prove with the added components? Big picture of the overall goal of the project.

Components (complexity (and expressiveness) increases along the way):
* Filter: takes a stream of data and delivers a new stream of data where the property is enforced--reasons over an infinite stream of date. Connects to the system correctness over traces. Predicate on a single piece of data, paying no attention to previous history--no temporal awareness, deciting if it passes or not. Can be very secure and very efficient.
* Monitor: Captures a relation over time on the data. Is able to reason about temporal properties. Supports multiple inputs and is able to do arbitrary computation for complex output over the inputs.

The attestation gate is an aside to show how the tools we have are rich enough to create other complex things that blend the two: filter and monitor.

This paper addresses only the verified components. Here is what we verify:

* The resulting assembly code exactly implements the specification--preserves the meaning of the original specification in the actual assembly code.

Scheduling and Trace Relationships: we assume a stream of correct data-types and the existence of some scheduler to schedule components appropriately. In the later section we give an example of a system that uses seL4 with a pacer to schedule over different domains to provide memory isolation. Scheduling atd the relation to the trace semantics is not part of this paper.


Revised outline:

Describe the system level context with the AGREE analysis that proves system level properties of streams. A system is a collection of components and connections. We assume a finite set of defined datatypes for the connections. A well-formed system is acyclic and provides some suitable set of input streams to consume.

Every component is defined by its inputs, outputs, and a contract. The contract language is simplified AGREE: requires, guarantees, eq, and expressions that include \emph{pre}. Requires are pre-conditions assumed on the input. Guarantees are post-conditions the output must meet. A component consumes a stream of input and produces and stream of output. A well-formed component is also acyclic. 

For a component to by synthesized, its contract must be strong enough to define the meaning of every output (e.g., the value assumed by the output and under what conditions it assumes that value). 

The AGREE analysis reasons about the system with its component composition proving that all contracts are respected. Failures are counter-examples where some input requirement is violated. The analysis provides a system level guarantee. 

Component synthesis compiles the contract to imperative code that implements the contract. The claim is that the final compiled code preserves the meaning of the contract: it produces the same output stream for any input stream that conforms to its pre-condition. The synthesis is accomplished with the CakeML framework that provides the additional guarantee that the post-condition holds over an infinite stream (e.g., that it the component does not terminate and always satisfies the post-condition if the pre-condition holds---Hoare logic extended to statement that diverge).

The semantics is synchronous data-flow. Computation takes zero time as does transport. The system must be well-formed. 

Motivate with some example: watchdog example on DNS monitoring for DOS attacks? Estimates the growth rate of the input request queue over time and sends a throttle signal if the growth is too fast (looks suspicious)---any bounded response of behavior based watch-dog is fine.
\end{comment}

\section{Simple Example}
\begin{figure*}
  \begin{center}
    \begin{tabular}{c}
      \includegraphics[scale=0.4]{flowchart.png} \\
    \end{tabular}
  \end{center}
\caption{AGREE failure certificate for initial design.}
\label{fig:flowchart}
\end{figure*}

\begin{figure*}[h]
  \begin{center}
    \begin{tabular}{c}
      \includegraphics[scale=0.4]{example.png}
    \end{tabular}
  \end{center}
\caption{Initial design for an automated UAV route planning system.}
\label{fig:example}
\end{figure*}

\begin{figure}
  \begin{center}
    \begin{tabular}{c}
      \includegraphics[scale=0.4]{example-certificate.png} \\
    \end{tabular}
  \end{center}
\caption{AGREE failure certificate for initial design.}
\label{fig:example-certificate}
\end{figure}

\egm{
  Merge the AGREE overview here into this section.
  Condense it down to just the opening two paragraphes that currently exist in the overview section.
  Walk through the first part of the example up to where AGREE fails because of Cyber Requirements.
  Introduce the new figure showing its workflow.
  Finish the example walking through the new figure. 
  Add the tests to occur before final AGREE check.
  Send to SPLAT to get verified CakeML code. 
}

\figref{fig:flowchart} illustrates a simplified \brfcs\ workflow with the shaded boxes representing the contributions discussed in this manuscript: test contracts, code contracts, and SPLAT synthesis.
The unshaded boxes are the existing tools that generate evidence for the assurance case.
The rest of this section follows an illustrative example through the workflow.

The baseline AADL model is a simple software system, named SW, for route planning
and automated control of a UAV. A picture of the AADL architectural
model for SW is in \figref{fig:example}. SW is loosely based on one of
the case studies in Section~\ref{sec:case-study}.  The source for the
entire model is found at \cite{repo}.

From time to time, SW receives a message on the \texttt{AutomationRequest} input,
forwarding it to a route planner (AI). The function of AI is to 
compute the flight path (a list of waypoints) for the UAV, outputting
the resulting \emph{mission command} on \texttt{AutomationResponse}.
The waypoint manager (WM) receives the mission command from AI and
starts the UAV flying the mission by putting an \emph{event} on the
\texttt{Start} output port, continuing to issue waypoints to the UAV
flight controller on the \texttt{Waypoint} port as the UAV location
updates with messages on \texttt{AirVehicleLocation}.
An event is repeatedly generated on \texttt{Alert} if there is no
mission command from a request. The AI component is third-party
software and WM is a legacy component.

%% that cannot be modified so it critically relies on
%% assumptions about its input behavior to guarantee its intended output
%% behavior.

The expected behavior of the SW system, and the components
implementing the system, are modeled with AGREE contract
specifications.  
These contracts assume and guarantee the absence
of any malicious, or unspecified, component behavior.  More precisely,
the contract for SW assumes that its inputs are \emph{well-formed} and that
there is never more than one automation request pending at a time.
Well-formed generally refers to a syntactic restriction on
data at a port. For example, a waypoint is well-formed if it falls
within bounds for latitude, longitude, and altitude.)  The guarantees
for SW ensure that
\begin{compactitem}
\item a start coincides with a new waypoint message being output;
\item a start is within one cycle of an automation request and if not, then it persistently alerts;
\item new waypoints coincide with air vehicle location updates; and
\item all outputs are well-formed.
\end{compactitem}

The contracts for the sub-components assume their inputs are
well-formed, and they guarantee their outputs are well-formed.  The AI
contract guarantees it only responds to automation requests and always
in the same cycle.  The legacy WM contract guarantees the following if
its input assumptions hold:
\begin{compactitem}
  \item it generates a start from a response from the AI and always in the same cycle;
  \item the start always coincides with a waypoint being output; and
  \item any further waypoints coincide with air vehicle location updates.
\end{compactitem}
These initial specifications pass verification, meaning that the
contract composition of the components with the system satisfies all
component input assumptions and system output guarantees.
AGREE background and the baseline contracts are discussed in detail in \secref{sec:agree} and \secref{sec:agree-semantics}.

\subsection{Adding cyber requirements}

A cyber-vulnerability analysis identifies the potential of a
supply chain attack through the AI route planner (provided by a
third-party vendor without source code).  Based on the analysis, the designer adds a requirement that the system must guard agaist malicious AI behavior, marks the component as \emph{untrusted}, and removes all output guarantees from its contract.  The AI output is now unconstrained in the
assume-guarantee reasoning of AGREE and is able to generate any value
on its output at anytime. 

The output from the AGREE analysis using the untrusted AI
specification is shown in \figref{fig:example-certificate}.  The red
exclamation points designate component assumptions or system output
guarantees that do not hold, and each failure comes with a
corresponding counter-example.  

The first violation is that the mission command on \texttt{AutomationResponse} from
the AI to the WM is no longer guaranteed to be well-formed.  The
consequence of that failing input assumption is that the WM outputs
are no longer guaranteed; they are effectively unconstrained.  These
missing output guarantees from the WM lead to the rest of the failures
in \figref{fig:example-certificate} because the WM provides the system
level outputs.

\begin{figure*}
  \begin{center}
    \begin{tabular}{c}
      \includegraphics[scale=0.3]{hardened.png}
    \end{tabular}
  \end{center}
  \caption{Cyber-hardened design for an automated UAV route planning system}
  \label{fig:hardened}
\end{figure*}

\begin{figure}
  \begin{center}
    \begin{tabular}{c}
      \includegraphics[scale=0.38]{agree-test-output.png} \\
      (a) \\ \\
      \includegraphics[scale=0.4]{hardened-certificate.png} \\
      (b)    
    \end{tabular}
  \end{center}
  \caption{AGREE verification certificates. (a) Test contract verification results. (b) Cyber-hardened design verification results.}
  \label{fig:hardened-certificate}
\end{figure}

\emph{Adding high assurance components}

The system designer now uses \brfcs\ to cyber-harden SW by inserting
high-assurance components in the form of a filter and a monitor, as
shown in \figref{fig:hardened}.
These isolate the AI component and prevent it from violating the WM component input assumptions.
A filter enforces an invariant over
each datum in the data stream by not forwarding input to its output if the invariant does not hold.
\brfcs\ generates the initial code contract for the filter which the designer completes by writing the invariant property.
That property it is usually based on the existing assumptions made by
downstream components that consume the filter output.

A monitor captures a relation on input data over time and is thus able
to reason about the temporal, and other invariant, properties of that input.  A monitor raises
an alert if the specified properties are ever violated.  
As with the filter, \brfcs\ generates a template code contract that is completed by the designer with the desired behavior that is typically defined by the SW system and its implementing component contracts. 
The code contract in our example states that an
automation response can only be generated in conjunction with an
automation request; and further, that response must come with the
request or in the next step after the request.  
Code contracts are discussed in detail in \secref{sec:code-contracts}.

Code contracts for high assurance components can be arbitrarily complex since they can have persistent state that evolves over time.
Test contracts allow the designer to validate the behavior, and general input to output properties, of a code contract through unit testing.
For example, one of the test contracts for the monitor is that it should not alert and pass its input when the monitor sees the \texttt{AutomationReponse} one step after it sees the \texttt{AutomationRequest}.
Test contracts are discussed in detail in \secref{sec:test-contracts}.

The AGREE analysis of the cyber-hardened implementation is shown in
\figref{fig:hardened-certificate}.
\figref{fig:hardened-certificate}(a) is the test contract results and \figref{fig:hardened-certificate}(b) is the hardened system composition.
Here AGREE has generated high-level
evidence justifying the claim that the high-assurance components meet their intended purposes.
Having passed AGREE verification, the high-assurance components are ready to be
synthesized.

\subsection{Synthesizing code from code contracts}

High-assurance components are automatically synthesized by SPLAT from
the code contracts to equivalent programs in the CakeML
language.
%% Too strong at the moment!
%The synthesis toolchain generates a proof that equates the
%% meaning of the AGREE specification to the behavior of the CakeML
%% program.
 In other words, for any set of input streams that meet the
 component's contract assumptions, the output streams produced from
 the synthesized CakeML code exactly match those defined by the
 high-assurance component's guarantees.  The CakeML compiler provides
 verified compilation to binaries for several different platforms,
 meaning that the resulting binaries exactly preserve the meaning of
 the original CakeML code \cite{cakeml}.

Preserving the input to output relationship of streams between the
AGREE contracts and CakeML lifts the AGREE contract verification
results to the actual deployed system.  If the contract model
verification succeeds, then the meaning of those results hold for the
deployed system.  These results, however, are only valid under
additional assumptions on the deployed system:
\begin{compactitem}
\item contracts for non-synthesized components accurately model their deployed
counterparts;
\item an appropriate schedule exists to sequence component
  execution following the dependent data-flow in the design; and
\item the communication fabric stitching components together works in harmony
  with the schedule.
\end{compactitem}
\noindent Support and automation for these aspects of the design process are
discussed in other works \cite{gearcase2020, dcrypps2019, 10.1007/978-3-030-89159-6_18, 10.1007/978-3-030-89159-6_17, sel4-2009, scheduled-agree, 9734792}.
Synthesis from code contracts is discussed in detail in \secref{sec:sythesis}.

\section{AGREE Contract Specification}
The goal of this section is to illustrate in more detail the process a
system designer follows to add cyber requirements to the AGREE
specifications and then automatically transform the system to insert
the high-assurance components. The AGREE specifications generated by
the transforms for the high-assurance components are also
explained. The section ends with a concise formal statement of the
meaning of an AGREE specification for a high-assurance component. This
meaning is what must be preserved by the synthesis.

\newsavebox{\sw}
\begin{lrbox}{\sw}
\begin{lstlisting}[style=agree]
eq req : bool = event(AutomationRequest);
eq avl : bool = event(AirVehicleLocation);
eq wp : bool = event(Waypoint);
eq rsp: bool = event(Start);
eq alrt : bool = event(Alert);

assume "Automation request is well-formed" :
    req => WELL_FORMED_AUTOMATION_REQUEST(AutomationRequest);
assume "Air vehicle location is well-formed" :
    avl => WELL_FORMED_WAYPOINT(AirVehicleLocation);

eq current : bool = (req = rsp);
eq previous : bool = (req and not rsp) ->
                      pre(req and not rsp) and (not req and rsp);
eq policy : bool = current or previous;
eq since : bool =  alrt or (alrt and (false -> pre(since)));

guarantee "Start includes a waypoint" :
    rsp => wp;
guarantee "Locations required after the start waypoint" :
    (wp and not rsp) => avl;
guarantee "Waypoint is well-formed" :
    wp => WELL_FORMED_WAYPOINT(Waypoint);
guarantee "Alert if start is not bounded relative to a request" :
    policy or since;
\end{lstlisting}
\end{lrbox}

\begin{figure}
  \begin{center}
    \scalebox{0.60}{\usebox{\sw}}
  \end{center}
  \caption{The SW component contract.}
  \label{fig:sw}
\end{figure}

The AGREE specification for the SW component in the example of
Section~\ref{sec:example} with the added cyber requirements is given
in \figref{fig:sw}. The AGREE specification language is a first-order
predicate calculus that uses stream concepts, and operators, from the
Lustre language \cite{10.1145/41625.41641}. As with Lustre, the
semantics are synchronous dataflow where the inputs, outputs, and
expressions are data streams that comply with the input
assumptions. Contracts are evaluated in dependency order with inputs
being propagated to outputs through all contracts until they
stabilize; as such, the contracts, and thereby the top-level model,
must be acyclic.\footnote{An apparent syntactic cycle, where a
component is linked back to itself, may be broken temporally by
inserting delay elements.} Once the contracts have stabilized, the
model takes a synchronous step to the next input data in the
stream. The semantics do not model computation or communication
delay. The output of one contract is seen at the input of any
downstream contract in the same step of the input data stream.

The AGREE model checker attempts to prove several properties of the
top-level model being verified. The first is that the output
guarantees of each component implementing the system are strong enough
to validate the input assumptions of any downstream component as well
as to satisfy the guarantees of the output of the top-level component
being verified (\ie, the system composition meets input assumptions
at each input as well as the guarantees on the system output). These
properties are reported in the expanded lists
in \figref{fig:example}(b) and \figref{fig:hardened}(b).  The next set
of properties prove that the contract specifications for each
component are \emph{self-consistent} \ie, a contract does not contradict
itself. These are the unexpanded results at the bottom of the
figures.

Returning back to the contract of \figref{fig:sw}, it uses \texttt{eq}
statements to define variables local to the contract
specification. For example, the \texttt{req} variable is equivalent to
the \texttt{event(AutomationRequest)} expression. In the AGREE
semantics, there is an implicit \emph{event} input (or output)
associated with every named event port in a component. The semantics
used here do not buffer these events so the implicit input (or output)
is only a boolean value. An \texttt{event} expression refers to that
implicit input (or output) and is true when data is placed on the
named port. The system contract here states assumptions on well-formed
input, followed by guarantees on properties about the output.

The \emph{Alert if start is not bounded relative to a request}
guarantee is an invariant on the expression \texttt{policy or since},
meaning that either the policy holds or the alert is
sounding. The \texttt{policy} is defined by two local
values: \texttt{current} and \texttt{previous}. The \texttt{current}
value is asserted when in the current time step there is a request
with a response, or there is no request and no response.

The value of \texttt{previous} in the current time step relies on values from the previous time step. The \texttt{->} operator designates initialization, as the previous time step is undefined in the first step of the system. The left operand to the operator is the initial value of \texttt{previous} at start, which in this example is \texttt{(req and not rsp)}, because seeing a request with no response is inconclusive in the first step of the system. The right operand is the value of \texttt{previous} after the initial step. Here the \texttt{pre} operator refers to the value of the expression \texttt{(req and not rsp)} in the prior time step, \texttt{previous} is true if the previous time step made a request without a matching response and the current time step has the matching response to that request with no new request.

The value of \texttt{since} in \emph{Alert if start is not bounded relative to a request} relies on its own value in the previous time step. The intuitive reading of the expression is that the alert has been true since the time when it first sounded. The first \texttt{alrt} sets \texttt{since} to true, and once the value of \texttt{since} is true, that value persists as long as \texttt{alrt} holds. The \emph{Alert if start is not bounded relative to a request} guarantee defines one requirement of a cyber-hardened system implementation. Together with the other guarantees, the contract models the expected input and output of the system as a whole.

\begin{figure}
  \begin{center}
    \begin{tabular}{c}
      \includegraphics[scale=0.3]{dialogue.png}
    \end{tabular}
  \end{center}
  \caption{Wizard for automatically transforming the model with a filter.}
  \label{fig:dialogue}
\end{figure}

\newsavebox{\flt}
\begin{lrbox}{\flt}
\begin{lstlisting}[style=agree]
eq policy : bool =
  WELL_FORMED_AUTOMATION_RESPONSE(Input);
guarantee Filter_Output "Filter output is well-formed" :
  if event(Input) and policy then
    event(Output) and Output = Input
  else not event(Output);
\end{lstlisting}
\end{lrbox}

\newsavebox{\mntr}
\begin{lrbox}{\mntr}
\begin{lstlisting}[style=agree]
const is_latched : bool =
  Get_Property(this, CASE_Properties::Monitor_Latched);
eq rsp : bool = event(Response);
eq req : bool = event(Request);
eq current : bool = (req = rsp);
eq previous : bool = (req and not rsp) ->
                      pre(req and not rsp) and (not req and rsp);
eq policy : bool = current or previous;
eq alert : bool = (not policy)
                -> ((is_latched and pre(alert)) or not policy);
guarantee Monitor_Alert
  "Alert port tracks alert variable" :
  event(Alert) = alert;
guarantee Monitor_Output
  "Output if not alerted" :
  if event(Alert) then (not event(Output)) else
  if event(Response) then (event(Output) and (Output = Response))
  else (not event(Output));
\end{lstlisting}
\end{lrbox}

\begin{figure}
  \begin{center}
    \begin{tabular}{c}
      \scalebox{0.60}{\usebox{\flt}} \\
      (a) \\
      \scalebox{0.60}{\usebox{\mntr}} \\
      (b)
    \end{tabular}
  \end{center}
  \caption{High-assurance component contracts. (a) The filter. (b) The monitor.}
  \label{fig:assurance}
\end{figure}

As noted previously, the original system fails to guarantee the cyber requirements. BriefCASE provides two transformations to address the failing requirements: inserting a filter and inserting a monitor. The component is added by selecting the connection in the model where the high-assurance component is to be added, and then choosing the appropriate transformation. The system designer can provide transform configuration parameters in a wizard, as shown in \figref{fig:dialogue}. The policy of the high-assurance component can be stated directly in the wizard, or it can be left  blank. In this example, the policy is specified as \texttt{WELL\_FORMED\_AUTOMATION\_RESPONSE(Input)}.
Additionally, because a transformation is ultimately driven by a cyber requirement, BriefCASE updates an embedded Resolute assurance case~\cite{resolute-destion}.  Resolute keeps track of the evidential artifacts necessary for supporting the requirement, and can be run at any time to determine whether those artifacts are valid.
%Additionally, the filter can be attached to any requirement the vulnerability analysis. This association is useful the the assurance case from the Resolute tool [CITE DESTION 2021]. The \emph{OK} button creates the component and inserts it accordingly into the implementation

The AGREE contract specification generated by the transform is shown in \figref{fig:assurance}(a). The guarantee is stylized for synthesis and completely defines the meaning of the output under every possible input.
%A similar dialogue exist for inserting monitors that allows the system designer to add and remove inputs as needed.
The resulting AGREE specification for the monitor in this example is shown in \figref{fig:assurance}(b). The \texttt{is\_latched} value makes the alert persistent, meaning that once the alert is raised, it is always raised.  This behavior is one of the several options available in the dialogue. The definition for \texttt{policy} is taken by the system developer from the contract in \figref{fig:sw}. As before, the guarantees for the outputs are autogenerated by the tool and completely define each output under every possible input.

\subsection{AGREE Semantics}
\label{agree-semantics}

Here the formal semantics of the AGREE contract specification are
briefly presented to make clear the meaning of a high-assurance
component.  These semantics are the basis of proofs showing that a
synthesized component has input to output behavior specified by its
contract.

Assume that all data flowing between components is in the form of byte
arrays, thus input and output is modelled at the level of what is
passed ``over the wire'' by the communication fabric. An \emph{environment},
$\theta: \lval \mapsto \konst{string}$, binds \emph{L-values} to
strings where an L-value is anything that can appear on the left-hand
side of an assignment such as a named port, a field in a record, a
entry in an array, a local value defined by an eq-statement etc.

Let $s$ be an AGREE contract specification for some high-assurance
component. The notation $\theta_s$ is used to denote the environment that
contains a binding for any L-value in the scope of $s$ and nothing else, and
the notation $\Theta_s$ denotes the universe of all such environments. The
semantics of $s$ are defined over streams, $\pi = \theta_1, \theta_2, \ldots$, which
are finite sequences of environments, $\pi \in \Theta_s^*$.

%% \begin{comment}
%% [\emph{So basically $\theta$ covers the inputs and the state, i.e. we can
%% generate a contig type for the inputs plus state vars.} ]

%% [Question: what is a \emph{total} environment? One that supplies at
%% least the bindings in $\theta$ needed to evaluate the spec?]
%% \end{comment}

The function $\konst{eval}\; s\; \pi$ evaluates $s$ on the stream,
$\pi$, and returns \konst{true} if $s$ is invariant along the entire
stream and \konst{false} otherwise. A guarantee $\mathcal{G}$ in $s$
is \emph{invariant} if $\mathcal{G}$ is true for each prefix of $\pi$,
while an eq-statement in $s$ is invariant if its binding in the
context of every step is equivalent to the computed value of its
associated expression in that same context. All guarantees and
eq-statements must be invariant in the stream for the function to
return \konst{true}.

The meaning of an AGREE contract specification is now defined as the
set of environment streams on which it is invariant.
%
\[
  \Lang{s} = \set{\pi \in \Theta_s^* \mid \konst{eval}\ s\ \pi = \konst{true}}
\]
%
\noindent Intuitively, any stream $\pi \in \mathcal{L}(s)$, at each step, binds
the L-values in the eq-statements in a way that is consistent with
their associated expressions and the guarantees are all true in that
same step.

We claim that a synthesized high-assurance component preserves the
input to output relationship in the specification $s$ over every
stream in $\Lang{s}$.  Let $\pi^\prime = \konst{SynthEval}\;
s\; \pi$ denote a function that synthesizes $s$ and then evaluates that
synthesized component on the stream $\pi$ to create a new stream
$\pi^\prime$ containing added output and other bindings. We say that
two streams are equivalent in regards to a specification, denoted as
$\pi =_s \pi^\prime$, if and only if the two streams are the same
length and agree on bindings for the input and output for $s$ at every
step of the streams.  We now formally state the correctness claim for
synthesis.
%
\[
  \forall \pi \in \Lang{s}.\; (\konst{SynthEval}\; s\; I_s(\pi)) =_s \pi\\
\]
%
where $I_s(\pi)$ returns the corresponding stream that only retains
bindings for inputs in each step and nothing else. The claim is that
the synthesized component generates the same output stream as the
specification for any stream belonging to the specification that is
restricted to just input bindings at each step.


\section{Synthesis}
Synthesis maps from model and specifications to code. The SPLAT tool
traverses the system architecture looking for occurrences of high
assurance components specified by code contracts; for each such
occurrence it generates a CakeML program. More generally, SPLAT supports
three kinds of behavioral specification from which to generate
code. In increasing order of expressiveness, these are:

\paragraph{Regular expressions.} Often the well-formedness of data on connections
can be enforced by matching against a regular expression. In other
words, the data can be characterized by a \emph{regular language}. As
is well known, regular expressions can be translated to efficient
deterministic finite state machine (DFA)
implementations. In \cite{formal-filter-synth-langsec,case-verified-filter}
we report on the creation and application of filters from a verified
regular expression-to-DFA compiler.

\paragraph{Contiguity types\cite{contiguity-types}.}
A contiguity type, like regular expressions, is effective at
describing data at the network message level, \eg, as flat
strings. The notation is however more expressive, being able to
declare many kinds of so-called \emph{self-describing} messages, \ie,
those where message structure is dependent on data held elsewhere in
the message. Figure \ref{fig:filter-spec} gives a contiguity type
specification for the welll-formed waypoints of our example system. We
have formalized and verified a contiguity type matcher. The matcher
code has been used in our case studies.

\paragraph{AGREE code contracts.} When full computational power is needed, we can use the
expressive power of AGREE code contracts to write arbitrarily complex
filter or monitor operations, as we have already seen. Our formal
model of AGREE syntax and semantics is applied to extract code
contracts and map them to equivalent logic functions.

\noindent For each of these three approaches, we take the following steps to
arrive at an executable.

\begin{enumerate}
\item Map from specification to computable function in logic, producing
a proof of equivalence.
\item Add parsers and printers (specified by contiguity types) for input and output.
\item Map to CakeML and add calls to input/output library routines via
      the CakeML foreign function interface.
\item Invoke the CakeML compiler.
\end{enumerate}

%% Starting from the semantics of the code contract described in
%% Section \ref{sec:code-contracts} a sequence of semantics-preserving
%% steps are taken:
%% \begin{itemize}
%% \item $\sem{{\Eqs}\cdot G}$ yields a function on arbitrary depth streams
%% \item transform to Pascalish
%% \item transform to logic function
%% \item attach buffer handling to inputs and outputs (contig type parsing)
%% \item translate to CakeML
%% \item run CakeML compiler
%% \end{itemize}

In the following, we examine further details in both filter and
monitor synthesis.

\subsection{Filter Generation}

A filter is intended to be simple, although it may make deep semantic
checks. A filter has one input port and one output; messages on the
input that the filter policy admits pass unchanged to the output port;
all others are dropped (not passed on). We have investigated two kinds
of filter. In the first, a relatively shallow scan of the input
suffices to enforce the policy. For example, we have used the
expressive power of regular expressions and Contiguity
Types \cite{contiguity-types} to enforce \emph{lightweight} bounds
constraints on GPS coordinates in UxAS messages. On the other hand, a
filter may need to parse the input buffer into a data structure
specified in AGREE and apply a user-defined \emph{wellformedness}
property, also specified in AGREE, to the data. Arbitrarily complex
wellformedness checks can be made in this
way. \figref{fig:filter-spec} shows a combination where the checking
specified by {\small\verb+WELL_FORMED_AUTOMATION_RESPONSE+} depends on
an underlying check specified by the contiguity type checking bounds
on waypoints.

The verdict of a filter is made and performed within one thread
invocation. Thus, in its given time slice, the
following steps must be completed:

\begin{enumerate}

\item The filter checks to see if there is any input available.  If there is none
then it yields control; otherwise:

\item The input is read (and parsed if need be);

\item The wellformedness predicate is evaluated on the input;

\item If the predicate returns \konst{true} then the input buffer
 is copied to the output, otherwise no action is taken; and

\item The filter yields control.
\end{enumerate}

\begin{remark}[Partiality]

Partiality is an important consideration: steps 2 and 3 above can
fail; the data might not be parseable or the wellformedness
computation could be badly written and fail at runtime. In such cases,
the filter should recover and yield control without passing the input
onwards. In these cases, the filter is behaving as it should, but we
must also guard against situations in which a \emph{correctly
specified} filter fails at runtime. This kind of defect arises when
the filter \emph{ought} to accept a message, but lack of resources
results in the filter failing to do so. For example, the parse of a
message might need more space than has been allocated; another example
could be if the time slice provided by the scheduler is too short for
the wellformedness computation to finish. Thus resource bounds need to
be included in the correctness argument. Some preliminary work on this
has been done in CakeML \cite{cakeml-space-cost}.

\end{remark}


\newsavebox{\contig}
\begin{lrbox}{\contig}
\begin{lstlisting}[style=myML]
  Waypoint =
    {Latitude  : f64
     Longitude : f64
     Altitude  : f32
     Check     : Assert
      (~90.0 <= Latitude and Latitude <= 90.0 and
       ~180.0 <= Longitude and Longitude <= 180.0 and
       1000.0 <= Altitude and Altitude <= 15000.0)}

  AutomationResponse =
    {TaskID : i64
     Length : u8
     Waypoints : Waypoint [3]}

 fun WELL_FORMED_AUTOMATION_RESPONSE(aresp) =
   (forall wpt in aresp.Waypoints, WELL_FORMED_WAYPOINT(wpt))
   and ... ;
\end{lstlisting}
\end{lrbox}

\begin{figure}
  \begin{center}
    \begin{tabular}{c}
      \scalebox{0.60}{\usebox{\contig}}
    \end{tabular}
  \end{center}
  \caption{Filter specification.}
  \label{fig:filter-spec}
\end{figure}


\newsavebox{\cml}
\begin{lrbox}{\cml}
\begin{lstlisting}[style=myML]
fun filter_step () =
 let val () = Utils.clear_buf buffer
     val () = API.callFFI "get_input" "" buffer
 in
    if WELL_FORMED_AUTOMATION_RESPONSE buffer
    then
      API.callFFI "put_output" buffer Utils.emptybuf
    else print"Filter rejects message.\n"
end
\end{lstlisting}
\end{lrbox}

\begin{figure}
  \begin{center}
    \begin{tabular}{c}
      \scalebox{0.60}{\usebox{\cml}}
    \end{tabular}
  \end{center}
  \caption{Synthesized CakeML for the filter.}
  \label{fig:filter-cakeml}
\end{figure}

The contiguity type specification and wellformedness predicate for
the filter are shown in \figref{fig:filter-spec} and the synthesized
CakeML code is in \figref{fig:filter-cakeml}. The code is called at
dispatch by the scheduler. The \texttt{API.callFFI} is the link to the
communication fabric to capture input and provide output. The body of
the function restates the filter contract to make the appropriate
assignments in a way that matches the truth value of the predicate in
the filter guarantee.  The auto-generated AGREE specification raises
an alert output when the relation is violated.

\subsection{Monitor Generation}

Monitors are intended to track and analyze the externally visible
behavior of system components through time. Therefore, they tend to
require more extensive computational ability than filters. In
particular, our basic notion of a monitor is that it embodies a
relation over its input and output streams, and is able to access the
value of a stream at any earlier point in time, if necessary. Monitors
commonly use state to keep track of earlier values, unlike filters
which, for us, are typically stateless components. (However, there is
nothing in our approach that forbids stateful filters: they can be
realized by monitors.) A monitor specification is mapped by code
generation to a state transformation function of the following
abstract type:
\[
\konst{stepFn} : \mathit{input} \times \mathit{stateVars} \to \mathit{stateVars} \times \mathit{output}
\]

The system scheduler \emph{activates} components in some order. It is
an obligation on the system that the scheduler follows some sensible
partial order of component activation and allows each component
sufficient time for its computation.  Activating a monitor component
takes the form of the following pseudo-code, in which the monitor
evaluates the \konst{stepFn} on its current inputs and the current
values of the state variables, returning the new state and the output
values.
\[
\begin{array}{ll}
 \mathit{(i_1,\ldots)} & = \konst{readInputs}(); \\
 (v_1,\ldots) & = \konst{readState}() ; \\
 ({v_1}',\ldots), ({o_1}',\ldots) & = \konst{stepFn} ((i_1,\ldots),(v_1,\ldots)) ; \\
 \multicolumn{2}{l}{\konst{writeState}({v_1}',\ldots);} \\
 \multicolumn{2}{l}{\konst{writeOutputs}({o_1}'\ldots);} \\
\end{array}
\]

\subsubsection{Initialization}

A monitor may need to accumulate a certain minimum number of
observations before being able to make a meaningful assessment of
behavior. Until that threshold is attained, the monitor is essentially
in its \emph{initialization} phase. In order for correct code to
be generated, monitor specifications need to spell out the values of
output ports when in their initialization phases. For example, suppose a
monitor does some kind of differential assessment of inputs at
adjacent time slices, alerting when (say) the measured location of a
UAV at times $t$ and $t+1$ is such that the distance between the two
locations is unusually large. Such a monitor needs two measurements
before making its first judgement, but at the time of its first
output, only one measurement will have been made. The specification
must then explicitly state the correct value for the first output.

\subsubsection{Step function}

The \konst{stepFn} works as follows:

\begin{enumerate}

\item Each input is parsed into data of the type specified by the port
  type;

\item New values for the state variables are computed, in dependency
  order. The discussion above on initialization now comes into
  play. Suppose the variable declarations have the following form:
\[
\begin{array}{l}
  v_1 = i_1 \longrightarrow e_1 \\
  \cdots \\
  v_n = i_n \longrightarrow e_n \\
\end{array}
\]
In the generated code, for the first invocation of \konst{stepFn} only,
the initializations are executed in order:
\[
\begin{array}{l}
  v_1 = i_1; \\
  \cdots \\
  v_n = i_n; \\
\end{array}
\]
In all subsequent steps, the \emph{non-initialization} assignments are performed:
\[
\begin{array}{l}
  v_1 = e_1; \\
  \cdots \\
  v_n = e_n; \\
\end{array}
\]

\item Values of the outputs are computed;

\item Outputs are written and the new state is written;

\item The monitor yields control.
\end{enumerate}

The \konst{stepFn} for the monitor of the example described in
Section~\ref{sec:example} is displayed in \figref{fig:monitor-cakeml}.

\newsavebox{\monFn}
\begin{lrbox}{\monFn}
\begin{lstlisting}[style=myML]
stepFn (Request,Response)
       (req,rsp,current,previous,policy,alert) =
let val stateVars' =
     if !initStep then
        let val req = event(Request)
            val rsp = event(Response)
            val current = (req = rsp)
            val previous = req and not(rsp)
            val policy = current or previous
            val alert = not policy
            val () = (intStep := False)
        in (req,rsp,current,previous,policy,alert)
        end
     else
        let val req = event(Request)
            val rsp = event(Response)
            val current = (req = rsp)
            val previous = pre(req and not rsp) and (not req and rsp)
            val policy = current or previous
            val alert = (is_latched and pre(alert)) or not(policy)
        in (req,rsp,current,previous,policy,alert)
        end
    val (_,rsp',_,_,_,alert') = stateVars'
    val Alert = if alert' then Some () else None
    val Output =
       if alert' then None else
       if rsp'   then Some Response
       else None
in
   (stateVars', (Alert,Output))
end
\end{lstlisting}
\end{lrbox}

\begin{figure}
  \begin{center}
    \begin{tabular}{c}
      \scalebox{0.60}{\usebox{\monFn}}
    \end{tabular}
  \end{center}
  \caption{Synthesized CakeML for the monitor.}
  \label{fig:monitor-cakeml}
\end{figure}

%% \subsection{Component Behavior}

%% Intuitively, for monitor specification $s$, \konst{stepFn} is the
%% concrete embodiment of $\konst{SynthEval}\;s$, as defined in Section
%% \ref{agree-semantics}. Its correctness amounts to showing that, given
%% a sequence of inputs, and an initial state meeting the initialization
%% constraints, iterating \konst{stepFn} produces a $\pi$ s.t. $\pi \in
%% \Lang{s}$; and taking the union over all input sequences and
%% initial states produces $\Lang{s}$ itself.


\section{UxAS Case Study}
In this section, we apply the BriefCASE tool to the development of a UAV surveillance system.  The system includes  UAV receives commands from a ground station to conduct
surveillance along a geographical feature such as a river. The on-board mission computer then generates a flight plan consisting ofa series of waypoints that the UAV must traverse to complete it smission. The UAV is also given a set of keep-in and keep-out zones that may constrain its flight path.

The initial software architecture

\section{Related Work}
Assume-guarantee reasoning for compositional verification in reactive systems is well-studied \cite{10.1007/978-3-642-28891-3_13, composition1, 10.1145/2658982.2527272, 10.1007/978-3-319-17524-9_7}. Automated proofs of realizability for assume-guarantee reasoning are useful for engineers implementing components in the system \cite{10.1007/978-3-319-17524-9_13, 10.1007/978-3-319-29613-5_7}. Algorithms for actual component synthesis for Lustre models using k-induction or IC3/PDR provide an automated path from the assume-guarantee reasoning to an actual satisfying node implementation \cite{katis2017synthesis, 10.1007/978-3-319-89963-3_10}. These synthesis algorithms generate code in the Lustre modeling language but do not provide a path to a low-level implementation that could be fielded.


\section{Conclusion}
The DARPA CASE program is creating tools for systems engineers to integrate cyber-vulnerability analysis and mitigation. The resulting BriefCASE tool suite includes analysis tools for generating cyber requirements, cyber resiliency tools for addressing the requirements, verification tools for ensuring design correctness, and synthesis tools for generating provably correct code. Several of the BriefCASE transforms (filter, monitor, gate) insert components into the model whose behavior can be formally specified in the AGREE language.
The SPLAT tool can then automatically generate CakeML implementations for these components, along with proofs of correctness for assurance that the implementation satisfies the specification.

BriefCASE was applied to a full-scale case study using the Air Force Research Laboratory's OpenUxAS software, exercising a range of built-in cyber resiliency mitigations to meet cyber-requirements. 
The size and scale of the study suggests BriefCASE meets the complexity demands of real-world design.

We are currently in the process of applying BriefCASE to the design of an application using the Collins Common Avionics Architecture System (CAAS)~\cite{caas} on the CH-47.  Other ongoing work includes adding support for uninterpreted functions, mechanizing the synthesis proof in HOL4 and lifting the proof results to infinite streams.

\section*{Acknowledgments}
This work was sponsored in part by the Defense Advanced Research
Projects Agency (DARPA).

% An example of a double column floating figure using two subfigures.
% (The subfig.sty package must be loaded for this to work.)
% The subfigure \label commands are set within each subfloat command,
% and the \label for the overall figure must come after \caption.
% \hfil is used as a separator to get equal spacing.
% Watch out that the combined width of all the subfigures on a 
% line do not exceed the text width or a line break will occur.
%
%\begin{figure*}[!t]
%\centering
%\subfloat[Case I]{\includegraphics[width=2.5in]{box}%
%\label{fig_first_case}}
%\hfil
%\subfloat[Case II]{\includegraphics[width=2.5in]{box}%
%\label{fig_second_case}}
%\caption{Simulation results for the network.}
%\label{fig_sim}
%\end{figure*}
%
% Note that often IEEE papers with subfigures do not employ subfigure
% captions (using the optional argument to \subfloat[]), but instead will
% reference/describe all of them (a), (b), etc., within the main caption.
% Be aware that for subfig.sty to generate the (a), (b), etc., subfigure
% labels, the optional argument to \subfloat must be present. If a
% subcaption is not desired, just leave its contents blank,
% e.g., \subfloat[].

% trigger a \newpage just before the given reference
% number - used to balance the columns on the last page
% adjust value as needed - may need to be readjusted if
% the document is modified later
%\IEEEtriggeratref{8}
% The "triggered" command can be changed if desired:
%\IEEEtriggercmd{\enlargethispage{-5in}}

% references section
\bibliographystyle{IEEEtran}
\bibliography{paper}

\appendix
\section{Contiguity Types}
The formal specification of a component, and the synthesis of that specification, relies on \emph{contiguity types} (cite contiguity). A contiguity type is a self-describing specification for messages. Its formalism has basis in formal languages. Similar to how a regular expression implies a set of words that form its language, so does a contiguity type specification imply a set of messages for its language where a message is a finite sequence of contiguous bytes (e.g., a string). 

What makes contiguity type specification more expressive than regular expressions is that it is self-describing meaning that the contents of the message itself may determine the rest of the message. An example is the \texttt{AutomationResponse} from the system in the previous section with its contiguity type specification.
{\small
\begin{verbatim}
  {TaskID : i64
   Length : u8
   Waypoints : Waypoint[Length]
  }
\end{verbatim}
}
\noindent The \texttt{Waypoints} array size depends on the value of \texttt{Length} so the actual number of bytes in the message depends on the contents of the message itself. 

The type specifications may also carry meta-information about the contents of the message.
{\small
\begin{verbatim}
  {Latitude : float
   lt-rng : Assert (-90 <= Latitude <= 90) 
   Longitude : float
   lng-rng : Assert (-180 <= Longitute <= 180)
   Altitude : float
   a-rng : Assert (10000 <= Altitude <= 15000)
  }
\end{verbatim}
}
\noindent Here the specification encodes the allowed ranges for each field of the waypoint. These can be checked while constructing a message from a sequence of bytes.

Every contiguity type specification has a corresponding \emph{matcher} that when given a message string returns true or false if that message belongs to the language of the specification. If the message does belong to the language, an \emph{environment} is provided to access each part of the message. An environment, $\theta: \lval \mapsto \konst{string}$ binds \emph{L-values} to strings. An L-value is an expression that can appear on the left hand side of an assignment. The syntax for $\lval$ is given in the next section.

The matcher itself is synthesized from the specification to CakeML. The synthesis includes a corresponding proof that the matcher recognizes the language of the specification, and the resulting environment, $\theta$, from a matched message produces the same message as the one matched when serialized. 

The inputs and outputs for a high-assurance component are defined by contiguity type specifications. The synthesis from the component specification relies on the corresponding matchers from the contiguity type specifications. Details of which are in the next section.
 

\section{System Model and Semantics}
\begin{figure}
  \[
    \begin{array}{rcl}
      \mathit{c}    & = & \konst{input}\ [(f : \tau)\ldots] \\
                    &   & \konst{output}\ [(f : \tau)\ldots] \\
                    &   & \konst{components}\ [(f : c)\ldots]\\ 
                    &   & \konst{eq}\ [(\lval : \tau := exp) \dots] \\
                    &   & \konst{assume}\ [bexp\ldots] \\
                    &   & \konst{guarantee}\ [bexp\ldots] \\ \\

      \mathit{lval} & = & f \mid \mathit{lval} \, [ \mathit{exp} ]
                          \mid \mathit{lval} . f \\ \\

      f             & = & \mathit{varName} \\ \\

      \mathit{exp}  & = & \konst{Loc}\; \mathit{lval}
                          \mid \konst{nLit}\; \konst{nat}
                          \mid \mathit{constname} \\
                    & | & \mathit{exp} + \mathit{exp}
                          \mid \mathit{exp} * \mathit{exp} \\
                    & | & (exp\ \rightarrow\ exp) \\
                    & | & (\konst{pre}\ exp) \\
                    & | & (\konst{ite}\ bexp\ exp\ exp)\\
                    & | & bexp \\ \\
                          
      \mathit{bexp} & = & \konst{bLoc}\; \mathit{lval}
                          \mid  \konst{bLit}\; \konst{bool}
                          \mid  \neg \mathit{bexp}
                          \mid  \mathit{bexp} \land \mathit{bexp} \\
                    & | & \mathit{exp} = \mathit{exp} 
                    \mid  \mathit{exp} < \mathit{exp}
\end{array}
\]
\caption{Syntax for component specifications.}
\label{fig:syntax}
\end{figure}

A \emph{system} in this model of computation is a \emph{component} that is defined by a specification (see \figref{fig:syntax}). A specification defines inputs, outputs, sub-components, local variables ($\konst{eq}$), assumptions for pre-conditions, and guarantees for post-conditions. A type $\tau$ is a contiguity type which is a self-describing dependent type specification (add citation). An $\mathit{lval}$ is a reference to an \emph{L-value} from compiler concepts and is an expression that can appear on the right hand side of an assignment. Alpha renaming is assumed so that every \emph{lval} is unique.

An environment, $\theta: \mathit{lval} \mapsto \konst{string}$ binds L-values to strings. $\Delta : \konst{string} \to \mathbb{N}$ binds constant names to numbers. Functions $\konst{toN}:\konst{string}\to\mathbb{N}$ and $\konst{toB}:\konst{string}\to\konst{bool}$ interpret byte sequences to numbers and booleans, respectively. 

The semantics are synchronous data-flow defined over a sequence of environments where $\theta^i$ is the $i^\mathrm{th}$ environment in the stream. Expression evaluation is defined in the context of the environment stream in \figref{fig:eval}.
\begin{figure*}
\[
\begin{array}{l}
\konst{eval}\; i\; e =
\mathtt{case}\; e\
 \left\{
 \begin{array}{lcl}
    \konst{Loc}\; \lval & \Rightarrow & \konst{toN}(\theta^i(\lval)) \\
    \konst{nLit}\; n & \Rightarrow & n  \\
    \mathit{constname} & \Rightarrow & \Delta(\mathit{constname})  \\
    e_1 + e_2 & \Rightarrow & \konst{eval}\; i \; e_1 + \konst{eval}\; i \; e_2  \\
    e_1 * e_2 & \Rightarrow & \konst{eval}\; i \; e_1 * \konst{eval}\; i \; e_2  \\
    e_1 \rightarrow e_2 & \Rightarrow &  \mathbf{if}\; i = 0\; \mathbf{then}\; \konst{eval}\; i \; e_1\; 
                                         \mathbf{else}\; \konst{eval}\; i \; e_2 \\
    (\konst{pre}\; e) & \Rightarrow &  \konst{eval}\; i-1 \; e
  \end{array}
 \right.
 \\ \\
\konst{evalB}\; i \; b =
\mathtt{case}\; b\
 \left\{
 \begin{array}{lcl}
    \konst{bLoc}\; \lval & \Rightarrow & \konst{toB}(\theta^i(\lval)) \\
    \konst{bLit}\; b & \Rightarrow & b \\
    \neg b & \Rightarrow & \neg(\konst{evalB} \; b)  \\
    b_1 \lor b_2 & \Rightarrow & \konst{evalB}\; i \;b_1 \lor \konst{evalB}\; i \;b_2   \\
    b_1 \land b_2 & \Rightarrow & \konst{evalB}\; i \;b_1 \land \konst{evalB}\; i \;b_2   \\
    e_1 = e_2 & \Rightarrow & \konst{eval} \;e_1 = \konst{eval}\; i \;e_2   \\
    e_1 < e_2 & \Rightarrow & \konst{eval} \;e_1 < \konst{eval}\; i \;e_2
  \end{array}
 \right.
\end{array}
\]
\caption{Expression evaluation in the context of a stream on environments.}
\label{fig:eval}
\end{figure*}

The initial environment stream only contains mappings for primary inputs along the entire stream. Stepping the component updates the current environment. In other words, at the $i^\mathrm{th}$ step, $\theta^i$ is updated with the result of each component evaluation. Stepping a component is recursively defined similar to the eval functions but track the path for the environment binding.  

\end{document}


