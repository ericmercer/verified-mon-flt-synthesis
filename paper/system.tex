A \emph{system} is a collection of \emph{components}, \emph{connections} between components, a \emph{scheduler} to order execution, and a \emph{system environment} for primary inputs. The computation for the component in this work is defined entirely by a step function:
\[
\konst{stepFn} : \mathit{inports} \times \mathit{stateVars} \to \mathit{stateVars}
\]
The scheduler \emph{activates} components in some order. Activating a component is defined as follows: 
\[
\begin{array}{ll}
 \mathit{inportVals} & = \konst{readInputs}(); \\
 (v_1,\ldots,v_k) & = \konst{readState}() ; \\
 ({v_1}',\ldots,{v_k}') & = \konst{stepFn} (\mathit{inportVals},\mathit{stateVars}) ; \\
 \multicolumn{2}{l}{\konst{writeState}({v_1}',\ldots,{v_k}');} \\
 \multicolumn{2}{l}{\konst{writeOutputs}({v_1}',\ldots,{v_k}');} \\
\end{array}
\]
It is assumed that the scheduler follows some sensible partial order of component activation and allows each component sufficient time for its computation.
