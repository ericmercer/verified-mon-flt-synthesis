\begin{figure}
  \[
    \begin{array}{rcl}
      \mathit{c}    & = & \konst{input}\ [(f : \tau)\ldots] \\
                    &   & \konst{output}\ [(f : \tau)\ldots] \\
                    &   & \konst{eq}\; f : \tau := \mathit{exp} \\
                    &   & \konst{policies}\ [\mathit{bexp}\ldots]\\
      
      \lval & = & f \mid \lval \, [ \mathit{exp} ]
                          \mid \lval . f \\ \\

      f             & = & \mathit{varName} \\ \\

      \mathit{exp}  & = & \konst{Loc}\; \lval
                          \mid \konst{nLit}\; \konst{nat}
                          \mid \mathit{constname} \\
                    & | & \mathit{exp} + \mathit{exp}
                          \mid \mathit{exp} * \mathit{exp} \\
                    & | & (\mathit{exp}\ \rightarrow\ \mathit{exp}) \\
                    & | & (\konst{pre}\ \mathit{exp}) \\
                    & | & (\konst{ite}\ \mathit{bexp}\ \mathit{exp}\ \mathit{exp})\\
                    & | & \mathit{bexp} \\ \\
                          
      \mathit{bexp} & = & \konst{bLoc}\; \lval
                          \mid  \konst{bLit}\; \konst{bool}
                          \mid  \neg \mathit{bexp}
                          \mid  \mathit{bexp} \land \mathit{bexp} \\
                    & | & \mathit{exp} = \mathit{exp} 
                    \mid  \mathit{exp} < \mathit{exp}
\end{array}
\]
\caption{Syntax for high-assurance component specifications.}
\label{fig:syntax}
\end{figure}

The specification language for high-assurance components is in (see \figref{fig:syntax}). It is stylized contracts that defines the inputs, outputs, local values, and policies for each output. Each output policy must be invariant over the input stream and completely define the meaning of its corresponding output. A type $\tau$ is a contiguity type which is a self-describing dependent type specification (add citation). An $\lval$ is a reference to an \emph{L-value} from compiler concepts and is an expression that can appear on the left hand side of an assignment. Alpha renaming is assumed so that every $\lval$ is unique.

An environment, $\theta: \lval \mapsto \konst{string}$ binds L-values to strings. $\Delta : \konst{string} \to \mathbb{N}$ binds constant names to numbers. Functions $\konst{toN}:\konst{string}\to\mathbb{N}$ and $\konst{toB}:\konst{string}\to\konst{bool}$ interpret byte sequences to numbers and booleans, respectively. 

The semantics are synchronous data-flow on a single clock defined over a sequence of environments where $\theta^i$ is the $i^\mathrm{th}$ environment in the stream. Expression evaluation is defined in the context of the environment stream in \figref{fig:eval}.

\begin{figure*}
\[
\begin{array}{l}
\konst{eval}\; i\; e =
\mathtt{case}\; e\
 \left\{
 \begin{array}{lcl}
    \konst{Loc}\; \lval & \Rightarrow & \konst{toN}(\theta^i(\lval)) \\
    \konst{nLit}\; n & \Rightarrow & n  \\
    \mathit{constname} & \Rightarrow & \Delta(\mathit{constname})  \\
    e_1 + e_2 & \Rightarrow & \konst{eval}\; i \; e_1 + \konst{eval}\; i \; e_2  \\
    e_1 * e_2 & \Rightarrow & \konst{eval}\; i \; e_1 * \konst{eval}\; i \; e_2  \\
    e_1 \rightarrow e_2 & \Rightarrow &  \mathbf{if}\; i = 0\; \mathbf{then}\; \konst{eval}\; i \; e_1\; 
                                         \mathbf{else}\; \konst{eval}\; i \; e_2 \\
    (\konst{pre}\; e) & \Rightarrow &  \konst{eval}\; i-1 \; e
  \end{array}
 \right.
 \\ \\
\konst{evalB}\; i \; b =
\mathtt{case}\; b\
 \left\{
 \begin{array}{lcl}
    \konst{bLoc}\; \lval & \Rightarrow & \konst{toB}(\theta^i(\lval)) \\
    \konst{bLit}\; b & \Rightarrow & b \\
    \neg b & \Rightarrow & \neg(\konst{evalB} \; b)  \\
    b_1 \lor b_2 & \Rightarrow & \konst{evalB}\; i \;b_1 \lor \konst{evalB}\; i \;b_2   \\
    b_1 \land b_2 & \Rightarrow & \konst{evalB}\; i \;b_1 \land \konst{evalB}\; i \;b_2   \\
    e_1 = e_2 & \Rightarrow & \konst{eval} \;e_1 = \konst{eval}\; i \;e_2   \\
    e_1 < e_2 & \Rightarrow & \konst{eval} \;e_1 < \konst{eval}\; i \;e_2
  \end{array}
 \right.
\end{array}
\]
\caption{Expression evaluation in the context of a stream on environments.}
\label{fig:eval}
\end{figure*}

The initial environment stream only contains mappings for the inputs along the entire stream. Stepping the component updates the current environment and checks the policies. In other words, at the $i^\mathrm{th}$ step, $\theta^i$ is updated with the result of the sequential evaluation of the \konst{eq}-statements in the specification, which are assumed to be in dependency order, and then the policies are checked for invariance. 

\begin{comment}
  A \emph{system} is a collection of \emph{components}, \emph{connections} between components, a \emph{scheduler} to order execution, and a \emph{system environment} for primary inputs. The computation for the component in this work is defined entirely by a step function:
\[
\konst{stepFn} : \mathit{inports} \times \mathit{stateVars} \to \mathit{stateVars}
\]
The scheduler \emph{activates} components in some order. Activating a component is defined as follows: 
\[
\begin{array}{ll}
 \mathit{inportVals} & = \konst{readInputs}(); \\
 (v_1,\ldots,v_k) & = \konst{readState}() ; \\
 ({v_1}',\ldots,{v_k}') & = \konst{stepFn} (\mathit{inportVals},\mathit{stateVars}) ; \\
 \multicolumn{2}{l}{\konst{writeState}({v_1}',\ldots,{v_k}');} \\
 \multicolumn{2}{l}{\konst{writeOutputs}({v_1}',\ldots,{v_k}');} \\
\end{array}
\]
It is assumed that the scheduler follows some sensible partial order of component activation and allows each component sufficient time for its computation.
\end{comment}