This section formalizes AGREE's assume-guarantee reasoning by defining the set of verification conditions that AGREE must check.
The verification conditions succinctly summarize what in means when AGREE proves that a system implementation satisfies it specification.
These conditions are useful in understanding how the semantics in synthesis differ from those in the assume-guarantee reasoning in \secref{sec:semantics}.

This section then introduces the AGREE contract specification language through the cyber-hardened system of \figref{fig:hardened}.
It details the specification of that top-level system interface.
It then discusses the specifications for the filter and the monitor.
These two specifications are those that are synthesized to equivalent CakeML by SPLAT.

\subsection{Verification Conditions}

\newcommand{\globally}{\ensuremath{\mathbf{G}}}
\newcommand{\historically}{\ensuremath{\mathbf{H}}}
\newcommand{\assumes}{\ensuremath{A}}
\newcommand{\guarantees}{\ensuremath{P}}
\newcommand{\dispatch}{\ensuremath{\mathit{dispatch}}}
\newcommand{\complete}{\ensuremath{\mathit{complete}}}
\newcommand{\same}[1]{\ensuremath{\mathit{same}(#1)}}
\newcommand{\inputs}{\ensuremath{I}}
\newcommand{\outputs}{\ensuremath{O}}
\newcommand{\system}{\ensuremath{S}}
\newcommand{\components}{\ensuremath{C}}
\newcommand{\component}{\ensuremath{c}}
\newcommand{\schedule}{\ensuremath{\phi}}
\newcommand{\valid}{\ensuremath{\mathit{valid}}}
\newcommand{\dpred}{\ensuremath{\delta^\phi}}
\newcommand{\dispred}{\ensuremath{\mathbb{D}^\phi}}
\newcommand{\compred}{\ensuremath{\mathbb{C}^\phi}}
\newcommand{\dispredp}{\ensuremath{\mathbb{D}^{\phi\prime}}}
\newcommand{\compredp}{\ensuremath{\mathbb{C}^{\phi\prime}}}

The AGREE specification language is based on stream concepts, and
operators, from the Lustre language \cite{10.1145/41625.41641}. Thus
the setting is synchronous dataflow where the inputs and outputs of
components are streams, and contracts express relationships between
input and output streams. When considering a system of components,
data flows through the components in dependency order, with inputs
being propagated to outputs through all contracts until they stabilize
(can't propagate further). Therefore, the subcomponent contracts, and
thus the top-level model, must be acyclic. (An apparent syntactic
cycle, where a component is linked back to itself, may be broken
temporally by inserting delay elements.)  Once the data propagation
has stabilized, the model proceeds to the next input data in the input
streams. The semantics do not model computation or communication
delay. The output of one contract is seen at the input of any
downstream contract in the same step of the input data stream.

From the system and component contracts, AGREE generates a set of
verification conditions to show that a system's component
implementation is correct~\cite{agree2013}.  The AGREE model checker
is then invoked to prove or disprove the verification
conditions. Contracts and verification conditions are expressed in
\emph{past-time linear temporal logic} (PLTL).\footnote{KLS: citation
needed.}  PLTL is a logic enhanced with temporal operators able to
reason about the truth values of formulas through time.  Its semantics
are defined relative to a point in time $i$ and a finite trace of
system states $\pi = s_0, s_1, \ldots, s_i$.

The two PLTL operators necessary for the AGREE generated verification
conditions are $\globally$ (globally) that looks forward in time along
the trace and $\historically$ (historically) that looks backward in
time along the trace.  These are defined as
\begin{eqnarray*}
 (\pi, i) \models \globally(f) & \iff & \forall j \ge i, (\pi, j) \models f \\
(\pi, i) \models \historically(f) & \iff & \forall 0 \le j \le i, (\pi, j) \models f
\end{eqnarray*}
The $\models$-operator is read as \emph{satisfies}.  A trace at a
moment in time satisfies $\globally(f)$ if and only if it satisfies
$f$ in the current and all future states of $\pi$.  $\globally(f)$ is
invariant from the current moment into the future and $\historically$ is
invariant from the beginning of the trace to the current moment.

A \emph{system} $\system = (\inputs, \outputs, \assumes,
\guarantees,C)$, where $\inputs$ is the input set, $\outputs$ is the
output set, $\assumes$ is the set of assumptions, $\guarantees$ is the
set of guarantees, and $C$ are subcomponents.  A subcomponent
$\component$ is, hierarchically, also a system, and may be designated
by its own tuple $(\inputs_\component, \outputs_\component,
\assumes_\component, \guarantees_\component, C_c)$.  From the
components and their connections, $\mathbb{I}_\component$ is defined
to be the set of components providing input to some component
$\component$ in the system, and $\mathbb{O}$ is defined to be the set
of components that provide the output for the system.  A system $S$ is
\emph{correct} if and only if for all components $c \in C$ the
following two verification conditions hold:
\begin{equation}
            \globally(\historically(\assumes \wedge
            \bigwedge_{\component^\prime \in \mathbb{I}_\component} P_{\component^\prime})
            \implies \assumes_\component)
\end{equation}
\begin{equation}
            \globally(\historically(\assumes \wedge
            \bigwedge_{\component^\prime \in \mathbb{O}} \guarantees_{\component^\prime})
            \implies \guarantees)
\end{equation}
Condition (1) verifies the input assumptions on each component under
the system assumptions and upstream component guarantees.  It checks
if the component guarantees and system assumptions are strong enough
to imply input assumptions on all immediate downstream components.
Condition (2) checks the output guarantees of the system under the
system assumptions and component guarantees that provide the output.
It checks if the guarantees on components providing primary outputs
are strong enough to imply the system guarantees.

If all the verification conditions hold (AGREE uses $k$-inductive
model checking to automatically prove or disprove each generated
verification condition), then the system is said to be \emph{correct},
meaning that the system composition meets input assumptions at each
input as well as the guarantees on the system output. A consequence of
this result is that $\globally(\historically(\assumes) \implies
\guarantees)$ holds for the system contract.

The expanded property lists in \figref{fig:example-certificate} and
\figref{fig:hardened-certificate} are the results from verifying or
disproving the above verification conditions.  The additional
unexpanded results at the bottom of the figures prove
\emph{self-consistency} in the contracts.  It is not uncommon to
accidentally write contracts that are self-contradicting.  For
example, a contract may guarantee an output be two different values in
the same moment of time.  AGREE generates additional verification
conditions that prove each component contract, and the composition of
contracts, self-consistent.

\subsection{Syntax and semantics of AGREE}
\label{agree-semantics}

We now give an overview of a formal model for AGREE. This provides a
setting in which we are able to relate the contract correctness
results discussed above, obtained via model-checking, with the code
generated from high-assurance contracts.  The syntax of AGREE is
essentially that of quantifier-free first order predicate logic
supplemented with a few temporal operators. The terms (\emph{e}) are
arithmetic expressions built from variables ($v$) and numeric and
boolean literals ($c$), while formulas (\emph{b}) are built using
logical connectives from atomic formulas (\emph{a}) based on the
familiar comparison operators.
\[
\begin{array}{rcl}
e & ::= & v \mid c \mid e \;\set{+,*,/}\; e \\
a & ::= & e\; \set{=,<}\; e \\
b & ::= & v \mid c \mid a \mid \neg b
            \mid b \; \set{\land,\lor,\imp,\iff}\; b
\end{array}
\]

There is also a conditional, $\itelse{b}{(-)}{(-)}$, for both terms
and formulas. Lastly, there are temporal operators $\konst{pre}(-)$,
\emph{delay} $(-) \to (-)$, and $\konst{Hist}(-)$.

The semantics of terms and formulas is in terms of \emph{streams of
values}. Values encompass at least booleans and numbers, but can be
readily extended to include records and arrays. A value stream is a
total function from time (natural numbers) to values:
\[
 \konst{stream} = \mathbb{N} \to \konst{value}
\]
Given an \emph{environment} $E : \konst{name} \mapsto \konst{stream}$
binding variable names to value streams, the semantics $\sem{-}^E_t$
of terms and formulas defines the meaning of compound syntax in terms
of the meaning of subexpressions. The value of a variable $v$ at time
$t$ is found by looking up the stream bound to $v$ in $E$ (call it
$s$) and returning $s_t$. Some clauses of the semantics follow,
omitting the temporal operators:
\[
\begin{array}{rcl}
\sem{v}^E_t & = & E(v)(t) \\
\sem{c}^E_t & = & c \\
\sem{e_1 + e_2}^E_t & = & \sem{e_1}^E_t + \sem{e_2}^E_t \\
   & \cdots & \\
\sem{b_1 \land b_2}^E_t & = & \sem{b_1}^E_t \land \sem{b_2}^E_t \\
   & \cdots & \\
\end{array}
\]

The temporal operators deal with time in more significant ways. The
value of $\konst{pre}(e)$ at time $t$ is the value of $e$ at time
$t-1$ (at time zero, \konst{pre} is undefined).  A delay $e_1 \to e_2$
temporally ``shifts'' $e_2$ by means of prepending the first element
of $e_1$ to it.

\[
\begin{array}{rcl}
\sem{\konst{pre}(e)}^E_t & = & \sem{e}^E_{t-1}, \mathrm{when}\ t > 0 \\
\sem{e_1 \to e_2}^E_t & = & \itelse{t=0}{\sem{e_1}^E_0}{\sem{e_2}^E_t} \\
\sem{\konst{Hist}(b)}^E_t & = & \forall n \leq t.\; \sem{b}^E_n
\end{array}
\]

Although basic, these definitions can be used to define higher-level
operators from PTLTL, such as \konst{Once} and \konst{Since}.

\subsection{Code contracts}
\label{code-contracts}

Generally, AGREE specifications do not describe the computation that a
component performs. This is entirely by design: AGREE is intended to
reason about component behavior solely at the specification
level. However, the syntax of AGREE specifications provides enough
expressiveness to support the notion of a \emph{code contract}: a
contract from which an implementation can be extracted. First we must
discuss a class of guarantees---\emph{output guarantees}---which
determine the values on all output ports of a component.

\begin{definition}[Output guarantee]
An \emph{output guarantee} is a stylized guarantee that fully
specifies the data written to an output port. There are three
possibilities according to whether the output port $p$ is
a \konst{data} port, an \konst{event} port, or an \konst{event data}
port:
\[
\begin{array}{ll}
\konst{data}: &  p = \mathit{e} \\
\konst{event}: &  \konst{event} (p) = \mathit{b} \\
\konst{event data}: & \itelse{b}{\konst{event} (p) \land p = e}{\neg \konst{event}(p)} \\
\end{array}
\]
\end{definition}

Informally, a code contract treats its \konst{eq} ``statements'' as
defining a list of assignments to state variables, and its output
guarantees as directives for producing output.

\begin{definition}[Code contract] A
  leaf component of the form $(I,O,A,P,\emptyset)$ is a
  \emph{code contract} if $\mathit{Eqs} \cup G \subseteq P$, where
\[\mathit{Eqs} = \set{v_1 = e_1, \cdots , v_n = e_n} \] is a non-empty set
of \konst{eq} statements and $G$ is the set of output guarantees, one for
each element of $O$. In the interpretation as code, the order of
elements of $\mathit{Eqs}$ is important, and is simply taken to be the
occurrence order of the \konst{eq} statements in the syntax. Thus we
will work with the
\emph{list} of equations $\mathit{Eqs} = [v_1 = e_1; \cdots ; v_n = e_n]$.
\end{definition}


\subsection{Contract Language Specification}
\newsavebox{\sw}
\begin{lrbox}{\sw}
\begin{lstlisting}[style=agree,numbers=left,escapeinside={*}{*}]
eq req : bool = event(AutomationRequest);*\label{line:sw-event-def-start}*
eq avl : bool = event(AirVehicleLocation);
eq wp : bool = event(Waypoint);
eq strt: bool = event(Start);
eq alrt : bool = event(Alert);*\label{line:sw-event-def-end}*

assume "Automation requests are well-formed" : *\label{line:sw-assume-1}*
  req => WELL_FORMED_AUTOMATION_REQUEST(AutomationRequest);
assume "Air vehicle locations are well-formed" : *\label{line:sw-assume-2}*
  avl => WELL_FORMED_WAYPOINT(AirVehicleLocation);    
assume "One automation request in flight at a time" : *\label{line:sw-assume-3}*
  true -> 
  (req => pre(Historically(not req) or Since(not req, strt)));
      
guarantee "Waypoints coincide with air vehicle locations":
  wp => avl;
guarantee "Starts include a new waypoint" :
  strt => wp;
guarantee "Waypoints are well-formed" : 
  wp => WELL_FORMED_WAYPOINT(Waypoint);
guarantee "Starts within one cycle of requests if not alerting" :
  (strt => ((not alrt) and req)) -> 
  (strt => ((not alrt) and (req or pre(req))));
guarantee "Alert if not started within one cycle of requests" :
    true -> ((pre(req and not strt) and not strt) => alrt);
guarantee "Once alerted always alerted" :
  not alrt or (Once(alrt) and alrt);
\end{lstlisting}
\end{lrbox}

\begin{figure}
  \begin{center}
    \scalebox{0.62}{\usebox{\sw}}
  \end{center}
  \caption{The SW component contract.}
  \label{fig:sw}
\end{figure}

The AGREE specification language uses stream concepts, and operators, from the Lustre language \cite{10.1145/41625.41641}.
As with Lustre, the semantics are synchronous dataflow where the inputs, outputs, and expressions are data streams that comply with the input assumptions.
Contracts are evaluated in dependency order with inputs being propagated to outputs through all contracts until they stabilize; as such, the contracts, and thereby the top-level model, must be acyclic.\footnote{An apparent syntactic cycle, where a component is linked back to itself, may be broken temporally by inserting delay elements.}
Once the contracts have stabilized, the model takes a synchronous step to the next input data in the stream.
The semantics do not model computation or communication delay.
The output of one contract is seen at the input of any downstream contract in the same step of the input data stream.
The language is best introduced through example.

The AGREE specification for the SW component in the example of Section~\ref{sec:example} is given in \figref{fig:sw}.
The specification uses \texttt{eq} statements to define variables local to the contract specification.
For example, \lineref{line:sw-event-def-start} defines the \texttt{req} variable to be equivalent to the \texttt{event(AutomationRequest)} expression.

All the named ports in the corresponding AADL component are in the scope of the specification, and there are additional implicit boolean \emph{event} inputs (or outputs) associated with event ports.
An \texttt{event} expression refers to that implicit input (or output) boolean value and is true when data is placed on the named port and false otherwise.
\linesref{line:sw-event-def-start}{line:sw-event-def-end} create local variables that are true when data is present on the corresponding event ports for the component.
The local variables here are purely for convenience in writing the specification.

Assumptions are best understood as being evaluated on the pre-state of a component while guarantees are evaluated on the post-state.
Contract languages such as Eifle and Dafny make this pre-state assume evaluation and post-state guarantee evaluation part of the semantics, and it relies on operators such as \texttt{old} it indicate when a guarantee should use a pre-state value in evaluating a guarantee.
AGREE differs from these languages in that it evaluates both the assumptions and guarantees in the post-state of the component.
But like other languages it relies on the operator \texttt{pre} to refer to a pre-state evaluation in both the assumptions and guarantees.
Since streams are associated with all expressions in AGREE meaning that they are defined through time, the \texttt{pre} operator simply returns the previous value of the enclosed expression (e.g., it looks back one step in time).


The \texttt{assume} statement is a string description followed by a predicate.
Those on \lineref{line:sw-assume-1} and \lineref{line:sw-assume-2} are implications requiring that when data is present it is well-formed.
The well-formed predicates themselves are defined elsewhere using AGREE functions.
The assumption on \lineref{line:sw-assume-3} uses the \emph{followed-by} operator, \texttt{->}, and the \texttt{pre} operator to constrain when requests can arrive at the input.


The followed-by expression defines what happens at the first instance in the stream, time 0, and then what follows after. 
Here the assumption is \texttt{true} at time 0, and then takes on the truth value of the implication in all future instances.
The implication uses the \texttt{pre} expression in its left operand. 
In the example, it is guarded by the followed-by since expressions are undefined before time 0.
So in the first instance at time 0, the assumption is \texttt{true}, and then at every instance after, it depends on when the request arrives.

The stream semantics in AGREE mean that it is possible to use PLTL operators that look back in time.

REPLACE ALL: 
The \emph{Alert if start is not bounded relative to a request} guarantee is an invariant on the expression \texttt{policy or since}, meaning that either the policy holds or the alert is sounding.
The \texttt{policy} is defined by two local values: \texttt{current} and \texttt{previous}.
The \texttt{current} value is asserted when in the current time step there is a request with a response, or there is no request and no response.

The value of \texttt{previous} in the current time step relies on values from the previous time step.
The \texttt{->} operator designates initialization, as the previous time step is undefined in the first step of the system.
The left operand to the operator is the initial value of \texttt{previous} at start, which in this example is \texttt{(req and not rsp)}, because seeing a request with no response is inconclusive in the first step of the system.
The right operand is the value of \texttt{previous} after the initial step.
Here the \texttt{pre} operator refers to the value of the expression \texttt{(req and not rsp)} in the prior time step, \texttt{previous} is true if the previous time step made a request without a matching response and the current time step has the matching response to that request with no new request.

The value of \texttt{since} in \emph{Alert if start is not bounded relative to a request} relies on its own value in the previous time step.
The intuitive reading of the expression is that the alert has been true since the time when it first sounded.
The first \texttt{alrt} sets \texttt{since} to true, and once the value of \texttt{since} is true, that value persists as long as \texttt{alrt} holds.
The \emph{Alert if start is not bounded relative to a request} guarantee defines one requirement of a cyber-hardened system implementation.
Together with the other guarantees, the contract models the expected input and output of the system as a whole.

\begin{figure}
  \begin{center}
    \begin{tabular}{c}
      \includegraphics[scale=0.3]{dialogue.png}
    \end{tabular}
  \end{center}
  \caption{Wizard for automatically transforming the model with a filter.}
  \label{fig:dialogue}
\end{figure}

\newsavebox{\flt}
\begin{lrbox}{\flt}
\begin{lstlisting}[style=agree,numbers=left]
eq policy : bool = 
  WELL_FORMED_AUTOMATION_RESPONSE(Input);

guarantee Filter_Output
  "Filter output is well-formed" :
  if event(Input) and policy then 
    event(Output) and Output = Input
  else not event(Output);
\end{lstlisting}
\end{lrbox}

\newsavebox{\mntr}
\begin{lrbox}{\mntr}
\begin{lstlisting}[style=agree,numbers=left]
const is_latched : bool = true;
const MAX_LATENCY : int = 1;
    
eq rsp : bool = event(Response);
eq req : bool = event(Request);

eq isPending : bool = Since(not rsp, req and not rsp);
eq latency : int = 0 -> (if req then 0 else pre(latency) + 1);

eq policy : bool = (rsp => req) ->
                   (    (isPending => latency < MAX_LATENCY)   
                    and (rsp => (req or pre(isPending))));
eq alert : bool = (not policy) -> 
                  ((is_latched and pre(alert)) or not policy);

assume "One outstanding request at a time" :
  (true -> (req => not pre(isPending))); 
                          
guarantee "Alert port tracks alert variable" :
  event(Alert) = alert;
guarantee "Output if not alerted" :
  if (not(alert) and rsp) then
    event(Output) and (Output = Response)
  else
    not (event(Output));    
\end{lstlisting}
\end{lrbox}

\begin{figure}
  \begin{center}
    \begin{tabular}{c}
      \scalebox{0.62}{\usebox{\flt}} \\
    \end{tabular}
  \end{center}
  \caption{Contract specification for high-assurance filter.}
  \label{fig:filter}
\end{figure}

\begin{figure}
  \begin{center}
    \begin{tabular}{c}
      \scalebox{0.62}{\usebox{\mntr}} \\
    \end{tabular}
  \end{center}
  \caption{Contract specification for high-assurance monitor.}
  \label{fig:monitor}
\end{figure}

As noted previously, the original system fails to guarantee the cyber requirements.
BriefCASE provides two transformations to address the failing requirements: inserting a filter and inserting a monitor.
The component is added by selecting the connection in the model where the high-assurance component is to be added, and then choosing the appropriate transformation.
The system designer can provide transform configuration parameters in a wizard, as shown in \figref{fig:dialogue}.
The policy of the high-assurance component can be stated directly in the wizard, or it can be left blank.
In this example, the policy is specified as \texttt{WELL\_FORMED\_AUTOMATION\_RESPONSE(Input)}.
Additionally, because a transformation is ultimately driven by a cyber requirement, BriefCASE updates an embedded Resolute assurance case~\cite{resolute-destion}.
REMOVE: Resolute keeps track of the evidential artifacts necessary for supporting the requirement, and can be run at any time to determine whether those artifacts are valid.

The AGREE contract specification generated by the transform is shown in \figref{fig:filter}.
The guarantee is stylized for synthesis and completely defines the meaning of the output under every possible input.
The resulting AGREE specification for the monitor in this example is shown in \figref{fig:monitor}.
The \texttt{is\_latched} value makes the alert persistent, meaning that once the alert is raised, it is always raised.
This behavior is one of the several options available in the dialogue.
The definition for \texttt{policy} is taken by the system developer from the contract in \figref{fig:sw}.
As before, the guarantees for the outputs are autogenerated by the tool and completely define each output under every possible input.
