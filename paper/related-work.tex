Assume-guarantee reasoning for compositional verification in reactive systems is well-studied \cite{10.1007/978-3-642-28891-3_13, composition1, 10.1145/2658982.2527272, 10.1007/978-3-319-17524-9_7}. Automated proofs of realizability for assume-guarantee reasoning are useful for engineers implementing components in the system \cite{10.1007/978-3-319-17524-9_13, 10.1007/978-3-319-29613-5_7}. Algorithms for actual component synthesis for Lustre models using k-induction or IC3/PDR provide an automated path from the assume-guarantee reasoning to an actual satisfying node implementation \cite{katis2017synthesis, 10.1007/978-3-319-89963-3_10}. These synthesis algorithms generate code in the Lustre modeling language but do not provide a path to a low-level implementation that could be fielded.

Contracts are similar to assume-guarantee reasoning but are targeted to programming languages. Contracts are often more expressive than assume-guarantee reasoning and can not only be stateful but higher-order \cite{10.1145/583852.581484}. As contracts are often written in the target language, synthesis is not a problem for monitoring but comes with significant overhead \cite{10.1007/978-3-642-28869-2_11}. A monitor for a contract can be removed when it can be statically proved that the code preserves the contract under all possible inputs and executions \cite{10.1145/3158139}.