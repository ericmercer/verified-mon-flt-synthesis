Assume-guarantee reasoning for compositional verification in reactive systems is well-studied \cite{10.1007/978-3-642-28891-3_13, agree2013, 10.1145/2658982.2527272, 10.1007/978-3-319-17524-9_7}. Automated proofs of realizability for assume-guarantee reasoning are useful for engineers implementing components in the system \cite{10.1007/978-3-319-17524-9_13, 10.1007/978-3-319-29613-5_7}. Algorithms for actual component synthesis for Lustre models using k-induction or IC3/PDR provide an automated path from the assume-guarantee reasoning to an actual satisfying node implementation \cite{katis2017synthesis, 10.1007/978-3-319-89963-3_10}. These synthesis algorithms generate code in the Lustre modeling language but do not provide a path to a low-level implementation that could be fielded.

Contracts are similar to assume-guarantee reasoning but are targeted to programming languages. Contracts are often more expressive than assume-guarantee reasoning and can not only be stateful but higher-order \cite{10.1145/583852.581484}. As contracts are often written in the target language, synthesis is not a problem for monitoring but comes with significant overhead \cite{10.1007/978-3-642-28869-2_11}. A monitor for a contract can be removed when it can be statically proved that the code preserves the contract under all possible inputs and executions \cite{10.1145/3158139}.

Lustre langugage definition and semantics \cite{10.1145/41625.41641,97300}. Classical clock directed compilation for synchronous dataflow \cite{10.1145/1379023.1375674,10.1145/2345141.2248426}.

A denotational Kahn semantics for synchronous dataflow is in \cite{10.1007/978-3-540-45212-6_10}.

State machine semantics can be added to synchronous dataflow \cite{10.1145/1086228.1086261}. The idea is to translate imperative constructs into equivalent synchronous dataflow constructs. The resulting Lustre can them be compiled to target platform. The goal is to provide a seamless connection between pure datalflow and pure control design.

There is a fully verified compiler that takes Lustre and turns it into a binary executable \cite{10.1145/3140587.3062358}. The first compilation pass turns synchronous dataflow in Lustre to assembly. The compilation is specified and verified in Coq. The key is in combining infinite sequences of dataflow models with incremental manipulation of memories akin to an imperative model. CompCert is used on the backend to create the final rendered executable.