The DARPA CASE program has created tools for systems engineers to
integrate cyber-vulnerability analysis and mitigation into their
development workflow. The resulting BriefCASE tool suite includes
analysis tools for generating cyber requirements, cyber resiliency
tools for addressing the requirements, verification tools for ensuring
design correctness, and synthesis tools for generating provably
correct code. Several of the BriefCASE transforms (filter, monitor,
gate) insert components into the model whose behavior can be formally
specified using the AGREE language.  The SPLAT tool can then
automatically generate CakeML implementations for these components,
along with proofs of correctness for assurance that the implementation
satisfies the specification.
%
BriefCASE was applied to a full-scale case study using the Air Force
Research Laboratory's OpenUxAS software, exercising a range of
built-in cyber resiliency mitigations to meet cyber-requirements.  The
size and scale of the study suggests BriefCASE meets the complexity
demands of real-world design.

Most recently, we have applied BriefCASE to the design of
an application using the Collins Common Avionics Architecture System
(CAAS)~\cite{caas} on the CH-47F Chinook helicopter as part of the
DARPA CASE program.  Other ongoing and future work includes
adding support for uninterpreted functions, mechanizing the
correctness-of-synthesis proofs in HOL4, and lifting the proof results
to infinite streams.
