The DARPA CASE program is creating an MBSE environment for designers to integrate cyber-vulnerability analysis and mitigation. The resulting BriefCASE tool suite in OSATE includes analyses to add cyber requirements to systems and architectural transformations on systems to satisfy cyber requirements. The filter transformation prevents malformed data from being propagated downstream while the monitor transformation checks temporal properties to detect malicious behavior. The components are specified by corresponding auto-generated AGREE contracts that only require policy annotations taken from the cyber requirements. The specifications are automatically synthesized to CakeML and that synthesis preserves the meaning of the specifications.

The BriefCASE analyses and transformations are applied to a full-scale case study on the Air Force Research Laboratory's OpenUxAS services requiring several filters and monitors to meet cyber-requirements. A video shows the synthesized system running on the ODROID platform with seL4. The size and scale of the study suggests BriefCASE meets the complexity demands of real-world design.

Ongoing work is applying BriefCASE in the Collin' Common Avionics Architecture System (CAAS), \cite{caas}, to protect against \emph{Automatic Dependent Surveillance-Broadcast} spoofing. That work adds support for uninterpreted functions. Other ongoing work is mechanizing the synthesis proof in HOL4 and lifting the proof results to infinite streams.